% Preambulo.

\documentclass[14pt,letterpaper,twoside]{extbook} % Texto a 14 puntos, tamaño carta, impreso a doble cara, tipo libro. Por defecto.
\usepackage[utf8]{inputenc} % Para escribir con acentos (Español).
\usepackage[T1]{fontenc} % Soporte opcional para la tipografía Linux Libertine. Mejora la tilde en los acentos.
\usepackage[spanish,mexico]{babel} % Idioma (Español México).
\usepackage[tt=false]{libertine} % Tipografía Linux Libertine para todo el documento. *tt=false - Desactiva las fuentes monoespaciadas.
%\usepackage[sfdefault]{libertine} % Tipografía Linux Biolinum para todo el documento. Útil para pantalla (día y noche)
%\usepackage{hanging} % Crea sangrías francesas. Muy útil cuando queremos nuestras referencias bibliográficas en esta forma.
%\usepackage[hang]{footmisc} % No sangra las notas al pie de página.
%\setlength\footnotemargin{8.5pt} % Espacio de separación entre el número (ancla) y el texto en las notas a pie de página.

% Índice de contenido (Estilo Robert Bringhurst).

%\usepackage{titletoc}
%\titlecontents{chapter}[2.25em]{}%
%{\contentslabel{2.25em}}{}%
%{\hspace{0.5em}{|}\hspace{0.5em}\oldstylenums{\thecontentspage}}
%\titlecontents{section}[3.45em]{}%
%{\contentslabel{2.25em}}{}%
%{\hspace{0.5em}{|}\hspace{0.5em}\oldstylenums{\thecontentspage}}
%\titlecontents{subsection}[5em]{}%
%{\contentslabel{2.25em}}{}%
%{\hspace{0.5em}{|}\hspace{0.5em}\oldstylenums{\thecontentspage}}

% Opciones avanzadas en microtipografía (interlineado, interletrado, protrusión, etcétera).

\usepackage[activate={true,nocompatibility},final=true,babel=true,tracking=true,kerning=true,spacing=false,factor=1100,stretch=10,shrink=10]{microtype}
\SetTracking[no ligatures=q]{encoding=*,shape=sc}{30} % Activa el espaciado de las letras en las versalitas (3 %). Hemos desactivado la ligatura 'q' en las versalitas pues afea las palabras 'enfoque', 'Querétaro', arqueológicas, etcétera.

% activate={true,nocompatibility} - Activa la protrusion y la expansión.
% final - Activa la microtipografia; usar "draft" para desactivar.
% tracking=true - Activa el espaciado (interletraje o interletrado) en palabras de varios caractéres.
% kerning=true - Activa el espaciado únicamente en un par de caractéres. Ejemplos: AV, Ta o Yo.
% spacing=false - Elimina los espacios elásticos (blanco) empleados por defecto en LaTeX en títulos, subtítulos, subsubtítulos y texto principal.
% factor=1100 - Agrega 10% a la protrusión (1000 por defecto).
% stretch=10, shrink=10 - Reduce el estrechamiento y el encogimiento (20/20 por defecto)

\usepackage[includehead,includefoot,headsep=1.5cm,footskip=2cm,left=4cm,right=3cm,top=3cm,bottom=3cm]{geometry} % Para los márgenes de página y la separación del encabezado y el pie.
\usepackage{fancyhdr} % Encabezados y pies de página personalizados.
\usepackage{emptypage} % Elimina el encabezado y número de página si una página no tiene texto.
\usepackage[all]{nowidow} % Correción de viudas
\usepackage{imakeidx} % Para la creación de nuestro Índice General y analítico.

\makeindex[name=nombres,title=Índice onomástico,columns=2]
\makeindex[name=lugares,title=Índice toponímico,columns=2]

\usepackage{etoolbox} % Remueve todas las negritas del Índice general, excepto el título :-( Ugly hack para el paquetes imakeidx.

% Remueve 'Capítulo N' al principio de cada capítulo.

\makeatletter
\def\@makechapterhead#1{%
  \vspace*{50\p@}%
  {\parindent \z@ \raggedright \normalfont
    \interlinepenalty\@M
    \Huge \mdseries #1\par\nobreak
    \vskip 40\p@
  }}
\def\@makeschapterhead#1{%
  \vspace*{50\p@}%
  {\parindent \z@ \raggedright
    \normalfont
    \interlinepenalty\@M
    \Huge \mdseries  #1\par\nobreak
    \vskip 40\p@
  }}
\makeatother

\makeatletter
\patchcmd{\l@chapter}{\bfseries}{}{}{}
\makeatother

\usepackage{graphicx} % Para insertar figuras en nuestro documento.
\usepackage{ccicons} % Para insertar iconos de Creative Commons.
\usepackage{hologo} % Para escribir palabras LaTeX, LaTeX 2e, ConTeXt, etcétera, en lenguaje TeX.
%\usepackage{draftwatermark} % Agrega marca de agua al documento.
%\SetWatermarkText{Borrador} % Personaliza la marca de agua. Borrador por defecto.
%\SetWatermarkLightness{0.5} % Establece el color de la marca de agua. Por defecto, el paquete está a 0.8
%\usepackage{fancybox} % Agrega cajas personalizadas al texto.
%\usepackage[grid=true]{eso-pic} % Muestra la cuadrícula de la página.
%\usepackage{showframe} % Muestra la disposición de los elementos de la página. (Encabezado, cuerpo de texto, pie y notas al margen).
\usepackage{float} % Para colocar mejor las imágenes.
%\usepackage[right]{lineno} % Muestra números de líneas (párrafo) del documento. Muy útil cuando estamos revisando ortografía, viudas, huérfanas, etcétera, y deseamos corregir los errores.
%\linenumbers % Agrega líneas de párrafo al documento.
\graphicspath{{imagenes/}} % Carpeta en donde están nuestras imágenes.
\usepackage{pdfpages} % Inserta documentos PDF en LaTeX. Muy últil cuando queremos agregar la portada diseñada en otro programa.

% PDF para pantalla (Día y noche)

\usepackage{xcolor} % Agrega color a elementos de la página (texto, página, etcétera). Útil si vamos a generar un PDF que sera leído en pantalla por la noche y no canse la vista.

% Color durazno (Noche)

%\definecolor{peach}{RGB}{255 218 185} % Definimos nuestro color (Durazno, por defecto).
%\pagecolor{peach} % Según estudios científicos, el color durazno es el ideal para leer un PDF en pantalla por la noche.

% Color amarillo claro (Día)

%\definecolor{amarillo claro}{RGB}{255 255 224} % Definimos nuestro color (Amarillo claro, por defecto).
%\pagecolor{amarillo claro} % Según estudios científicos, el color amarillo claro es el ideal para leer un PDF en pantalla por el día.

\usepackage[bookmarks,breaklinks=true,citecolor=black,colorlinks=true,linkcolor=black,urlcolor=black,pdftitle={La fiesta del Sagrado Corazón de Jesús: un legado de la actividad devocional de la fábrica La Estrella, Tlaxcala},pdfsubject={Licenciatura en Antropología Social, Benemérita Universidad Autónoma de Puebla (BUAP)},pdfauthor={Blanca Irma Alejo Aguilar},pdfkeywords={fiesta mayordomía fabrica Estrella Chiautempan Tlaxcala},pdfproducer={pdflatex 3.14159265-2.6-1.40.17 (TeX Live 2016/Debian)},pdfcreator={Noel Merino Hernández (muxkernel@gmail.com)}]{hyperref} % Índice general con marcadores y enlaces dinámicos.

\urlstyle{rm} % Desactiva las fuentes monoespaciadas y usa las romanas. Hack para el paquete hyperref.

% Sangre, sudor y lágrimas inician aquí.

\begin{document}
\pagenumbering{Roman} % Número de página en romanos.
\setcounter{page}{1} % Inicia la numeración. Esta será la página {1}.
\parindent=5mm % Establece la sangría para todo el documento
\parskip=0mm % Establece el espacio entre párrafos para todo el documento. Por defecto lo dejamos en 0mm.

% Portada.

% ¡Que hueva diseñar una portada en LaTeX! Mejor la creamos con LibreOffice Draw, generamos el PDF y la insertamos como como pagina 1 (Anverso) y 2 (Reverso), para que se conserve la paginación con el paquete pdfpages. Mas tarde pondré el código de la portada para que se genere automáticamente. Esto queda pendiente.

\includepdf{00}

\newpage
\pagestyle{empty}
\null\vfill

% Portadilla.

\pagestyle{empty}
\begin{titlepage}
\begin{center}
\pagenumbering{Roman}
\setcounter{page}{3}
\large Benemérita Universidad Autónoma de Puebla\\
\large Facultad de Filosofía y Letras\\
\large Colegio de Antropología Social\\
\vfill
\large {\textit{La fiesta del Sagrado Corazón de Jesús: un legado de la actividad devocional de la fábrica La Estrella, Tlaxcala}}
\vfill 
\large \textsc{Tesis}
\vfill
\large que para obtener el título de:
\vfill
\large \textsc{Licenciada en Antropología Social}
\vfill
\large Presenta:
\vfill
\large \textsc{Blanca Irma Alejo Aguilar}
\vfill
\large Directora de tesis:
\vfill
\large \textsc{Dra. Leticia Villalobos Sampayo}
\vfill
\large Puebla, México \hfill Marzo, \oldstylenums{2019}
\end{center}
\end{titlepage}

\newpage
\pagestyle{empty}
\null\vfill

\newpage
\pagestyle{empty}
\null\vfill

\newpage
\pagestyle{empty}
\null\vfill
\begin{scriptsize}
\begin{minipage}{7.5cm}

\hspace{-0.6cm} \noindent \copyright \thinspace 2019 Benemérita Universidad Autónoma de Puebla

\hspace{-0.6cm} \noindent \copyright \thinspace 2019 Colegio de Antropología Social

\hspace{-0.6cm} \noindent \copyright \thinspace 2019 Blanca Irma Alejo Aguilar, del texto y las fotos

\hspace{-0.6cm} \noindent \copyright \thinspace 2019 \href{julian2che@gmail.com}{Julián y Víctor Osorno}, de la corrección ortográfica y estilo

\hspace{-0.6cm} \noindent \copyright \thinspace 2019 \href{noel_merino@yahoo.com.mx}{Noel Merino Hernández}, del diseño y maquetación

\hspace{-0.6cm} \noindent \copyright \thinspace 2019 \textsc{inegi}, del mapa en la página \pageref{chiautempan} \\
\end{minipage}
\end{scriptsize}
\\
\begin{scriptsize}
\begin{minipage}{7.5cm}
 
\begin{minipage}{7.5cm}
\href{https://creativecommons.org/licenses/by/4.0/deed.es}{\includegraphics[width=2.5cm]{14}} \\
\end{minipage}

\noindent Esta tesis se distribuye bajo una licencia Creative Commons Atribución \oldstylenums{4.0} Internacional (\textsc{cc by} \oldstylenums{4.0}). Por lo tanto, eres libre de \emph{compartir, copiar y redistribuir} el material en cualquier medio o formato. \emph{Adaptar, remezclar, transformar y construir} a partir del material para cualquier propósito, incluso comercialmente. El licenciante no puede revocar estas libertades en tanto usted siga los terminos de la licencia, bajo los siguientes términos: \\

\noindent \ccAttribution \thinspace Atribución. Usted debe dar crédito de manera adecuada, brindar un enlace a la licencia, e indicar si se han realizado cambios. Puede hacerlo en cualquier forma razonable, pero no de forma tal que sugiera que usted o su uso tienen el apoyo del licenciante. \\

\noindent No hay restricciones adicionales. No puede aplicar términos legales ni medidas tecnológicas que restrinjan legalmente a otras a hacer cualquier uso permitido por la licencia. \\

\noindent Avisos: \\

\noindent Usted no tiene que cumplir con la licencia para elementos del material en el dominio público o cuando su uso esté permitido por una excepción o limitación aplicable. \\

\noindent No se dan garantías. La licencia podría no darle todos los permisos que necesita para el uso que tenga previsto. Por ejemplo, otros derechos como publicidad, privacidad, o derechos morales pueden limitar la forma en que utilice el material. \\

\noindent Para mayores informes sobre esta licencia, visite: \\

\href{https://creativecommons.org/licenses/by/4.0/deed.es}{https://creativecommons.org/licenses/by/4.0/deed.es} \\

Esta tesis está disponible en: \\

\href{https://github.com/tuxkernel/tesis/raw/master/tesis.pdf}{\textbf{https://github.com/tuxkernel/tesis/raw/master/tesis.pdf}}
\end{minipage}
\end{scriptsize}

\newpage
\pagestyle{empty}
\null\vfill

\newpage
\pagestyle{empty}
\null\vfill

\newpage
\pagestyle{empty}

% Dedicatoria

\begin{flushright}
\textit{A Tania, Vania y} Willy
\end{flushright}
\begin{flushright}
\textit{A mi madre}\textsuperscript\textdagger
\end{flushright}

% Agradecimientos

\newpage
\pagestyle{empty}
\null\vfill

\chapter*{\centering\mdseries\Large\textsc{Agradecimientos}}
\pagestyle{empty}
\addcontentsline{toc}{chapter}{Agradecimientos}
\pagenumbering{Roman}
\setcounter{page}{11}
\markboth{Agradecimientos}{Agradecimientos}

\noindent Aunque la redacción de este trabajo es de mi autoría, fue posible gracias al apoyo que recibí de las siguientes personas. En este sentido, agradezco al señor Isabel Lima\textsuperscript\textdagger\,su entusiasmo y colaboración a la causa. Asimismo, al señor José Corona\textsuperscript\textdagger\,por compartir sus experiencias laborales y personales; al señor Alejandro Benítez, quien compartió información de importancia y me permitió reanudar esta investigación; su devoción al Sagrado Corazón y esfuerzos para que la fiesta se siguiera celebrando contribuyeron a que el trabajo saliera adelante. También al licenciado Rogelio Meneses, quien abrió mis ojos sobre la importancia y transcendencia de Chiautempan. A mi amiga \textit{Pituca}, pues gracias a sus relaciones con la gente el trabajo salió avante. Hay muchas personas que contribuyeron para que este proyecto fuera una realidad y no figuran sus nombres; a ellos muchas gracias.

Estaré eternamente agradecida con mis maestros, especialmente con Julio Glockner y Manlio Barbosa, quienes inyectaron ánimo y confianza; con la doctora Rosalba Ramírez Rodríguez, por su cálido recibimiento en el Colegio de Antropología Social y por motivarme a terminar el trabajo; con la doctora Rosalina Estrada por sus \textit{porras} y, desde luego, gran cariño.

Agradezco profundamente a la doctora Leticia Villalobos Sampayo por asumir la dirección de este trabajo. Gracias a su profesionalismo, compromiso y dedicación este trabajo fue posible. Igualmente, aprecio el apoyo y consejos de los doctores Luis Arturo Jiménez e Isaura Cecilia García.

Agradezco el apoyo de mi familia, especialmente el de Tania y Vania, mis hijas; a\textit{Willy}, mi esposo, y José Luis, mi hermano. Gracias a su paciencia, comprensión y amor sin condiciones este trabajo fue posible. Estoy agradecida con mi abuelita \textit{Joaquinita}\textsuperscript\textdagger\,y con mis padres, especialmente con mi \textit{madre}\textsuperscript\textdagger, ejemplo de amor, trabajo y dedicación... sé que si estuvieras aquí lo habrías disfrutado tanto como yo.

Agradezco a Dios por haberme dado la vida y las fuerzas necesarias para terminar este trabajo.

% Índice

\cleardoublepage
\pagestyle{fancy}
\fancyhf{}
\fancyhead[RO,LE]{\hfill \textit{Contenido} \hfill}
\fancyfoot[RO,LE]{\hfill \thepage \hfill}
\renewcommand{\headrulewidth}{0pt}
\pagenumbering{Roman}
\setcounter{page}{13}
\renewcommand{\contentsname}{Contenido}

% Remueve los puntos en los índices

\makeatletter
\renewcommand\@dotsep{200}
\makeatother

\tableofcontents
\addcontentsline{toc}{chapter}{\mdseries Contenido}
\cleardoublepage
\renewcommand{\listfigurename}{Imágenes}
\listoffigures
\addcontentsline{toc}{chapter}{\mdseries Imágenes}

\chapter*{Siglas}
\pagestyle{empty}
\pagenumbering{Roman}
\setcounter{page}{19}
\addcontentsline{toc}{chapter}{Siglas}
\begin{tabular}{ll}
\textsc{buap} & Benemérita Universidad Autónoma de Puebla \\
\textsc{crom} & Confederación Regional Obrera Mexicana \\
\textsc{inafed} & Instituto Nacional para el Federalismo y el Desarrollo Municipal \\
\textsc{inegi} & Instituto Nacional de Estadística, Geografía e Informática \\
\end{tabular} 

\chapter*{\centering\mdseries\Large\textsc{Introducción}}
\pagestyle{fancy}
\fancyhf{}
\fancyhead[RO,LE]{\hfill \textit{Introducción} \hfill}
\fancyfoot[RO,LE]{\hfill \thepage \hfill}
\renewcommand{\headrulewidth}{0pt}
\pagenumbering{arabic}
\setcounter{page}{21}
\addcontentsline{toc}{chapter}{Introducción}
\section*{\mdseries\large\textsc{Justificación}}
\addcontentsline{toc}{section}{Justificación}

\noindent La elección del tema surgió en los años ochenta, cuando realizaba trabajo de campo organizado por la Benemérita Universidad Autónoma de Puebla (\textsc{buap}). Durante esa década y como estudiantes del Colegio de Antropología Social de dicha universidad, interactuamos con extrabajadores de la fábrica La Estrella\index[nombres]{Estrella, La (fábrica)}, en Chiautempan\index[lugares]{Chiautempan}, quienes nos brindaron información respecto de la fiesta del Sagrado Corazón de Jesús\index[nombres]{Sagrado Corazón de Jesús}. De ese primer acercamiento nació el interés por conocer y profundizar más sobre la fiesta, los exobreros y su entorno.

De tal manera que este trabajo es una relectura de ese pasado y la reconfiguración de la historia del presente. Se trata de <<la elaboración de una historia del presente para la cual ya no es el pasado lo que explica el presente, sino es el presente mismo lo que guía una o varias relecturas del pasado>> (Augé\index[nombres]{Augé, Marc} \oldstylenums{2006, 13}). Justo desde el presente debemos reconocer que Tlaxcala\index[lugares]{Tlaxcala} tiene mucha riqueza en sus costumbres que se han vuelto tradiciones, y hay que redescubrir esta historia a través de su estudio.

En México\index[lugares]{México} tenemos diferentes fiestas a lo largo del año, unas quizá de más importancia que otras, pero todas son conmemoraciones destacables. Así como se reinventa la tradición, también se reinventan las fiestas, por lo que el tema es sugerente y, al mismo tiempo, cambiante.

Por estas razones, resulta oportuno rescatar las vivencias de ese grupo de obreros que, sin saberlo, lograron una transformación social importante, porque aunque sólo buscaban mejorar sus condiciones de vida, con su ingreso a la fábrica se suscitó un cambio para ellos, para su familia y para su entorno social.

En este sentido, la transformación que provocaron también fue cultural, pues su visión espiritual fue materializada en la festividad del Sagrado Corazón de Jesús\index[nombres]{Sagrado Corazón de Jesús}, que ahora es su legado. No obstante, no se cuenta con documentos que expliquen su historia, costumbres, tradiciones y, por tanto, el sentido de la celebración; por ello esta investigación resulta un buen aporte al registro documental desde una perspectiva antropológica.

\section*{\mdseries\large\textsc{Planteamiento del problema}}
\addcontentsline{toc}{section}{Planteamiento del problema}

\noindent La fiesta del Sagrado Corazón de Jesús\index[nombres]{Sagrado Corazón de Jesús} se celebra en Chiautempan\index[lugares]{Chiautempan}, municipio perteneciente al estado de Tlaxcala\index[lugares]{Tlaxcala}. Esta población es conocida por su actividad textil, la elaboración de artesanías, el comercio y su gran número de habitantes, así como por su fe católica que ha dado origen a grandes celebraciones.

En Chiautempan\index[lugares]{Chiautempan}, después de la Semana Santa, los barrios continúan las fiestas que, al igual que las fiestas de mayordomías, concentran la atención de sus pobladores. Se sabe que las organizaciones de mayordomías son muy completas y festejan durante una parte del año; una de ellas es la del Sagrado Corazón de Jesús\index[nombres]{Sagrado Corazón de Jesús}.

Esta festividad nació en la fábrica textil La Estrella\index[nombres]{Estrella, La (fábrica)}, donde los dueños, los obreros, sus familias y la gente del pueblo celebraban misas cuando la factoría estaba en su apogeo. Se hizo así porque hicieron del Sagrado Corazón de Jesús\index[nombres]{Sagrado Corazón de Jesús} su Santo Patrón. También se invitaba a comisiones de otras fábricas para desfilar por las calles del poblado con banderas, música y cohetes.

Con el tiempo, hubo serios problemas en la industria textil y muchas fábricas se vieron obligadas a cerrar, entre ellas La Estrella\index[nombres]{Estrella, La (fábrica)}, que cerró en \oldstylenums{1970}. La celebración del Santo Patrón, no obstante, ha continuado por casi cincuenta años. Así, con el propósito de dar respuesta a los objetivos de este trabajo, nos proponemos investigar la fiesta del Sagrado Corazón\index[nombres]{Sagrado Corazón de Jesús} en diferentes épocas y circunstancias, y considerando que se ha realizado desde mucho tiempo atrás, resulta conveniente estudiar otras celebraciones similares para observar cómo se ha venido desarrollando y transformando esta fiesta en particular.

\section*{\mdseries\large\textsc{Objetivos}}
\addcontentsline{toc}{section}{Objetivos}

\noindent El trabajo tiene por \textit{objetivo general} describir la fiesta del Sagrado Corazón de Jesús\index[nombres]{Sagrado Corazón de Jesús} en tres momentos: \oldstylenums{1}) cuando la fábrica estaba activa, \oldstylenums{2}) tras el cierre de la misma acaecido en el año de \oldstylenums{1989} y, \oldstylenums{3}) en la actualidad. \textit{Específicamente}, la investigación busca: \oldstylenums{1}) observar los cambios y permanencias de la fiesta, \oldstylenums{2}) conocer la importancia y el significado de la misma para los extrabajadores de la fábrica y, \oldstylenums{3}) conocer las actividades que los organizadores han realizado para mantener a flote la fiesta en los últimos años.

\section*{\mdseries\large\textsc{Metodología}}
\addcontentsline{toc}{section}{Metodología}

\noindent En esta investigación hicimos prácticas etnográficas para profundizar en costumbres y tradiciones de los extrabajadores de la fábrica La Estrella\index[nombres]{Estrella, La (fábrica)} y las relacionamos con la mayordomía del Sagrado Corazón de Jesús\index[nombres]{Sagrado Corazón de Jesús}, Santo Patrón de la fábrica. Realizamos trabajo de campo para tener una observación directa y participativa. También entrevistamos, en diferentes momentos, a distintos actores para obtener los puntos de vista de esas historias; las entrevistas consideradas más relevantes se transcribieron fielmente para no perder la esencia de la información brindada por los entrevistados. Gracias a ello se consiguió comprender los fenómenos sociales desde la perspectiva de sus actores.

El estudio se realizó en dos partes: la primera tomó los datos empíricos recolectados en la década de los ochenta. Las entrevistas se realizaron en fechas cercanas a la festividad del Sagrado Corazón\index[nombres]{Sagrado Corazón de Jesús}. Se platicó con extrabajadores de La Estrella\index[nombres]{Estrella, La (fábrica)}, logrando obtener información referente a su vida laboral y cotidiana, aunque no se pudo concluir la investigación con el trabajo de campo organizado por el Colegio de Antropología durante ese periodo. En la segunda parte se presenciaron las fiestas de \oldstylenums{2014, 2015, 2017 y 2018}; no obstante, para la realización de esta tarea todavía encontramos a algunos extrabajadores y, de igual forma, se contactó a familiares y personas vinculadas con las fiestas de mayordomía y obtuvimos información valiosa. Actualmente la mayoría de estos extrabajadores ha fallecido, lo cual es una gran limitante para la indagación, pues no existen memorias escritas sobre la fiesta ---el único cuaderno existente lo perdió uno de los mayordomos.

Este trabajo también está fundamentado en investigación bibliográfica y en entrevistas, por ende, es descriptivo-cualitativo. Se hizo con base en la observación, sin influir; buscando en las personas solamente fuentes de conocimiento. Se investigó también en libros, hemerotecas, páginas y artículos de \textit{Internet}, de acuerdo con el objeto de estudio. Se buscó material en la biblioteca de Chiautempan\index[lugares]{Chiautempan}, desafortunadamente no se encontró bibliografía relacionada con el tema.

\section*{\mdseries\large\textsc{Descripción del contenido}}
\addcontentsline{toc}{section}{Descripción del contenido}

\noindent El trabajo está dividido en tres capítulos. En el primero (pág. \oldstylenums{\pageref{Capitulo_1}}) se incluye información referente a aspectos conceptuales sobre la temática desde el punto de vista antropológico, definiciones de antropólogos que a través de los años han explicado el tema y sobre la necesidad de recurrir a prácticas de campo. Se realizó una búsqueda cuidadosa de conceptos y definiciones y se investigó la influencia que han tenido los actos religiosos, del mismo modo que las fiestas y mayordomías en México\index[lugares]{México}. También se aborda el tema de lo que podría haber sido la primera mayordomía en el estado de Tlaxcala\index[lugares]{Tlaxcala}.

En el segundo capítulo (pág. \oldstylenums{\pageref{Capitulo_2}}) se tratan temas de interés sobre Santa Ana Chiautempan\index[lugares]{Chiautempan} y se hace una breve historia acerca de la fábrica de hilados y tejidos de algodón La Estrella\index[nombres]{Estrella, La (fábrica)}, su fundación y el traslado de la empresa de Amaxac de Guerrero\index[lugares]{Amaxac} de Guerrero\index[lugares]{Guerrero|see{Amaxac}} a Chiautempan\index[lugares]{Chiautempan}. Se muestra información de la mayordomía como organización y se describe brevemente la vida de los trabajadores en aquel tiempo. Se incluyen algunos testimonios que describen parte de su vida en el ámbito laboral, incluyendo información sobre la devoción y las fiestas anuales dedicadas al Sagrado Corazón de Jesús\index[nombres]{Sagrado Corazón de Jesús}.

El tercer capítulo (pág. \oldstylenums{\pageref{Capitulo_3}}) incluye información relacionada con el cierre de la fábrica y las ceremonias de los años \oldstylenums{1989 (\S\,3.1.1, pág. \pageref{Fiesta_1989}), 2014 (\S\,3.1.2, pág. \pageref{Fiesta_2014}) y 2015 (\S\,3.1.3, pág. \pageref{Fiesta_2015}}). Se muestran detalles de la vida de los actores. Se puede apreciar cómo la <<reinvención de la tradición>> ha permitido la permanencia de la fiesta. En ese lapso, se observa la influencia de la religión en esta sociedad y cómo se (re)produce la identidad local dentro de la familia. Durante los días en que la mayordomía se prepara para llevar a cabo la celebración, se van integrando poco a poco los miembros de la familia, pero como es mucho el trabajo, la fiesta permite que algunas amistades cercanas colaboren en la preparación de alimentos.

La fiesta es un acontecimiento social que se inicia en la familia, pero donde también participan personas de distintos lugares; juntos comparten ritos y símbolos enmarcados en la celebración.

\chapter{\mdseries Fiesta y mayordomía}\label{Capitulo_1}
\pagestyle{fancy}
\fancyhf{}
\fancyhead[RO,LE]{\hfill \textit{Capítulo 1. Fiesta y mayordomía} \hfill}
\fancyfoot[RO,LE]{\hfill \thepage \hfill}
% \addcontentsline{toc}{chapter}{Fiesta y mayordomía}
% \markboth{Capítulo 1. Fiesta y mayordomía}{Capítulo 1. Fiesta y mayordomía}

\section*{\mdseries\large\textsc{1.1. El estado de la cuestión}}
\addcontentsline{toc}{section}{1.1. El estado de la cuestión}

\noindent El presente trabajo tiene como objetivo estudiar la Fiesta del Sagrado Corazón de Jesús\index[nombres]{Sagrado Corazón de Jesús}, legado devocional de los extrabajadores de la fábrica La Estrella\index[nombres]{Estrella, La (fábrica)}, de Chiautempan\index[lugares]{Chiautempan}, Tlaxcala\index[lugares]{Tlaxcala}. Esta fiesta, con el tiempo, se convirtió en mayordomía y en la actualidad sobrevive, a pesar de los cambios históricos de la población.

Las fiestas y mayordomías en México\index[lugares]{México} han sido objeto de investigación. Diversos académicos han explorado estos temas, aunque aquí solamente se mencionarán a los estudiosos considerados más relevantes para explicar el caso planteado.

Las fiestas de mayordomía siempre han estado vinculadas con la religión y con la sociedad, e incluyen ritos, procesiones y comida en abundancia. A principios del siglo pasado, Van Gennep\index[nombres]{Van Gennep, Arnold}, antropólogo francés, analizó y describió un conjunto de ceremonias, fiestas y rituales en su libro \emph{Ritos de paso}. También dio importancia a los actos religiosos relacionados con la estructuración de la familia y de la sociedad, ritos que han acompañado a la humanidad a través del tiempo y han influido para conformar tradiciones.

\begin{quotation}
\noindent Es preciso señalar que, por lo general, solo la puerta principal, bien consagrada por un rito especial, bien en virtud de su orientación en una dirección favorable, es la sede de ritos de entrada y salida, careciendo el resto de las aberturas de ese mismo carácter de margen entre el mundo familiar y el mundo exterior (Van Gennep \oldstylenums{2008, 43}\index[nombres]{Van Gennep, Arnold}).
\end{quotation}

\noindent Van Gennep\index[nombres]{Van Gennep, Arnold} menciona varios ritos que todavía persisten en la iglesia católica y en la sociedad cristiana, entre ellos el bautismo, el matrimonio y las peregrinaciones. Es importante destacar que, según él, con la práctica de estos ritos se crean \textit{uniones}, es decir, vínculos de compromiso que difícilmente se rompen. Para confirmar lo anterior, cita a Ciszewski\index[nombres]{Ciszewski} quien afirma que <<la fraternización (social) crea un parentesco más poderoso que la consanguinidad natural>> (Van Gennep \oldstylenums{2008, 52}\index[nombres]{Van Gennep, Arnold}).

De manera similar, con el devenir del tiempo se ha venido creando una relación cercana entre los extrabajadores de la fábrica La Estrella\index[nombres]{Estrella, La (fábrica)} y sus familiares, quienes formaron una fraternidad más fuerte y continuaron con la celebración del Sagrado Corazón\index[nombres]{Sagrado Corazón de Jesús}, compartiendo la misma devoción hasta crear lo que ahora es una costumbre que ha sido asumida por las nuevas generaciones.

Por su parte, Clifford Geertz\index[nombres]{Geertz, Clifford}, en \textit{La interpretación de las culturas}, otorgando mucha importancia a lo imaginario, a lo simbólico, a la religión, habla de un sistema de símbolos duraderos que regulan e influyen en grupos de personas, en sus estados de ánimo, porque la religión armoniza las acciones humanas.

De acuerdo con Geertz\index[nombres]{Geertz, Clifford}, algunos actos rituales pueden realizarse diaria, semanal o anualmente. Dentro del ámbito católico, se realizan ceremonias en las que, desde su origen, se han utilizado símbolos que resultan necesarios. Son símbolos establecidos que fácilmente pueden entender quienes son miembros de la hermandad. Existe un sistema de símbolos que, aunque no es propiamente un lenguaje, funciona como tal, de manera que <<se usa el término para designar cualquier objeto, acto, hecho, cualidad o relación que sirva como vehículo de una concepción ---la concepción es el significado del símbolo>> (Geertz \oldstylenums{2003, 90}\index[nombres]{Geertz, Clifford}).

Considerando la importancia y la influencia que tiene la religión en los grupos sociales, Geertz\index[nombres]{Geertz, Clifford} cita a Langer\index[nombres]{Langer}, quien afirma que <<tal vez ya sea hora de que la antropología social, y especialmente la relacionada con la religión, cobre conciencia de esta circunstancia>> (Geertz
\oldstylenums{2003, 88}\index[nombres]{Geertz, Clifford}).

Desde los inicios de la conmemoración que nos incumbe, ha existido una relación muy cercana entre los obreros y el Sagrado Corazón de Jesús\index[nombres]{Sagrado Corazón de Jesús}, que se convirtió en el símbolo religioso que permitió crear una tradición de arraigo no sólo entre los ahora extrabajadores, sino también entre sus familiares y amigos. En este sentido, nos apoyamos en Geertz\index[nombres]{Geertz, Clifford}, quien afirma que es necesario reconocer que <<los símbolos sagrados tienen la función de sintetizar el \textit{ethos} de un pueblo ---el tono, el carácter y la calidad de vida, su estilo moral y estético--- y su cosmovisión>> (Geertz \oldstylenums{2003, 89}\index[nombres]{Geertz, Clifford}).

\subsection*{\mdseries\large\textsc{1.1.1. Métodos y técnicas}}
\addcontentsline{toc}{subsection}{1.1.1. Métodos y técnicas}

\noindent Para este tipo de investigación fue necesario recurrir a las prácticas etnográficas, es decir, a trabajo de campo, y así relacionarse poco a poco con el grupo de personas involucradas con la festividad del Sagrado Corazón\index[nombres]{Sagrado Corazón de Jesús} para obtener información relacionada con sus costumbres y tradiciones.

A propósito de esto, Rosana Guber\index[nombres]{Guber, Rosana} hizo una compilación publicada por el Centro de Antropología Social de Buenos Aires\index[lugares]{Buenos Aires (Argentina)}, relacionada con experiencias de trabajo de campo de diez antropólogas. Mientras que María Pozzio\index[nombres]{Pozzio, María} hizo la reseña en la que afirmó que <<\textit{estar allí} significó una experiencia transformadora para ellas ---como personas y como investigadoras--- y para sus respectivos trabajos de investigación>> (Pozzio \oldstylenums{2006, 131}\index[nombres]{Pozzio, María}). Explica, también, que el libro no es un manual, pero ayuda. Todos sabemos que, buscando información, no es fácil ser aceptada en una comunidad ajena. He aquí cómo narra su experiencia:

\begin{quotation}
\noindent Las autoras escriben sobre tramos de sus vidas que se enlazaron e hicieron posible la emergencia de <<técnicas>>, <<datos>> y <<análisis>>, y con ello muestran que la etnografía se hace como se enseña: haciendo, estando, contando, sintiendo, mirando, escuchando con los otros y con uno /a ---investigador que produce conocimiento sobre el mundo social---, conociéndose a sí mismo en esa instancia tan sacralizada pero a la vez tan personal e intransferible que es el trabajo de campo etnográfico (Pozzio \oldstylenums{2006, 133}\index[nombres]{Pozzio, María}).
\end{quotation}

\noindent Desde otra perspectiva, Marc Augé\index[nombres]{Augé, Marc}, antropólogo francés, en \textit{Hacia una antropología de los mundos contemporáneos}, explica que para realizar estudios de antropología y desarrollar un tema en esta disciplina se debe hacer buen uso de la observación. Es muy importante, considera, tener a los informantes frente a los ojos para escuchar personalmente la información de la propia naturaleza de sus testimonios, porque:

\begin{quotation}
\noindent La elaboración de una historia del presente (para la cual ya no es el pasado lo que explica el presente, sino que es el presente mismo lo que guía una o varias relecturas del pasado) es por sí misma, si no un objeto para el antropólogo, por lo menos el signo de que algo importante ha cambiado en una de las cosmologías que el antropólogo puede legítimamente estudiar si se propone tener en cuenta la observación de su propia sociedad (Augé \oldstylenums{2006, 13}\index[nombres]{Augé, Marc}).
\end{quotation}

\noindent Augé\index[nombres]{Augé, Marc} también muestra varios temas de importancia para la antropología. Por ejemplo, explica el camino a seguir cuando ya se tiene un objeto de estudio, que bien podría ser una familia, una aldea, etcétera; después de ello, el antropólogo tiene la posibilidad <<de ir al terreno de estudio para verificar la validez y el alcance de sus hipótesis, de modo que puede intentar establecer dicha validez partiendo de una serie de indicios>> (Augé \oldstylenums{2006, 23}\index[nombres]{Augé, Marc}).

Asimismo, es necesario entender que <<para el antropólogo, el sentido es siempre el sentido social, es decir, las significaciones instituidas y simbolizadas de la relación de uno con los demás>> (Augé \oldstylenums{2006, 22}\index[nombres]{Augé, Marc}). La simbolización ha estado muy vinculada a todos los grupos sociales, pero, sobre todo, cuando se vive en el mismo espacio. De manera que:

\begin{quotation}
\noindent Esta simbolización del espacio constituye, para quienes nacen en una sociedad dada, un a priori partiendo del cual se construye la experiencia de todos y se forma la personalidad de cada uno: en ese sentido, esa simbolización es a la vez una matriz intelectual, una constitución social, una herencia y la condición primera de toda historia individual o colectiva (Augé \oldstylenums{2006, 16}\index[nombres]{Augé, Marc}).
\end{quotation}

\subsection*{\mdseries\large\textsc{1.1.2. Sistema de cargos}}
\addcontentsline{toc}{subsection}{1.1.2. Sistema de cargos}

\noindent En la religión católica fácilmente se pueden encontrar simbolismos que, con el paso del tiempo, se han fusionado con las antiguas creencias prehispánicas, dando origen a nuevos ritos. Algunos de ellos se podrían observar en las celebraciones religiosas que se llevan a cabo en Jurica\index[lugares]{Jurica}, donde la fiesta patronal y el \textit{sistema de cargos} son de mucha importancia.

Jurica\index[lugares]{Jurica} está ubicado al norte del estado de Querétaro\index[lugares]{Querétaro}. Ahí Osorio Franco\index[nombres]{Osorio Franco, Lorena} realizó una investigación etnográfica en la que observó que, a pesar de los cambios por la urbanización de la ciudad y de algunas dificultades con la iglesia local, los devotos de la imagen continúan con la celebración y buscan la manera de que sea igual que en los años pasados:

\begin{quotation}
\noindent El acercamiento a la vida religiosa permite observar cómo se construye y se reconstruye la identidad local, ya que la religión, como sostiene Portal\index[nombres]{Portal} (\oldstylenums{1997, 130}), pese a que es un elemento importante de la identidad local, esta no se reduce a la homogeneidad, equilibrio o armonía absoluta. [Es] en el ciclo ceremonial, donde la identidad se construye a partir de la integración que se logra a través de la realización de la fiesta. La vida cotidiana y el ritual son ámbitos compenetrados; la devoción al santo patrono, el sistema de cargos, las fiestas, las creencias, se estructuran y reproducen desde el ámbito familiar y encuentran eco en la comunidad porque comparten códigos identitarios que les son comunes (Osorio \oldstylenums{2014, 202}\index[nombres]{Osorio Franco, Lorena}).
\end{quotation}

\noindent Los poblados construyen costumbres, cada una de ellas con su debida importancia. La mayordomía, muy ligada con la fiesta de los pueblos, es de suma importancia para los habitantes de Jurica\index[lugares]{Jurica}. Es una fiesta que, como se decía anteriormente, a pesar de los avatares, continúa. Sin embargo, es innegable que el crecimiento de la ciudad ha afectado a la población, sus costumbres y tradiciones. Al respecto, Osorio\index[nombres]{Osorio Franco, Lorena}, citando a Medina\index[nombres]{Medina, Andrés}, señala:

\begin{quotation}
\noindent En la mancha urbana no es fácil identificar la presencia de los pueblos, sobreviven uno que otro. Sin embargo, una mirada atenta a la vida que bulle en los intersticios de la gran masa de cemento permite observar los juegos pirotécnicos y las explosiones de los cohetes a lo largo de la mayor parte de los días del año, o bien embotellamientos de tráfico provocados por largas procesiones [...] La identidad de los individuos se define principalmente por el conjunto de sus pertenencias sociales. Las categorías o grupos de pertenencia más importantes ---aunque no los únicos---, en tanto alimentan la identidad personal, son la clase social, la etnicidad, las colectividades territorializadas (localidad, región, nación), los grupos de edad y género. La pertenencia social implica compartir, aunque sea parcialmente, los modelos culturales (de tipo simbólico-expresivo) de los grupos o colectivos en cuestión (Osorio \oldstylenums{2014, 204-205}\index[nombres]{Osorio Franco, Lorena}).
\end{quotation}

\noindent La mayordomía en Jurica\index[lugares]{Jurica} sobrevive por los esfuerzos de la población. La razón principal es el arraigo de sus costumbres y tradiciones, que siguen cumpliendo con el compromiso que ellos mismos se han propuesto. Actualmente, los juriquenses no sólo realizan la mayordomía, sino que han establecido que las fiestas se deben efectuar en tiempo y forma, como se ha hecho en los años anteriores:

\begin{quotation}
\noindent Para los juriquenses la fiesta es y debe de ser lo que se espera de ella, es decir, la pervivencia clara y actualizada de su tradición. El que la fiesta se haga como <<siempre se ha hecho>> es lo que da certidumbre a la gente, en el sentido de que alimenta su pertenencia y su propia reproducción cultural (Osorio \oldstylenums{2014, 214}\index[nombres]{Osorio Franco, Lorena}).
\end{quotation}

\noindent Regresando al caso de la celebración del Sagrado Corazón de Jesús\index[nombres]{Sagrado Corazón de Jesús}, patrón de la fábrica La Estrella\index[nombres]{Estrella, La (fábrica)}, no se encontró información, tampoco de las mayordomías de Santa Ana Chiautempan\index[lugares]{Chiautempan}, a pesar de que son muchas y bien organizadas, y de que en esta parte ellos sí quieren ser mayordomos y cuidan bien sus mayordomías, sobre todo las de mayor importancia: la de la Señora Santa Ana y la del Padre Jesús del Convento.

Sin embargo, sí hay información relevante sobre las mayordomías de la población de San Juan Ixtenco\index[lugares]{Ixtenco, San Juan}, Tlaxcala\index[lugares]{Tlaxcala}. Se trata de una construcción que tiene sus orígenes desde antes de la llegada de Hernán Cortés\index[nombres]{Cortés, Hernán} y de la llegada de los evangelizadores que acompañaban a los conquistadores.

A principios del siglo pasado (\oldstylenums{1906-1908}) había muchas comunidades fabriles en el vecino estado de Puebla\index[lugares]{Puebla}, tal es el caso de Atlixco\index[lugares]{Atlixco}, donde había celebraciones religiosas de importancia en su centro de trabajo. Era ahí donde se festejaba a su santo patrón, una imagen católica que ocupaba un lugar especial en el corazón de los trabajadores y dentro de las instalaciones de la fábrica:

\begin{quotation}
\noindent Sabemos que el día de San Miguel, el Santo Patrono de Atlixco\index[lugares]{Atlixco}, se festejaba en grande, bajo la organización de la Iglesia católica local. Otras fiestas importantes eran las de los muertos [...] Menos generales eran otras festividades, como las de San Agustín y de Nuestra Señora del Carmen, en cuyo honor había fiesta en las aldeas fabriles que llevaban esos nombres, mientras que en el pueblo de El León\index[lugares]{León, El (fábrica)} la fiesta se realizaba el día del Sagrado Corazón de Jesús\index[nombres]{Sagrado Corazón de Jesús} (Gamboa \oldstylenums{2001, 191}\index[nombres]{Gamboa, Leticia}).
\end{quotation}

\noindent El propósito principal de recordar las celebraciones de Atlixco\index[lugares]{Atlixco} es por la similitud que tenían con otras comunidades fabriles. La información anterior es de una entrevista realizada a un obrero, ya que hace tiempo las festividades religiosas en las fábricas textiles eran frecuentes, como era el caso de las celebraciones que se realizaban en la fábrica de hilados y tejidos de algodón, que estuvo ubicada en Santa Ana Chiautempan\index[lugares]{Chiautempan}.

Para finalizar, Madrigal\index[nombres]{Madrigal, David}, quien también estudió las fiestas y mayordomías, indica que hay elementos muy relevantes en las festividades del barrio de San Miguelito\index[lugares]{Miguelito, San (Barrio)}, que la celebración persiste por el sistema de cargos, es decir, por las mayordomías:

\begin{quotation}
\noindent La reproducción y preservación de la identidad comunitaria son aspectos en los que se observa la importancia del sistema de cargos urbano y las mayordomías como mediadores entre las fuerzas contrarias que tienden hacia la fragmentación (Madrigal \oldstylenums{2011, 132}\index[nombres]{Madrigal, David}).
\end{quotation}

\noindent San Miguelito\index[lugares]{Miguelito, San (Barrio)} es un barrio situado en el Centro Histórico de la Ciudad de San Luis Potosí\index[lugares]{Potosí, San Luis}. Es ahí donde el sistema de cargos ha evolucionado a través del tiempo. Se trata de un lugar donde existe una lucha por la sobrevivencia de las festividades religiosas, en la que los sobrevivientes de grupos de indios y campesinos persisten y se adaptan para conservar la fiesta de su santo patrón.

El sistema de cargos y la persistencia de las mayordomías están afectados por los cambios sociales y económicos, por los que se van creando situaciones a través del tiempo que no han sido fáciles de resolver:

\begin{quotation}
\noindent Dependiendo del caso, el estudio del sistema de cargos y de sus mayordomías requiere la comprensión de la dinámica de construcción de las identidades desde la misma ciudad, más propiamente, desde el espacio urbano (Madrigal \oldstylenums{2011, 139}\index[nombres]{Madrigal, David}).
\end{quotation}

\noindent A pesar del panorama anterior, Portal\index[nombres]{Portal}, citada por Madrigal\index[nombres]{Madrigal, David}, afirma de manera acertada:

\begin{quotation}
\noindent Las mayordomías y los sistemas de cargos urbanos no deben ser vistos como especies culturales en extinción, sino como formas contemporáneas de apropiación de lo moderno (Madrigal \oldstylenums{2011, 139}\index[nombres]{Madrigal, David}).
\end{quotation}

\section*{\mdseries\large\textsc{1.2. Los estudios sobre las fiestas y mayordomías en México\index[lugares]{México}}}
\addcontentsline{toc}{section}{1.2. Los estudios sobre las fiestas y mayordomías en México}

\noindent Las fiestas son una parte trascendental en la vida de los mexicanos. Antes de la Conquista ya se realizaban grandes celebraciones que se han ido modificando por diversos factores, como la imposición de la religión católica; sin embargo, se ha conservado la esencia principal, que es conmemorar algo nuclear para el grupo social.

La fiesta es necesaria por estar unida a la vida social, por todo lo que se involucra en ella: está muy ligada a los recuerdos, creencias y costumbres. Hasta se podría considerar como una tregua necesaria para poder continuar con los compromisos de la vida cotidiana, porque es parte esencial en las diferentes etapas de la vida de cada persona, así como de grupos humanos.

Existen diferentes tipos de fiestas, cada una con sus propias características, entre ellas están las populares, las religiosas y las cívicas. Según Azuela\index[nombres]{Azuela} (\oldstylenums{2005}), las fiestas cívicas están ligadas a la patria, al pueblo y al poder. Ejemplo de ello son las fiestas del primer Centenario de Independencia en \oldstylenums{1910}, donde se fusionaron esos tres elementos. Otra fiesta cívica conocida es la del \oldstylenums{20} de noviembre, que recuerda el inicio de la Revolución Mexicana, donde se involucra el gobierno, la sociedad, escolares y otros grupos (Medina \oldstylenums{2007}\index[nombres]{Medina, Andrés}).

Las fiestas cívicas son acontecimientos históricos importantes en el país, relacionadas con la patria y el pueblo:

\begin{quotation}
\noindent Las conmemoraciones civiles tienen la doble función de lealtad hacia el pasado y de compromiso con el presente (Florescano y Santana \oldstylenums{2016, 238}\index[nombres]{Florescano y Santana}).
\end{quotation}

\noindent De acuerdo con Medina\index[nombres]{Medina, Andrés} (\oldstylenums{2007}), también está la celebración del \oldstylenums{10} de mayo, en la que se festeja el Día de las Madres, adoptada en México\index[lugares]{México} en \oldstylenums{1922}. En este día se hacen diversos regalos a las mamás, generalmente son aparatos que ayudan en los quehaceres domésticos. Otra festividad es la del Día del Niño, celebrada desde el \oldstylenums{30} de abril de \oldstylenums{1924}. Dichas festividades nacieron inicialmente como fiestas escolares, ya con el tiempo se han convertido en fiestas nacionales de cierta importancia en el país.

Al investigar y tratar de encontrar definiciones acertadas acerca de la fiesta mexicana, la búsqueda podría llegar a ser un tema difícil y desgastante, sin embargo, y de acuerdo con el tema de investigación, resulta conveniente por ahora destacar las palabras de Zarauz López\index[nombres]{Zaraus López}:

\begin{quotation}
\noindent La fiesta popular puede tener distintos orígenes: cívico, familiar, religioso, o bien una mezcla de distintos factores, pero su característica general es la suma espontánea a la celebración aunque existan estímulos externos (del comercio, del sistema, de instituciones). Abarca todos los espacios, rural, urbano, las poblaciones pequeñas, provincianas y citadinas; cruza por distintos ámbitos sociales y no necesariamente está circunscrito a una clase; puede tener una connotación patriótica, religiosa o cultural y puede ser antigua o relativamente contemporánea (Florescano y Santana \oldstylenums{2016, 112}\index[nombres]{Florescano y Santana}).
\end{quotation}

\noindent Una característica de las fiestas populares es que hay participación de todos los sectores sociales. Estos festejos se convierten precisamente en fiestas populares y pertenecen al pueblo, se hacen hereditarias y, por ello, garantizan su permanencia. En sí, son las más conocidas entre todos los mexicanos. Ejemplos de ellas son: el Día de Muertos, las tan conocidas Posadas, del \oldstylenums{16 al 23} de diciembre; la Nochebuena, \oldstylenums{24} de diciembre; Navidad, \oldstylenums{25} de diciembre; el fin de año y, para finalizar, el Año Nuevo.

Las fiestas religiosas están ligadas profundamente a la vida familiar. Desde los primeros meses de vida de una persona, es normal que los padres preparen el bautizo del recién nacido, lo cual representa un rito religioso acompañado de una celebración religiosa, en la que se reúnen familiares y amigos para compartir con ellos los momentos más significativos de la vida. Y así sucesivamente, la vida de las personas continúa acompañada de fiestas religiosas. Sigue la presentación de tres años, la Primera Comunión, la Confirmación, la celebración de los Quince Años de las mujeres, las bodas y, finalmente, hasta cuando hay un fallecimiento, también hay actos religiosos.

Las fiestas religiosas en México\index[lugares]{México} son frecuentes y del dominio público, entre ellas, la del \oldstylenums{12} de diciembre, cuando se festeja a la Virgen de Guadalupe\index[nombres]{Guadalupe, Virgen de}, que según la historia, se apareció en \oldstylenums{1531}. Por su importancia, esta celebración se ha convertido en una tradición esencial para los mexicanos. Actualmente, es más que un símbolo de fe, es otro escudo nacional; tan es así, que la imagen se encuentra en diferentes templos fuera del país, inclusive en la Ciudad del Vaticano\index[lugares]{Vaticano, El}, cerca de la tumba de San Pedro\index[nombres]{Pedro, San}, el jerarca de la Iglesia Católica.

Sin duda, la celebración a la Guadalupana es la más significativa de las fiestas religiosas en México\index[lugares]{México}, donde se involucra prácticamente a todo el pueblo, efecto y resultado del pasado y del presente. Los peregrinos empiezan el camino para llegar a la Basílica de Guadalupe\index[nombres]{Guadalupe, Basílica de}, en la Ciudad de México\index[lugares]{México}. Transitan en diferentes medios de transporte: camionetas, carros, bicis y hasta caminando. Se celebran misas en todos los templos católicos del país, que en esos días permanecen rodeados de gente, de vendimia, de comida, bebida, antojitos, golosinas, y todos con mucha alegría.

Octavio Paz\index[nombres]{Paz, Octavio}, Premio Nobel de Literatura, se refirió a los días cercanos al \oldstylenums{12} de diciembre y a la actitud que se refleja en muchos de los mexicanos. Él explicaba que en días normales, durante todo el año, el mexicano es <<huraño, inseguro e introvertido>>, pero con la fortaleza y la fe en la Virgen de Guadalupe\index[nombres]{Guadalupe, Virgen de}, había cambios positivos en él, se convertía en una persona diferente:

\begin{quotation}
\noindent El tiempo suspende su carrera, hace un alto y en lugar de empujarnos a un mañana siempre inalcanzable y mentiroso, nos ofrece un presente redondo y perfecto, de danza y juerga, de comunión y comilona con lo más antiguo y secreto de México\index[lugares]{México}. El tiempo deja de ser sucesión y vuelve a ser lo que fue, y es, originariamente un presente en donde pasado y futuro al fin se reconcilian (Paz \oldstylenums{2004, 51}\index[nombres]{Paz, Octavio}).
\end{quotation}

\noindent Homobono Martínez\index[nombres]{Martínez, Homobono}, sociólogo español, inspirado en el antropólogo y sociólogo francés Marcel Mauss\index[nombres]{Mauss, Marcel}, explicaba que la fiesta es <<un hecho social total, de expresión ritual y simbólica, sagrada y profana, vinculada a las entidades colectivas, estructuradora del calendario y del espacio; objeto de estudio de las ciencias sociales y en particular de la antropología>> (Martínez \oldstylenums{2004, 33}\index[nombres]{Martínez, Homobono}).

De hecho Ruiz Rodríguez\index[nombres]{Ruiz Rodríguez} se manifiesta de acuerdo con Mauss\index[nombres]{Mauss, Marcel}, y afirma también que \textit{la fiesta es un hecho social} y la define de una manera concreta y precisa. Ambos autores comparten esta opinión y se podrían complementar para establecer una buena definición y así tener una expresión en la que no sería necesario agregar nada:

\begin{quotation}
\noindent La fiesta es un hecho social en el que se conjugan muchas dimensiones y aspectos en un evento que irrumpe en el curso de la vida cotidiana movilizando universos de sentido, valores sociales y propiciando transformaciones culturales. La fiesta está cargada de significados y es una experiencia emotiva multisensorial; imágenes, movimientos, sabores, olores, sentimientos y sonidos tejen y destejen historias e identidades de individuos y colectivos (Florescano y Santana \oldstylenums{2016, 413}\index[nombres]{Florescano y Santana}).
\end{quotation}

\noindent Renée de la Torre\index[nombres]{De la Torre, Renée} explica cómo se fortalecen las relaciones con estas fiestas, aun con personas que están ausentes, pero que añoran la tierra y, pese a la distancia, evocan los recuerdos de otros tiempos. Sin embargo, otros no se conforman con eso, por lo que un buen número de personas viajan para disfrutar de las fiestas, así como de un tiempo con la familia y amigos. Estos casos suceden muchas de las veces en las fiestas del pueblo. Es por eso que De la Torre\index[nombres]{De la Torre, Renée} manifiesta que:

\begin{quotation}
\noindent Las fiestas religiosas son contenedoras de una tupida socialidad donde se gestan intercambios y compromisos mutuos; se establecen relaciones duraderas y compadrazgos rituales, y se emparientan y mantienen relaciones entre paisanos aun a distancia (Florescano y Santana \oldstylenums{2016, 245}\index[nombres]{Florescano y Santana}).
\end{quotation}

\subsection*{\mdseries\large\textsc{1.2.1. Mayordomías en México\index[lugares]{México}}}
\addcontentsline{toc}{subsection}{1.2.1. Mayordomías en México}

\noindent Las fiestas patronales con mayordomía destinadas al santo patrón son de las principales celebraciones religiosas en México\index[lugares]{México}, diferentes en cada región aunque se encuentren en el mismo estado. Esta variación depende a veces de los organizadores y el empeño puesto en su realización ---tiempo que le dedican--- o por su situación económica ---aspecto determinante en una fiesta patronal.

En nuestro país se dan con mucha frecuencia porque muchas de ellas coinciden con la fiesta del pueblo, y en estos casos, <<México\index[lugares]{México} es el escenario de un sinfín de rituales ---fiestas en las cuales se celebra a los santos patronos y a las vírgenes protectoras>> (Florescano y Santana \oldstylenums{2016, 243}\index[nombres]{Florescano y Santana}).

Generalmente, estas fiestas son organizadas por un grupo de personas, se les llama \textit{mayordomías}, que siempre incluyen lo más importante. Obligadamente deben tener y venerar a una imagen, a la que en los días de fiesta se le acompaña con procesiones, flores, cohetes, banda de música, mariachis, comida y bebida en abundancia.

Aunque se ha investigado acerca de las fiestas patronales, hay mucho todavía por estudiar debido a las innovaciones que van surgiendo día con día. Por ejemplo, para entender los cambios y características actuales en los sistemas de cargos, se van realizando nuevas investigaciones. Medina Hernández\index[nombres]{Medina, Andrés}, un estudioso del tema, realizó investigaciones etnográficas en la Ciudad de México y asegura que:

\begin{quotation}
\noindent En Milpa Alta\index[lugares]{Milpa Alta}, uno de los pueblos originarios más activos políticamente, se comienza a realizar cambios en la organización de las mayordomías para enfrentar los elevados costos implicados en la realización de los ceremoniales comunitarios [...] Paralelamente a los cambios en la organización de las mayordomías se realizan diversas innovaciones (Medina \oldstylenums{2007, 63}\index[nombres]{Medina, Andrés}).
\end{quotation}

\section*{\mdseries\large\textsc{1.3. El enfoque antropológico de la fiesta}}
\addcontentsline{toc}{section}{1.3. El enfoque antropológico de la fiesta}

\noindent La palabra antropología es de origen griego y combina dos conceptos: \textit{anthropos}, que significa <<hombre>>, y \textit{logos}, <<relacionado con el conocimiento>>. Por esta razón, la antropología es la ciencia que estudia a los seres humanos de forma integral, sus características físicas y su cultura.

Sin ánimo de construir una definición perfecta de antropología, la idea es solamente entender el tema de la fiesta y tener una explicación de acuerdo con el objetivo en cuestión. Es real que la antropología estudia de forma integral al ser humano, las formas de relación social, las creencias, las tradiciones de los pueblos en fechas determinadas y muestra la riqueza cultural a través de sus costumbres, todo esto por medio de la observación. Por consiguiente, en la fiesta, que es un hecho social en la que se combinan diferentes aspectos de la vida de un grupo de seres humanos, se cumple con un determinado enfoque antropológico:

\begin{quotation}
\noindent La fiesta indica un rito social, compartido entre un grupo de personas, donde se marca un cierto acontecimiento a modo de celebración, y donde predomina el sentimiento positivo de la vida. El hecho de que una fiesta sea un rito implica que los participantes adopten un rol para la ocasión, por lo general ejercido con espontaneidad. Las expresiones de la fiesta suelen ser el intercambio de dones, la comida y la bebida, la música y el baile, el juego y la chanza, junto a su carácter gozoso y emocional (Borobio \oldstylenums{2011, 14}\index[nombres]{Borobio, Dionisio}).
\end{quotation}

\noindent Al continuar con el tema, se explica que la fiesta es compartir con los demás, en grupos grandes o pequeños, puesto que para una fiesta no necesariamente tiene que haber muchedumbre. Se puede investigar a un grupo reducido de personas o bien una fiesta patronal de un
pueblo, una familia, hasta una cena de navidad en familia o un cumpleaños. Borobio\index[nombres]{Borobio, Dionisio} da mucha importancia a la familia:

\begin{quotation}
\noindent No hay familia que no festeje, ni hay fiesta sin ritos. Si hay un ámbito en el que los ritos juegan una función transmisiva y festiva, es precisamente la familia, por eso titulo esta breve reflexión relacionando familia, ritos y fiesta (\oldstylenums{2011, 12}).
\end{quotation}

\noindent Algo más de sumo interés en la vida del hombre es la necesidad de sentir un rito, algo sagrado o relacionado con ello. Es algo que generalmente no se puede explicar con claridad. Al respecto, Borobio\index[nombres]{Borobio, Dionisio} acertadamente dice:

\begin{quotation}
\noindent Se trata de una experiencia que con frecuencia va unida a una determinada ritualidad, a veces relacionada con una manifestación hierofánica, por la que una realidad visible (árbol, bosque, río, montaña, fuego, viento, animales, hombres, lugares, etcétera) se relaciona con una realidad invisible, con lo Divino, lo Absoluto, lo Trascendente... Por sí misma esta experiencia de lo sagrado normalmente da lugar y reclama una expresión simbólica y una representación ritual que hace accesible y cercano a lo Invisible (\oldstylenums{2011, 13}).
\end{quotation}

\noindent Para finalizar lo relacionado con el enfoque antropológico, retomando las ideas de Guber\index[nombres]{Guber, Rosana}, debe quedar bien claro que en una fiesta lo más importante es la gente. Es decir, cómo es la fiesta para ellos, cómo piensan, cómo sienten y el modo en que se hace. Al tratar de conocer esta información, según las Ciencias Sociales y de acuerdo con Guber\index[nombres]{Guber, Rosana}, se usará el método etnográfico, por medio del trabajo de campo: 

\begin{quotation}
\noindent <<Como enfoque la etnografía es una concepción y práctica de conocimiento que busca comprender los fenómenos sociales desde la perspectiva de sus miembros>> (Guber \oldstylenums{2001, 5}\index[nombres]{Guber, Rosana}).
\end{quotation}

\noindent Como es frecuente en las Ciencias Sociales, se usará la descripción, algo muy propio de las investigaciones en antropología. Estas ciencias observan tres niveles de comprensión: el nivel primario o <<reporte>>, que es lo que se informa que ha ocurrido (el <<qué>>); la <<explicación>> o comprensión secundaria, que alude a sus causas (el <<porqué>>); y la <<descripción>> o comprensión terciaria, que se ocupa de lo que ocurrió para sus agentes (el <<cómo es>> para ellos) (Guber \oldstylenums{2001, 5}\index[nombres]{Guber, Rosana}).

\section*{\mdseries\large\textsc{1.4. La mayordomía como forma de organización}}
\addcontentsline{toc}{section}{1.4. La mayordomía como forma de organización}

\noindent Un punto importante de la investigación es el referente a la mayordomía y las tareas de los mayordomos, encargados de organizar la celebración de una figura conocida en la Iglesia Católica. La imagen podría ser un santo como San Francisco de Asís\index[nombres]{Asís, Francisco de}, patrón del pueblo de Tepeyanco\index[lugares]{Tepeyanco}, Tlaxcala\index[lugares]{Tlaxcala}, o bien una imagen como el Sagrado Corazón de Jesús\index[nombres]{Sagrado Corazón de Jesús}.

En el barrio de San Miguelito\index[lugares]{Miguelito, San (Barrio)} la fiesta del santo patrón está relacionada con el ceremonial religioso, el sistema de cargos y la vida urbana. Para ahondar un poco más en ello, se retomará la propuesta de Ana Portal\index[nombres]{Portal}, quien, por su experiencia académica y conocimientos sobre los pueblos de la Ciudad de México\index[lugares]{México}, es quien más se aproxima a la investigación de San Miguelito\index[lugares]{Miguelito, San (Barrio)}. Al respecto, la investigadora afirma:

\begin{quotation}
\noindent A partir del barrio de San Miguelito\index[lugares]{Miguelito, San (Barrio)}, le hacemos las siguientes adecuaciones: \oldstylenums{1}) existe una sola jerarquía, la de los mayordomos y sus ayudantes; \oldstylenums{2}) el número de mayordomías depende de la división territorial del barrio que hacen los propios mayordomos para abarcar a la comunidad de creyentes y simpatizantes de la fiesta patronal, incluso más allá de los límites políticos actuales del barrio; \oldstylenums{3}) los cargos están apoyados por otras formas de agrupación como las coordinaciones de sector, y el comité para la organización de las actividades incluidas en la fiesta principal está a cargo de la parroquia; \oldstylenums{4}) los requisitos y restricciones para ocupar los cargos se rigen por criterios como el arraigo en el barrio, la voluntad para generar una red de participación económica entre las familias que viven en el sector que corresponde a cada mayordomo, la disponibilidad de tiempo y la evaluación que hace todo el cuerpo de mayordomos de la labor realizada el año anterior, o de la labor que se realizará en la siguiente celebración patronal; \oldstylenums{5}) los requisitos para ocupar los cargos son definitivamente flexibles; se ajustan a cada caso [...] \oldstylenums{6}) el financiamiento de las actividades de cada mayordomía, en general, es solventada por los creyentes y simpatizantes del barrio [...] \oldstylenums{7}) el sistema de cargos urbano y las formas de agrupación que lo apoyan son independientes de la esfera política; \oldstylenums{8}) las relaciones de reciprocidad entre sectores del barrio y mayordomos se da a partir de redes familiares y de amistad entre vecinos, y \oldstylenums{9}) los mayordomos y ayudantes llevan un registro escrito de la cooperación comunitaria en el barrio, pero este no es tan detallado, y cuando los mayordomos rinden cuentas, generalmente omiten a pequeños y nuevos contribuyentes (Madrigal \oldstylenums{2011, 142-143}\index[nombres]{Madrigal, David}).
\end{quotation}

\noindent Cuando se realizó la investigación, los mayordomos comentaron que la fiesta ha decaído por los cambios sociales y económicos normales de cada lugar. Sin embargo, se han agregado nuevos eventos a la celebración.

\begin{quotation}
\noindent En esos eventos son los jóvenes quienes se han incorporado a la celebración de manera exitosa. Con ello se han introducido obras de teatro, muñecos gigantes con figuras políticas y otros atractivos que permiten hacer una celebración relajada y divertida. Así, se puede observar cómo estos cambios han ido conformando: Un proceso de transformación social y cultural local que se expresa en la emergencia de nuevos actores, nuevas identidades juveniles, otras formas de convivencia, otras tecnologías de información y comunicación utilizadas por la comunidad barrial, que también inciden en la forma de percibir la tradición de la fiesta patronal (Madrigal \oldstylenums{2011, 149}\index[nombres]{Madrigal, David}).
\end{quotation}

\noindent Es importante apuntar que generalmente se buscan mayordomos mayores a cincuenta años, preferentemente casados por la iglesia, aunque se dan varios casos en los que no es así, y cuando esto sucede, el sacerdote presiona para que se casen. El cargo dura un año, durante el cual, a partir de haber tomado el cargo, el mayordomo ahorra lo que más puede, aunque también se ayuda con las cooperaciones, porque en estos casos es permitido pedir al pueblo para la realización de la festividad. En las fiestas de mayordomía siempre hay comida en abundancia.

\section*{\mdseries\large\textsc{1.5. La etnografía de la fiesta y mayordomía en Tlaxcala\index[lugares]{Tlaxcala}}}
\addcontentsline{toc}{section}{1.5. La etnografía de la fiesta y mayordomía en Tlaxcala}

\noindent La fe católica llegó a Tlaxcala\index[lugares]{Tlaxcala} en una época más temprana que el resto del país, se podría decir que al mismo tiempo que la Conquista. De hecho, en esta ciudad se encuentra la pila en la que bautizaron a los primeros evangelizados. En aquel tiempo Tlaxcala\index[lugares]{Tlaxcala} era una república independiente, gobernada por cuatro senadores. Años más tarde fue aliada a la Corona española, por lo que los mismos conquistadores apadrinaron a estos cuatro gobernantes, en \oldstylenums{1520}. Actualmente, en la catedral y exconvento de Tlaxcala\index[lugares]{Tlaxcala} se puede apreciar frente a la pila bautismal donde aparece información de esta historia:

\begin{quotation}
\noindent \textsc{En esta fuente recibieron la fe católica los cuatro senadores de la Antigua República de Tlaxcala\index[lugares]{Tlaxcala}. El acto religioso tuvo lugar en el año de \oldstylenums{1520}, siendo ministro Dn. Juan Díaz\index[nombres]{Díaz, Juan} Capellán del Ejército Conquistador y padrinos el Capitán Dn. Hernando Cortés\index[nombres]{Cortés, Hernán} y sus distinguidos oficiales Dn. Pedro de Alvarado\index[nombres]{Alvarado, Pedro de}, Dn. Andrés de Tapia\index[nombres]{Tapia, Andrés de}, Dn. Gonzalo de Sandoval\index[nombres]{Sandoval, Gonzalo de} y Dn. Cristóbal de Olid\index[nombres]{Olid, Cristobal de}. A Maxixcatzin\index[nombres]{Maxixcatzin} se le dio el nombre de Lorenzo\index[nombres]{Lorenzo|see{Maxixcatzin}}, a Xicoténcatl\index[nombres]{Xicoténcatl} de Vicente\index[nombres]{Vicente|see{Xicoténcatl}}, a Tlahuexolotzin\index[nombres]{Tlahuexolotzin} el de Gonzalo\index[nombres]{Gonzalo|see{Tlahuexolotzin}} y a Zitlalpopocatl\index[nombres]{Zitlalpopocatl} el de Bartolomé\index[nombres]{Bartolomé|see{Zitlalpopocatl}}. Así lo refieren las historias escritas por Camargo\index[nombres]{Camargo}, Torquemada\index[nombres]{Torquemada} y Betancourt\index[nombres]{Betancourt}} (Pila bautismal del exconvento y catedral de Tlaxcala\index[lugares]{Tlaxcala}).
\end{quotation}

\noindent Además, ahí mismo se encuentra lo que es llamado el primer púlpito de Latinoamérica, con una leyenda que dice: 

\begin{quotation}
\noindent <<Aquí Tubo Principio El S. Evangelio en este Nuevo Mundo>> (Púlpito del exconvento y catedral de Tlaxcala\index[lugares]{Tlaxcala}).
\end{quotation}

\noindent Toda esta información es necesaria para entender cómo la población tlaxcalteca ha estado envuelta en la religión católica y en sus celebraciones, prácticamente desde que llegaron los primeros evangelizadores a México\index[lugares]{México}. Sin embargo, también es necesario señalar que los tlaxcaltecas ya tenían una cultura y religión propias, así como sus festejos.

Las fiestas y las mayordomías en Tlaxcala\index[lugares]{Tlaxcala} mantienen una estrecha relación con el sistema de cargos. Es un tema extenso, aunque con bibliografía insuficiente. Por ahora, se inicia esta investigación partiendo de lo que podría ser el origen de las fiestas y mayordomías en el estado de Tlaxcala\index[lugares]{Tlaxcala}. Es necesario aclarar que a pesar del gran esfuerzo, sólo quedará en una aproximación, porque en el estado se viven costumbres parecidas, pero cada mayordomía tiene algo específico que las diferencia.

Como se decía anteriormente, la fe católica llegó temprano al estado. El cronista de la ciudad de San Juan Ixtenco\index[lugares]{Ixtenco, San Juan}, Mateo Cajero\index[nombres]{Cajero, Mateo}, invirtió mucho tiempo para el rescate de documentos antiguos. En algunos casos, tuvo que hacer uso de la paleografía para comprender los textos. Ahora bien, en relación con la religión católica y las mayordomías de aquel tiempo, Cajero\index[nombres]{Cajero, Mateo} afirma que:

\begin{quotation}
\noindent A Ixtenco\index[lugares]{Ixtenco, San Juan} se le puede llamar cuna de la evangelización por ser el primer pueblo indígena que hizo los cimientos de su iglesia antes de la Conquista, fueron los primeros que escucharon la palabra de Dios, por conducto del hombre que adelantaron para inspeccionar el nuevo continente. La doctrina cristiana, la tenemos muy arraigada, se ha demostrado hasta nuestros días como muy pocos pueblos o quizá ninguno; en el siglo \textsc{xix} todavía con un buen número de celebraciones, que poco a poco la civilización moderna la está consumiendo, pero la fe y la creencia en Dios nos ha dado fuerza para continuar y mantener las tradiciones del pueblo (Cajero \oldstylenums{2009, 56}\index[nombres]{Cajero, Mateo}).
\end{quotation}

\noindent En su obra acerca del enviado español, Cajero\index[nombres]{Cajero, Mateo} afirma que fue alrededor de \oldstylenums{1510} cuando trataron de construir un templo, del que sólo se puede apreciar una parte de los cimientos. Sin embargo, años más tarde consiguieron la edificación de otro. Los habitantes de Ixtenco\index[lugares]{Ixtenco, San Juan}, guiados por el sacerdote del lugar, instauraron la primera cofradía que después dio lugar a la mayordomía, que años después se ofreció al santo patrón del pueblo: San Juan\index[nombres]{Juan, San (santo patrón)}.

De la fuente del Archivo Parroquial, Libro de Cofradías \oldstylenums{1731 caja 32}, el autor afirma:

\begin{quotation}
\noindent Los naturales buenamente fundaron su cofradía, los padres ordenan las siguientes constituciones: que cada año el día de Corpus Christi\index[nombres]{Corpus Christi}, en la octava se ha de celebrar su festividad con misa cantada, sermón y procesión, por los hermanos vivos y difuntos, por lo cual, de la limosna se han de dar \$\oldstylenums{12.00} pesos al cura, en cada mes una misa cantada con diácono, la limosna ha de ser de \$\oldstylenums{3.00} pesos, cada hermano por su inscripción pagará 4 reales más medio real para gastos de aceite de la lámpara (Cajero \oldstylenums{2009, 57}\index[nombres]{Cajero, Mateo}).
\end{quotation}

\noindent Desde ese momento se establecieron los lineamientos para la primera cofradía en Ixtenco\index[lugares]{Ixtenco, San Juan}. Se organizó el cuidado del dinero de las cooperaciones para invertir, sólo en la celebración, adornos para el culto divino, entre otros. Además, tenían otras obligaciones, pues el pueblo debía reunirse en la iglesia para la festividad de Nuestra Señora la Santísima Virgen y rezar el rosario en voz alta, con devoción; al mismo tiempo, pedir por las almas de los fieles difuntos de la cofradía. Desde entonces se hablaba de cofradías y de mayordomos. Para elegir mayordomos de la corporación se buscaban a personas honradas, ejemplares, justas, honestas y de buena moral.

Para entonces ya se tenía la cofradía, mayordomos, diputados y los debidos lineamientos para la organización y su buen funcionamiento en la población de Ixtenco\index[lugares]{Ixtenco, San Juan}. Todavía no había en sí una mayordomía como la que actualmente se conoce ---según Cajero\index[nombres]{Cajero, Mateo}---, y fue hasta \oldstylenums{1731} cuando ya se estableció algo similar: la mayordomía consagrada al Santo Patrón de Ixtenco\index[lugares]{Ixtenco, San Juan}, San Juan Bautista\index[nombres]{Bautista, San Juan (santo patrón)}.

Estaba la misa mensual cantada, con responso para los fieles difuntos, como ya había sido costumbre y con el debido respeto al patrón del pueblo. Año tras año se celebraba la misa a San Juan\index[nombres]{Bautista, San Juan (santo patrón)} con tres sacerdotes en el altar principal, acompañada de su procesión.

Al día siguiente se celebraba otra misa, se repicaba doble con las campanas y nuevamente responso para los difuntos. Se acordó también el pago del sacerdote: <<por sus servicios \$\oldstylenums{4.00} por cada sacerdote>> (Cajero \oldstylenums{2009}\index[nombres]{Cajero, Mateo}). Todo eso se debía mantener con los pagos puntuales de la cofradía. Esperaban contar con el acuerdo de todos los mayordomos y darle mucha atención a las cuentas de dinero año con año, que quedaban plasmadas en el libro.

Para los cambios de cargos, se buscaba también que la designación de los mayordomos y diputados fuera por votación. Antes de retirarse, debían entregar un informe del dinero que se había recibido y gastado: <<Y así el \oldstylenums{15 de junio de 1731}, se cantó la primera misa al Señor San Juan Bautista, por los naturales vivos y difuntos, \oldstylenums{9} días después, el \oldstylenums{24} de junio se cantó la
misa de la natividad de San Juan>> (Cajero \oldstylenums{2009, 62}\index[nombres]{Cajero, Mateo}).

Cajero\index[nombres]{Cajero, Mateo} señala que, entre sus más lejanas memorias, recuerda que pasaba un hombre a su casa, enviado por el mayordomo, el día anterior a la misa mensual, y hacía la invitación a las personas en su domicilio, y en la madrugada, cuando aún dormían. Entonces no había puertas en las casas, por lo que el enviado podía entrar hasta cerca de la cama, y decía:

\begin{quotation}
\noindent Queridos hermanos en nuestro Señor Jesucristo deseamos que el padre eterno te haya permitido ver otro nuevo día, venimos a su sagrada morada a decirle a usted que se ha llegado el día, el momento, la fecha, que con todo nuestro amor al Santísimo Patrón el Señor San Juan Bautista\index[nombres]{Bautista, San Juan (santo patrón)} juntos los creyentes iremos con fe a celebrar su santísima misa como es la costumbre de cada mes (Cajero \oldstylenums{2009, 140}\index[nombres]{Cajero, Mateo}).
\end{quotation}

\noindent Actualmente, en San Juan Ixtenco\index[lugares]{Ixtenco, San Juan} hay dos mayordomías principales que se celebran mensualmente: la de San Juan Bautista\index[nombres]{Bautista, San Juan (santo patrón)}, el santo patrón del pueblo, y la de Corpus Christi\index[nombres]{Corpus Christi}. Hay también otras celebraciones durante todo el año en las que se festeja al santo de su devoción, con la responsabilidad de cumplir con los deberes y las obligaciones que implica. Aun en los casos de personas que por razones de trabajo viven fuera del pueblo, a ellas también se les asigna su comisión y asisten a cumplir su encargo.

Entre los compromisos más costosos de la mayordomía están los de la banda de música, los juegos pirotécnicos y los gastos de la alfombra. Para estos casos piden cooperación de los paisanos que viven fuera del país. Están también los gastos menores, como el adorno floral de la iglesia, cohetes, música para la procesión, por mencionar algunos.

Hay otras mayordomías en Ixtenco\index[lugares]{Ixtenco, San Juan} pero que se festejan anualmente como la Virgen de Guadalupe\index[nombres]{Guadalupe, Virgen de}, la Virgen del Carmen\index[nombres]{Carmen, Virgen del}, el Señor de Chalma\index[lugares]{Chalma}, la Buena Muerte, Santiago Caballero, la Virgen de la Natividad, San Isidro Labrador (Cajero \oldstylenums{2009, 143}\index[nombres]{Cajero, Mateo}).

Cabe mencionar que, además de lo que es conocido sobre el tema, ellos tienen otra modalidad de mayordomía. Se trata de algunas personas que ya han tomado el cargo durante varios años. Esto sucede con imágenes muy queridas en el pueblo, por ejemplo: La Preciosa Sangre, coordinada por Lucía Díaz. Son numerosas las mayordomías en San Juan Ixtenco\index[lugares]{Ixtenco, San Juan}, por lo que solamente se mencionarán algunas: Los Fieles Difuntos, la Asunción de la Virgen, Asociación de la Adoración Nocturna, San Judas Tadeo, San Miguel Arcángel. De manera formal, se declaró en San Juan Ixtenco\index[lugares]{Ixtenco, San Juan} la primera cofradía:

\begin{quotation}
\noindent En fecha de \oldstylenums{26 de junio de 1665}, estando presentes los párrocos Nicolás Hernández Clavero, Antonio Jurado, nombraron los mayordomos, Francisco Bans de Reyes y Juan Antonio, escribano, los diputados Juan Andrés, Francisco Felipe, Diego Mateo, Juan Miguel, Juan Melchor, Diego López, testigos, Bartolomé Hernández de Salazar y Pedro Bermúdez, entregando sus originales al Excelentísimo Señor Don Diego de Osorio Escobar, siendo secretario Simón Báez en la Ciudad de Puebla\index[lugares]{Puebla} (Cajero \oldstylenums{2009, 58}\index[nombres]{Cajero, Mateo}).
\end{quotation}

\noindent Las mayordomías en Ixtenco\index[lugares]{Ixtenco, San Juan} son muy conocidas en Tlaxcala\index[lugares]{Tlaxcala}, sobre todo por el entusiasmo de sus pobladores. Por alguna razón les llaman también <<matumas>>. Ellos tienen muchas fiestas de mayordomías durante el año. Hasta los nuevos santos de Tlaxcala\index[lugares]{Tlaxcala} que canonizaron recientemente en Roma ya tienen su fiesta de mayordomía.

\subsection*{\mdseries\large\textsc{1.5.1. San Felipe Cuauhtenco\index[lugares]{Cuauhtenco, San Felipe}}}
\addcontentsline{toc}{subsection}{1.5.1. San Felipe Cuauhtenco}

\noindent Ibarra García\index[nombres]{Ibarra García, Miguel} investigó sobre las fiestas y mayordomías en San Felipe Cuauhtenco\index[lugares]{Cuauhtenco, San Felipe} y encontró que a lo largo del año se realizan también varias celebraciones religiosas, aunque las principales son las que tienen a cargo las mayordomías. En Cuauhtenco\index[lugares]{Cuauhtenco, San Felipe} la fiesta del pueblo es la más importante, donde se festeja, cada \oldstylenums{5} de febrero, a San Felipe de Jesús\index[nombres]{Jesús, San Felipe de}.

De acuerdo con el calendario, continúan las festividades con la Santa Cruz\index[nombres]{Santa Cruz}, el \oldstylenums{3} de mayo; San Isidro Labrador\index[nombres]{Labrador, San Isidro}, el \oldstylenums{15} de mayo; San Bernardino de Siena\index[nombres]{Siena, San Bernardino de}, el \oldstylenums{20} de mayo; Corpus Christi\index[nombres]{Corpus Christi}, también en mayo; Nuestra Señora de la Luz\index[nombres]{Luz, Nuestra Señora de la}, entre mayo y junio; el Sagrado Corazón de Jesús\index[nombres]{Sagrado Corazón de Jesús}, siempre en junio, y continúan con otras cinco celebraciones que tienen mayordomía.

Hay grupos que no pertenecen a mayordomías pero organizan algunos eventos importantes, como la peregrinación al Señor de Tepalcingo\index[lugares]{Tepalcingo}, en Morelos\index[lugares]{Morelos}, o la Virgen de Guadalupe\index[nombres]{Guadalupe, Virgen de} en la Ciudad de México\index[lugares]{México}; durante la cuaresma, la peregrinación a pie a Huamantla\index[lugares]{Huamantla}; la peregrinación ciclista a San Juan de los Lagos\index[lugares]{Lagos, San Juan de los}, y otras más.

La comunidad de San Felipe Cuauhtenco\index[lugares]{Cuauhtenco, San Felipe} es muy activa en las celebraciones católicas y tiene una estrecha relación con los pueblos circunvecinos. En los días de fiesta se visitan mutuamente. Sus celebraciones tienen mucha formalidad:

\begin{quotation}
\noindent A las que se acude sólo por rigurosa invitación por escrito obsequiada por las fiscalías respectivas. Se visitan las siguientes comunidades: Barrio de la Luz, San Pedro Xochiteotla\index[lugares]{Xochiteotla, San Pedro}, San Pedro Tlalcuapan\index[lugares]{Tlalcuapan, San Pedro}, San Pedro Munoztla\index[lugares]{Munoztla, San Pedro}, San Bartolomé Cuahuixmatlac\index[lugares]{Cuahuixmatlac, San Bartolomé}, San Damián Tlacocalpan\index[lugares]{Tlacocalpan, San Damián}, San Francisco Tetlanohcan\index[lugares]{Tetlanohcan, San Francisco}, San Rafael Tepatlaxco\index[lugares]{Tepatlaxco, San Rafael}, Guadalupe Tlachco\index[lugares]{Tlachco, Guadalupe}, Santa Cruz\index[lugares]{Santa Cruz} Tlaxcala\index[lugares]{Tlaxcala}, y Santa Cruz Guadalupe\index[lugares]{Guadalupe, Santa Cruz} (Ibarra \oldstylenums{2010, 182}\index[nombres]{Ibarra García, Miguel}).
\end{quotation}

\noindent Ibarra García\index[nombres]{Ibarra García, Miguel} retoma las palabras de Miranda Pelayo\index[nombres]{Pelayo, Miranda} (\oldstylenums{1968}) para explicar qué es una cofradía:

\begin{quotation}
\noindent En sus inicios [...] fue una hermandad formada por individuos movidos por el deseo de adorar o rendir homenaje a un santo en particular y fue este mismo su fin principal durante la Colonia. Posteriormente, la denominación fue tomada por grupos corporados cercanos a la Iglesia Católica que se encargaron de llevar a cabo las festividades religiosas a lo largo del año.
\end{quotation}

\noindent Más adelante, Ibarra\index[nombres]{Ibarra García, Miguel} explica que en Cuauhtenco\index[lugares]{Cuauhtenco, San Felipe} cofradía es todo el grupo de personas que pertenecen a alguna organización, empezando por el grupo de la Fiscalía, que se compone de cinco fiscales y los mayordomos de las once imágenes:

\begin{quotation}
\noindent Las mayordomías, en general, se componen de tres individuos el Mayordomo, propiamente dicho, un Devotado y un Topile. Al igual que en la fiscalía, cada uno de ellos con funciones específicas señaladas. Existe una excepción en la conformación de las mayordomías, es la del Santo Patrón de la comunidad, San Felipe de Jesús\index[lugares]{Jesús, San Felipe de}. Para esta se agregan, además de los miembros mencionados, cuatro Tequihuas (Ibarra \oldstylenums{2010, 185}\index[nombres]{Ibarra García, Miguel}).
\end{quotation}

\noindent Cuauhtenco\index[lugares]{Cuauhtenco, San Felipe} es una comunidad perteneciente al municipio de San Bernardino Contla\index[lugares]{Contla, San Bernardino} y está adscrito a la parroquia de este; se encuentra próximo a Chiautempan\index[lugares]{Chiautempan}, por lo que servirá de referencia para este estudio, a falta de información abundante de Chiautempan\index[lugares]{Chiautempan}.

Cabe mencionar que Chiautempan\index[lugares]{Chiautempan} es uno de los municipios más grandes del estado de Tlaxcala\index[lugares]{Tlaxcala}; es una verdadera lástima que no se haya encontrado bibliografía relacionada con sus fiestas y mayordomías.

\subsection*{\mdseries\large\textsc{1.5.2. Tepeyanco\index[lugares]{Tepeyanco}, Tlaxcala\index[lugares]{Tlaxcala}}}
\addcontentsline{toc}{subsection}{1.5.2. Tepeyanco, Tlaxcala}

\noindent Otras investigaciones sobre la fiesta de Tepeyanco\index[lugares]{Tepeyanco} y el sistema de cargos las realizó Moctezuma Pérez\index[nombres]{Moctezuma Pérez, Sergio}, quien afirma que se desconoce exactamente en qué año se fundó, pero es probable que <<Sí estamos seguros que es característico de las sociedades pertenecientes al Altiplano Central. La religiosidad es un elemento que ha estado presente en San Francisco Tepeyanco\index[lugares]{Tepeyanco} por lo menos desde el siglo \textsc{xvi}, cuando se inició la construcción de una iglesia y un monasterio por parte de los franciscanos españoles llegados al continente americano>> (Moctezuma \oldstylenums{2013, 67}\index[nombres]{Moctezuma Pérez, Sergio}).

Moctezuma\index[nombres]{Moctezuma Pérez, Sergio} comenta la definición y tareas de los mayordomos:

\begin{quotation}
\noindent El mayordomo es una persona preferentemente mayor de \oldstylenums{50} años, casado y considerado responsable, quien se encarga de preparar la celebración en honor a un santo ---por ejemplo, San Francisco de Asís\index[nombres]{Asís, Francisco de}, San Sebastián\index[nombres]{San Sebastián} o San José\index[nombres]{San José}---, o una imagen ---como puede ser la Trinidad, el Sagrado Corazón de Jesús\index[nombres]{Sagrado Corazón de Jesús} o la Santa Cruz\index[nombres]{Santa Cruz}---, o bien una festividad como en el caso de la Semana Santa [...] Esta se realiza mediante una misa, en la que participan los feligreses y a quienes se les convida al final una comida ---que puede consistir en arroz, guisado, pan y tortillas, refrescos, cervezas y bebidas alcohólicas, como tequila o ron. Además de lo anterior, se contrata a un grupo de músicos para que amenicen la procesión, la misa y la fiesta en la casa del mayordomo. También se deben comprar cohetes y arreglos florales para el interior de la iglesia. A las personas que ayudan al mayordomo se les da de comer durante los días que sean necesarios. El cargo de mayordomo dura un año. Por ejemplo, el santo de Tepeyanco\index[lugares]{Tepeyanco} es San Francisco de Asís\index[nombres]{Asís, Francisco de} y se celebra el día \oldstylenums{4} de octubre. Ese día, al terminar la misa, el mayordomo saliente propone a tres candidatos a cubrir la festividad del año siguiente. Las personas votan a cada candidato y quien obtenga más votos será el nuevo mayordomo (Moctezuma \oldstylenums{2013, 75}\index[nombres]{Moctezuma Pérez, Sergio}).
\end{quotation}

\noindent Para ser elegido mayordomo es importante tener la simpatía del pueblo. Después de la votación, el elegido ofrece ese mismo día galletas y bebidas en su casa; además, se le entrega una alcancía para que pueda trabajar con todos los preparativos del año siguiente. También hay celebraciones de menor responsabilidad que la fiesta patronal, pero importantes para las familias de Tepeyanco\index[lugares]{Tepeyanco} como las posadas, la peregrinación a Ocotlán\index[lugares]{Ocotlán} y Corpus Christi\index[nombres]{Corpus Christi}.

\section*{\mdseries\large\textsc{1.6. Los estudios sobre la vinculación fábrica-fiesta en Tlaxcala\index[lugares]{Tlaxcala}}}
\addcontentsline{toc}{section}{1.6. Los estudios sobre la vinculación fábrica-fiesta en Tlaxcala}

\noindent Algunas autoras han investigado cómo se originó la industria textil en las zonas de Puebla\index[lugares]{Puebla} y Tlaxcala\index[lugares]{Tlaxcala}. Existe información muy completa e interesante referente a las fábricas, los dueños, los empleados, sus trabajadores, migración de los obreros, vivienda, sindicatos, etcétera. Pero es muy escasa sobre los festejos religiosos que involucraban a la comunidad obrera, probablemente porque el tema de la clase obrera simplemente dejó de ser atractivo. Al respecto, Gamboa\index[nombres]{Gamboa, Leticia} afirma:

\begin{quotation}
\noindent Las condiciones de trabajo y de vida, la cultura, las mentalidades, la estructura y otros aspectos que explican el proceso de formación de toda clase social, no habían merecido la atención sino de unos cuantos investigadores. Así el resultado común fue una historia política, ocupada de los hechos espectaculares y en consecuencia atrapada en la coyuntura (Gamboa \oldstylenums{2001, 15-16}\index[nombres]{Gamboa, Leticia}).
\end{quotation}

\noindent La metodología empleada por Gamboa\index[nombres]{Gamboa, Leticia} fue a través de entrevistas, investigación en archivos y prensa, con lo que se puede conocer el panorama general de las fábricas de ese tiempo.

Cabe mencionar que a principios del siglo pasado había muchas comunidades de obreros en Puebla\index[lugares]{Puebla} como Atlixco\index[lugares]{Atlixco}, donde había celebraciones religiosas de importancia en sus centros de trabajo. Era ahí donde se festejaba a su santo patrón, es decir, a la imagen católica que ocupaba un lugar especial en el corazón de los trabajadores:

\begin{quotation}
\noindent Sabemos que el día de San Miguel\index[nombres]{Miguel, San}, el Santo Patrono de Atlixco\index[lugares]{Atlixco}, se festejaba en grande, bajo la organización de la Iglesia católica local. Otras fiestas importantes eran las de los muertos [...] Menos generales eran otras festividades, como las de San Agustín\index[nombres]{Agustín, San} y de Nuestra Señora del Carmen\index[nombres]{Carmen, Nuestra Señora del}, en cuyo honor había fiesta en las aldeas fabriles que llevaban esos nombres, mientras que en el pueblo de El León\index[lugares]{León, El (fábrica)} la fiesta se realizaba el día del Sagrado Corazón de Jesús\index[nombres]{Sagrado Corazón de Jesús} (Gamboa \oldstylenums{2001, 191}\index[nombres]{Gamboa, Leticia}).
\end{quotation}

\noindent Por lo tanto, el propósito principal de recordar las celebraciones de Atlixco\index[lugares]{Atlixco} es por la similitud que tenían con las otras comunidades fabriles, en este caso, con las festividades religiosas de la fábrica de hilados y tejidos de algodón que estuvo en Chiautempan\index[lugares]{Chiautempan}, donde se festejaba también al santo patrón de la fábrica: el Sagrado Corazón de Jesús\index[nombres]{Sagrado Corazón de Jesús}.

\section*{\mdseries\large\textsc{1.7. Marco teórico}}
\addcontentsline{toc}{section}{1.7. Marco teórico}
\subsection*{\mdseries\large\textsc{1.7.1. La fiesta}}
\addcontentsline{toc}{subsection}{1.7.1. La fiesta}

\noindent La fiesta es un tema muy extenso e interesante. En nuestro país tenemos una larga trayectoria de festividades. Antes de la Conquista ya se celebraban suntuosas fiestas como los ceremoniales a los dioses prehispánicos. Llevamos en las venas memorias de lo que ha sido nuestro pasado. Probablemente por esta razón, en México\index[lugares]{México} aparecen muchas fiestas durante todo el año.

Ciertamente, eso nos distingue del resto del mundo. Al respecto, Octavio Paz\index[nombres]{Paz, Octavio} consideraba:

\begin{quotation}
\noindent Somos un pueblo ritual. Y esta tendencia beneficia a nuestra imaginación tanto como a nuestra sensibilidad, siempre afinadas y despiertas. El arte de la Fiesta envilecido en casi todas partes se conserva intacto entre nosotros. En pocos lugares del mundo se puede vivir un espectáculo parecido al de las grandes fiestas religiosas de México\index[lugares]{México}, con sus colores violentos, agrios y puros, sus danzas, ceremonias, fuegos de artificio, trajes insólitos, y la inagotable cascada de sorpresas de los frutos, dulces y objetos que se venden en esos días en plazas y mercados (Paz \oldstylenums{2004, 51}\index[nombres]{Paz, Octavio}).
\end{quotation}

\noindent Para las fiestas, la parte social es lo principal, porque no es posible hacer una fiesta sin personas, por lo que la fiesta es considerada como <<un hecho social total>>. Según Mauss\index[nombres]{Mauss, Marcel}, la fiesta es <<una celebración cíclica y repetitiva, de expresión ritual y vehículo simbólico, que contribuye a significar el tiempo (calendario) y a demarcar el espacio>> (Martínez \oldstylenums{2004, 34}\index[nombres]{Martínez, Homobono}).

Homobono Martínez\index[nombres]{Martínez, Homobono} describe así a la fiesta popular: <<La fiesta es un hecho social total, de expresión ritual y simbólica, sagrada y profana, vinculada a las identidades colectivas, estructuradora del calendario y del espacio; objeto de estudio de las ciencias sociales y en particular de la antropología>> (\oldstylenums{2004, 33}).

Mijaíl Bajtín\index[nombres]{Bajtín, Mijaíl} (\oldstylenums{1974}) está de acuerdo con lo anterior. Las fiestas se vinculan con las celebraciones religiosas, porque siempre han estado ligadas al descanso y a la relajación. Por lo tanto, es necesario dar un sentido más positivo a la fiesta. La idea debe provenir <<del mundo de los objetivos superiores de la existencia humana, es decir, del mundo de los ideales>> (Florescano y Santana \oldstylenums{2016, 11-12}\index[nombres]{Florescano y Santana}).

Podría decirse que la \textit{fiesta} tiene una evocación profunda, como todo lo que se encierra en ella. Necesariamente hay una serie de vínculos entre lo sacro y lo profano, las misas y las procesiones, así como bailes, comidas y bebidas, ferias, juegos y palenques. En la fiesta es muy importante la organización, pero también la participación:

\begin{quotation}
\noindent La fiesta propicia la transmisión de tradiciones a través de las cuales se heredan conocimientos y memorias compartidas. Llevar a cabo una fiesta, sea comunitaria, familiar o dentro de un espacio determinado, conlleva la participación de costumbres, mitos, ritos y deberes que vienen de tradiciones anteriores, pasan de una generación a la siguiente, y de ese modo contribuyen a preservar conocimientos que no solo constituyen elementos importantes de la identidad común, sino que involucran el respeto al otro, la responsabilidad con la sociedad y el desarrollo de la vida personal dentro de una colectividad (Florescano y Santana \oldstylenums{2016, 13}\index[nombres]{Florescano y Santana}).
\end{quotation}

\noindent Es necesario mencionar que hay diferentes tipos de fiesta, y las de mayor transcendencia son las patronales. Es frecuente que por estas celebraciones algunas personas que viven en otros lugares, incluso en el extranjero, regresen, lo que permite que se encuentren parientes y amigos, sobre todo en comunidades rurales. Por ello se pueden entender fácilmente las palabras de Soledad González\index[nombres]{González, Soledad}:

\begin{quotation}
\noindent La fiesta ha sido un espacio privilegiado donde se interceptan lo público y lo privado, donde se expresan las relaciones y compromisos de los grupos sociales que participan en ella, donde se afirman, pero también se recrean, las pertenencias comunitarias y la producción de orden y sentido para las sucesivas generaciones (Florescano y Santana \oldstylenums{2016, 277}\index[nombres]{Florescano y Santana}).
\end{quotation}

\subsection*{\mdseries\large\textsc{1.7.2. Los pueblos nahuas}}
\addcontentsline{toc}{subsection}{1.7.2. Los pueblos nahuas}

\noindent Ortega Olivares\index[nombres]{Ortega Olivares, Mario} y Fabiola Mora\index[nombres]{Mora, Fabiola}, en sus investigaciones sobre las fiestas patronales de dos pueblos nahuas en el Distrito Federal\index[lugares]{Distrito Federal}, citan a Andrés Medina\index[nombres]{Medina, Andrés}, quien realizó investigaciones para la antropología mexicana, en su arduo trabajo de campo en las zonas de Chiapas\index[lugares]{Chiapas} y los pueblos mayas. Medina\index[nombres]{Medina, Andrés} (\oldstylenums{1996}) explica claramente que las fiestas y los sistemas de cargos mantienen una relación muy estrecha:

\begin{quotation}
\noindent La fiesta mexicana sólo es posible en la tradición rural por la existencia de una estructura organizativa que está en el meollo de la comunidad, el sistema de cargos, y en la que se articulan de manera compleja y original los procesos socioeconómicos, religiosos y étnicos que constituyen a la comunidad nacional, pero principalmente la india, de raíz mesoamericana (Ortega y Rosales \oldstylenums{2014, 52}\index[nombres]{Ortega y Rosales}).
\end{quotation}

\noindent Ya se llevaban a cabo ciertas fiestas entre estos grupos antes de la llegada de los españoles, justo porque las fiestas aparecen en las comunidades humanas como una necesidad social. Así lo sostiene Roger Caillois\index[nombres]{Caillois, Roger} en \textit{El hombre y lo sagrado}:

\begin{quotation}
\noindent En las civilizaciones llamadas primitivas el contraste ofrece mucho más relieve. La fiesta dura varias semanas, varios meses, interrumpidos por periodos de reposo, de unos cuatro o cinco días. A veces eran precisos varios años para reunir la cantidad de víveres y de riquezas que se verían no solo consumidos o gastados con ostentación, sino también destruidos y derrochados pura y simplemente, porque el derroche y la destrucción, formas del exceso, entran por derecho propio en la esencia de la fiesta (Caillois \oldstylenums{2006, 102}).
\end{quotation}

\noindent Las fiestas son necesarias porque en ellas se vive un mayor grado de ánimo y renovación, se olvidan las preocupaciones de la vida diaria y las actividades cotidianas. En los seres humanos siempre está presente la necesidad de un descanso, de renovación y socialidad que se
cumple en la fiesta:

\begin{quotation}
\noindent Lo festivo, al romper con lo cotidiano, da paso a un tiempo especial, vinculado con el principio original y lo sagrado. La fiesta comunitaria propicia el intercambio entre distintos pueblos, reafirma los valores colectivos y fortalece los pactos entre los individuos, las comunidades y sus santos patronos y deidades protectoras (Florescano y Santana \oldstylenums{2016, 12}\index[nombres]{Florescano y Santana}).
\end{quotation}

\noindent En conclusión, la fiesta es necesaria entre los pueblos, porque es una celebración social relacionada con el calendario y los actos rituales. En ella se vincula la identidad de un grupo con lo sagrado y lo profano, aunque también se hace presente el descanso y el compromiso, y se rompe con lo cotidiano.

\subsection*{\mdseries\large\textsc{1.7.3. La mayordomía}}
\addcontentsline{toc}{subsection}{1.7.3. La mayordomía}

\noindent Hacer una definición de algo tan cambiante como las mayordomías, que varían en cada región, resulta una tarea difícil. Pero aquí planteamos algunas posturas que pueden dar unidad al sentido de la mayordomía:

\begin{quotation}
\noindent Las mayordomías se pueden definir como las instituciones comunitarias administradas por los grupos encargados de las celebraciones rituales, ceremoniales y de las festividades religiosas y que, de alguna manera, se articulan con el control social de la comunidad. De hecho, las mayordomías se conforman y eligen, unánimemente, por consenso interno: siempre apegados a las tradiciones comunitarias (Madrigal \oldstylenums{2011, 138}\index[nombres]{Madrigal, David}).
\end{quotation}

\noindent De igual forma, la antropóloga Paz Moreno\index[nombres]{Moreno, Paz} describe de manera precisa los detalles y las formas de organización para poder llevar a cabo una celebración mediante la mayordomía:

\begin{quotation}
\noindent En la mayoría de los pueblos campesinos mexicanos existe un sistema de fiesta anual, a veces llamado de cofradías, otras de cargos, cuya base organizativa es la elección de un encargado ---mayordomo--- que asume varias obligaciones durante todo el año que ocupa el cargo, como cuidar la capilla, colocar las flores, y hacer que se celebren los rezos, misas, y novenas para honrar al santo o a la virgen en cuestión. Sin embargo, el papel fundamental del mayordomo es organizar y pagar todos los gastos de la fiesta anual en honor del santo patrón o de la virgen patrona: los músicos, los danzantes, invitar y corresponder a los mayordomos de otras fiestas, a las autoridades, y a todos los participantes del barrio o del pueblo. Además, el mayordomo deberá costear los adornos de la iglesia ---incluidas las velas, las flores y el incienso---, así como decorar para la fiesta el atrio y las calles. También costeará los fuegos pirotécnicos, los cohetes de la fiesta y todos los gastos e invitaciones en que hayan incurrido sus ayudantes. El mayordomo, que compite en derroche y esplendor con los mayordomos de otros barrios y con los anteriores de su propio barrio, tiene que sufragar unos gastos enormes para los que, en muchos casos, ha ahorrado él y su familia ---a veces emigrando temporalmente a Estados Unidos--- durante años (Moreno \oldstylenums{2014, 103}\index[nombres]{Moreno, Paz}).
\end{quotation}

%\newpage
%\pagestyle{empty}
%\null\vfill

\chapter{\mdseries La relación entre la fábrica y la devoción}\label{Capitulo_2}
\pagestyle{fancy}
\fancyhf{}
\fancyhead[RO,LE]{\hfill \textit{Capítulo 2. La relación entre la fábrica y la devoción} \hfill}
\fancyfoot[RO,LE]{\hfill \thepage \hfill}
% \markboth{Capítulo 2. La relación entre la fábrica y la devoción}{Capítulo 2. La relación entre la fábrica y la devoción}

\section*{\mdseries\large\textsc{2.1. Santa Ana Chiautempan (Contexto)}}
\addcontentsline{toc}{section}{2.1. Santa Ana Chiautempan (Contexto)}

% Los paquetes 'siunitx' y 'textcomp' tienen problemas con el paquete de fuentes 'Linux Libertine' en lo que se refiere a la escritura de grados y minutos (signos). Para resolver este problema, usaremos el acento normal, que se asemeja más al signo de minuto. Ugly hack :-(

\noindent El municipio de Santa Chiautempan se ubica en el estado de Tlaxcala (\emph{véase} \oldstylenums{Imagen \ref{chiautempan}, pág. \pageref{chiautempan}}). Según el \textit{Compendio de información geográfica municipal} \oldstylenums{2010}, Chiautempan cuenta con <<\oldstylenums{32} localidades y una población total de \oldstylenums{66\,149} habitantes. Está ubicado en el altiplano central mexicano a \oldstylenums{2\,280} metros sobre el nivel del mar, se sitúa en un eje de coordenadas geográficas entre los \oldstylenums{19}°\,\oldstylenums{19}\'\ \textsc{n} y \oldstylenums{98}°\thinspace\oldstylenums{12}\' \ \textsc{o}>> (\textsc{inegi}, \oldstylenums{2010}).

El municipio colinda, al norte, con Apetatitlán de Antonio Carvajal (\oldstylenums{002}), Contla de Juan Cuamatzi (\oldstylenums{018}) y San José Teacalco (\oldstylenums{052}); al este, con San José Teacalco (\oldstylenums{052}) y San Francisco Tetlanohcan (\oldstylenums{050}); al sur, con San Francisco Tetlanohcan (\oldstylenums{050}) y La Magdalena Tlaltelulco (\oldstylenums{048}); al oeste, con La Magdalena Tlaltelulco (\oldstylenums{048}) y Tlaxcala (\oldstylenums{030}) (\textsc{inegi}, \oldstylenums{2010}).

\subsection*{\mdseries\large\textsc{2.1.1. Superficie}}
\addcontentsline{toc}{subsection}{2.1.1. Superficie}

\noindent Comprende una superficie de \oldstylenums{77.09}\thinspace km\textsuperscript{2} lo que representa \oldstylenums{1.9}\% de la superficie total del estado, que asciende a \oldstylenums{3\thinspace991.14}\thinspace km\textsuperscript{2} (\textsc{inegi}, \oldstylenums{2010}).

\begin{figure}
\begin{center}
\renewcommand{\figurename}{Imagen}
\href{http://cuentame.inegi.org.mx/mapas/pdf/entidades/div_municipal/tlxmpios.pdf}{\includegraphics[width=14.6cm]{01}}
\caption[Estado de Tlaxcala]{Estado de Tlaxcala. Fuente: \textsc{inegi}}
\label{chiautempan}
\end{center}
\end{figure}

\subsection*{\mdseries\large\textsc{2.1.2. Clima}}
\addcontentsline{toc}{subsection}{2.1.2. Clima}

\noindent Posee un clima templado subhúmedo con lluvias en verano de mayor humedad (\oldstylenums{59}\%); templado subhúmedo con lluvias en verano de humedad media (\oldstylenums{37}\%) y semifrío subhúmedo con lluvias en verano de mayor humedad (\oldstylenums{4}\%). Tiene una temperatura que oscila entre los \oldstylenums{8 y los 16}°\,\textsc{c}; la precipitación pluvial es de \oldstylenums{800-1000 mm.}

\subsection*{\mdseries\large\textsc{2.1.3. Hidrografía}}
\addcontentsline{toc}{subsection}{2.1.3. Hidrografía}

\noindent Santa Ana Chiautempam está ubicada en la subcuenca del río Zahuapan ---cuenca del río Atoyac---, en la región hidrológica perteneciente al río Balsas. Sus corrientes de aguas son intermitentes durante todo el año; no posee cuerpos de agua.

\subsection*{\mdseries\large\textsc{2.1.4. Uso del suelo y vegetación}}
\addcontentsline{toc}{subsection}{2.1.4. Uso del suelo y vegetación}

\noindent Se ha destinado el \oldstylenums{62}\% a la agricultura y \oldstylenums{32}\% a su zona urbana; el \oldstylenums{6}\% es vegetación (bosque).

\subsection*{\mdseries\large\textsc{2.1.5. Uso potencial de la tierra}}
\addcontentsline{toc}{subsection}{2.1.5. Uso potencial de la tierra}

\noindent El \oldstylenums{52}\% se ha utilizado en la agricultura de tracción animal, el \oldstylenums{7}\% en la de tracción animal estacional y el \oldstylenums{9}\% en la mecanizada; el resto no es apta para la agricultura (\oldstylenums{32}\%). En lo que se refiere al aspecto pecuario, el \oldstylenums{52}\% se ha empleado para cultivo de tracción animal, el \oldstylenums{9}\% para maquinaria y el \oldstylenums{7}\% para el aprovechamiento de vegetación natural diferente al pastizal; el resto no es apto para uso pecuario (\oldstylenums{32}\%).

\subsection*{\mdseries\large\textsc{2.1.6. Zona urbana}}
\addcontentsline{toc}{subsection}{2.1.6. Zona Urbana}

\noindent La zona urbana se ubica sobre rocas ígneas extrusivas del neógeno y suelo aluvial del cuaternario, en lomerío de tobas; sierra volcánica con estrato volcanes o estrato volcanes aislados y llanura aluvial con lomerío de piso rocoso o cementado sobre áreas donde originalmente había suelos denominados luvisol y cambisol. Tiene climas templado subhúmedo con lluvias en verano, de mayor humedad, y templado subhúmedo con lluvias en verano, de humedad media, y está creciendo sobre terrenos previamente ocupados para agricultura (\textsc{inafed}, s/f).

\subsection*{\mdseries\large\textsc{2.1.7. Monumentos históricos}}
\addcontentsline{toc}{subsection}{2.1.7. Monumentos históricos}

\noindent El exconvento franciscano de Nuestra Señora de los Ángeles se edificó entre \oldstylenums{1564 y\,1585}. También es conocido como exconvento del Padre Jesús, y se encuentra dividido en un claustro bajo y otro alto. Dos contrafuertes enmarcan una fachada austera del templo, típica de la orden franciscana. Resaltan arcos de medio punto peraltados de columnas toscanas y una capilla posa donde se colocaba la custodia durante la procesión de Corpus Christi. El claustro alto, ocupado por una congregación de padres escolapios, posee arcos peraltados y rebajados sobre columnas toscanas en el primero y segundo nivel. En los muros de los corredores lucen capillas hornacinas neoclásicas y cuadros con pinturas al óleo de motivos de la orden, cuyo origen data de los siglos \textsc{xvii} y \textsc{xviii}.

La parroquia de Nuestra Señora de Santa Ana se sitúa frente al exconvento franciscano y es una construcción donde se aprecian diferentes estilos arquitectónicos. Su construcción inició en \oldstylenums{1626} y finalizó a mediados del mismo siglo. La fachada, construida básicamente con cantera gris, consta de dos cuerpos: en el primero se encuentra la entrada al edificio a través de un arco dovelado, flanqueado por columnas pareadas de estilo toscano. En el segundo cuerpo hay una ventana en forma de estrella perteneciente al coro, una representación del sol y la luna, y en el remate, una cruz. La fachada semeja un retablo donde aparecen las imágenes de Santa Ana, San Joaquín y la Virgen María; al centro, San José y el Padre Eterno. La torre a la derecha es de dos cuerpos con pilastras jónicas decoradas con motivos barrocos de argamasa, perteneciente al orden popular. Posee también arcos de medio punto, cornisa tablereada y en el remate, una bóveda con linternilla y cruz de hierro. En la parte posterior se encuentra una cúpula con linternilla que cubre el presbiterio.

La Parroquia de Nuestra Señora del Carmen se construyó en el siglo \textsc{xix} y cuenta con una fachada de estilo neogótico. Las portadas y las cúpulas están hechas de petatillo rojo, revestimiento de ladrillo y azulejo. Posee arcos ojivales y los remates son en forma piramidal. En el interior sobresale una serie de vitrales de origen francés, ubicados en la parte superior de los muros laterales. También presenta una techumbre que cubre la bóveda de forma ojival con arista y nervaduras. Es una de las construcciones religiosas que no presenta los típicos elementos arquitectónicos de la región tlaxcalteca.

El templo de La Soledad es una iglesia edificada en el siglo \textsc{xix} con materiales de piedra y adobe. Su fachada principal es de ladrillo y tiene un arco de medio punto que sirve de acceso, una ventana seguida de un arco de medio punto y, de remate, una cruz; se encuentra flanqueada por dos torres de dos cuerpos. En la parte posterior hay una cúpula con una gran linternilla. Su cubierta es abovedada a cañón corrido; se localiza en la calle Hidalgo Sur \oldstylenums{3} de la ciudad de Santa Ana Chiautempan.

La hacienda de San Juan Tzitzimpa fue construida en el siglo \textsc{xix}. Su cuerpo es rectangular a un solo nivel. Su fachada principal es de aplanado blanco y posee un torreón cercano al acceso principal; es de un solo cuerpo en forma circular y cornisa de pecho de pichón. Los
muros son de piedra y adobe, y la cubierta, de viguería de madera con tejamanil de forma plana. Esta hacienda contaba con un despacho, administración, capilla, tinacal, tienda de raya, bodega, zaguanes, caballerizas, macheros, troje, sillero, cocina, comedor, recámaras, establo, corrales y otra capilla fuera de la construcción. Se encuentra en la localidad de San Pedro Tlalcuapan, carretera de terracería San Bartolomé-San Pedro, de norte a sur, a \oldstylenums{1}\thinspace km (\textsc{inafed}, s/f).

\subsection*{\mdseries\large\textsc{2.1.8. Cultura popular}}
\addcontentsline{toc}{subsection}{2.1.8. Cultura popular}

\subsubsection{\mdseries\large\textsc{Artesanías}}

\noindent Actualmente, Santa Ana Chiautempan es un centro textil de gran relevancia. Aquí se pueden encontrar tanto factorías de grandes dimensiones como pequeños talleres familiares en donde aún se elaboran los textiles con fibras naturales, como la lana y el algodón. Sus productos comprenden una extensa variedad de colores: sarapes, jorongos, saltillos, cobijas y los famosos gobelinos, conocidos internacionalmente ---por el arte de pintar tejiendo---; además de sarapes, mantas de viaje, suéteres, cobertores, bufandas, entre otros. Por su calidad y reconocida belleza, estos productos tienen gran demanda en el extranjero (\textsc{inafed}, s/f\,).

\subsubsection{\mdseries\large\textsc{Gastronomía}}

\noindent En Santa Ana Chiautempan existe una gran variedad de alimentos como el mole prieto de guajolote (tlilmolli), la barbacoa de carnero, la barbacoa blanca de hoyo o en mixiote, los nopales y el guisado de haba. En dulces, frutas en conservas como el capulín, el tejocote, la
pera, durazno y calabaza. Esto, acompañado con la deliciosa bebida del pulque natural, que complementa la delicia de la gastronomía de este municipio (\textsc{inafed}, s/f\,).

\section*{\mdseries\large\textsc{2.2. El origen de la fábrica textil La Estrella y sus trabajadores}}
\addcontentsline{toc}{section}{2.2. El origen de la fábrica textil La Estrella y sus trabajadores}

\noindent Durante el Porfiriato se alcanzó la estabilidad económica del país. Se creó una infraestructura de transporte que permitió el desarrollo de la industria y fue entonces cuando se fusionaron una serie de hechos que dieron pauta a la implantación de la industria textil en Tlaxcala. Años antes ya se había iniciado la construcción del ferrocarril, con el gobierno de Benito Juárez, pero fue con Porfirio Díaz que se consumó la obra.

La \textit{pax} porfiriana sentó las bases para que los empresarios poblanos se decidieran a invertir en Tlaxcala, apoyados por el entonces gobernador, el coronel Próspero Cahuantzi (\oldstylenums{1885-1911}). El auge de las fábricas textiles en Tlaxcala empezó a ser detonado a mediados del siglo \textsc{xix}, esto se veía muy favorecido por la ubicación y la construcción de las vías del ferrocarril, que facilitó la transportación de materia prima y producto terminado.

Se instalaron fábricas en diversos puntos del estado, por ejemplo, en Amaxac de Guerrero, San Luis Apizaquito, en los municipios de Tetla y Zacatelco. Fue a principios del siglo \textsc{xx} que empezaron los cambios notables en las zonas rurales, originados por las industrias.

En la localidad de Amaxac de Guerrero, cerca de Santa Cruz, se instalaron dos fábricas textiles: Santa Elena y La Estrella. Esta última empezó sus funciones en \oldstylenums{1876}, fundada por el señor Agustín del Pozo, con residencia en la ciudad de Puebla. Para el funcionamiento de la fábrica ocuparon un antiguo molino de trigo que estaba en decadencia en ese mismo lugar.

\begin{quotation}
\noindent La Estrella funcionaba con maquinaria de origen europeo, que era movida por fuerza motriz de vapor, con una potencia de \oldstylenums{80} caballos de fuerza; en cuanto a las instalaciones esta contaba con un canal de abastecimiento que se localizaba en las cercanías de la fábrica Santa Elena, aprovechando el caudal del río Tequisquiatl que desembocaba en el Zahuapan, el cual abastecía a una turbina que hacía funcionar la planta hidroeléctrica que daba movimiento a la maquinaria. Daba ocupación a \oldstylenums{120} obreros, principalmente de la localidad y comunidades aledañas, entre las que se encuentran Santa Cruz Tlaxcala, San Bernardino Contla y Belén (Meza \oldstylenums{2009, 55}).
\end{quotation}

\noindent De acuerdo con Meza, Amaxac de Guerrero era una entidad poco comunicada, situación que causó serios problemas para cubrir las necesidades de la fábrica, a tal grado que en temporada de lluvias no había manera de transportar la materia prima para la producción. Los caminos eran malos y empeoraban mucho en esa temporada. De hecho, por ello, muchas veces la fábrica permanecía completamente incomunicada.

Como consecuencia, el dueño trasladó la fábrica a Santa Ana Chiautempan, en \oldstylenums{1908}, a sólo unos metros de la estación del ferrocarril, en la calle Bernardo Picazo núm.\thinspace\oldstylenums{1}, en Chiautempan. De esa manera, por la situación geográfica, la fábrica se vio muy favorecida económicamente. Pese a ello:

\begin{quotation}
\noindent En \oldstylenums{1912} aproximadamente Agustín del Pozo vendió la fábrica a la compañía Rafael Cobo y Secada, una sociedad mexicana, donde se reunieron algunas fortunas privadas en el seno de un grupo familiar, con la ayuda de amigos y allegados para invertir en el giro mercantil de la industria textil, que les permitió acrecentar su capital y poderío (Meza \oldstylenums{2009, 57}).
\end{quotation}

\noindent Eran varios dueños los que integraban la compañía. Rafael Cobo Secada, originario del estado de Veracruz, es el más recordado por la población, junto con el administrador, Fermín Solana Cobo; pero se reconocen también como propietarios a los señores Alfredo Cobo Carranza y Agustín Roji.

Cuando inició La Estrella, <<los propietarios radicaban en Puebla, desde donde solucionaban los problemas que ocurrían en la fábrica. Sus oficinas se ubicaron en la Calle \oldstylenums{3} Sur>> (Meza \oldstylenums{2009, 57}). Todavía se recuerda a la primera fábrica, le llaman La Estrella vieja, y la principal razón para explicar su ubicación es que Amaxac de Guerrero contaba con agua:

\begin{quotation}
\noindent Había unas cascadas y con eso se generaba la corriente [...] Y después ya la instalaron aquí en Santa Ana, y aun así la planta generadora de energía la tenían allá en Amaxac de Guerrero, entonces cuando se llegaba a ir la energía eléctrica, echaban a andar allá las turbinas para mandar corriente a la fábrica en Santa Ana [...] Se transportaba la energía para la fábrica por medio de cables (Andrés López Rosas, comunicación personal, \oldstylenums{23 de junio 2017}).
\end{quotation}

\noindent Para \oldstylenums{1907-1908}, cuando la fábrica se instaló en Chiautempan, tenía otras ventajas a su favor. Una de ellas es que siempre contaba con energía eléctrica por la explotación del agua en Amaxac de Guerrero. Otra ventaja era su cercanía con el ferrocarril que cargaba y descargaba material sumamente necesario para la producción y, después, se llevaba el producto terminado. Cuando empezó la fábrica en Chiautempan, ya se habían introducido nuevos avances tecnológicos que facilitaban la producción. Además, el número de artesanos había crecido y se había especializado en diferentes artesanías, como el tallado de madera, elaboración de cestas y productos de barro, herramientas de agricultura, etcétera. Finalmente, Chiautempan ya estaba incluida entre los pueblos que se habían desarrollado, pero también los pueblos circunvecinos: La Magdalena, San Bernardino Contla, Atlihuetzía, Santa Cruz Tlaxcala, Guadalupe Ixcotla, Tetla, San Pedro Xochiteotla, Belén.

Todos estos pueblos tenían una tradición artesanal y estaban constituidos por gente trabajadora y dispuesta a enrolarse en las industrias. En ese tiempo había una gran movilidad de personas dispuestas a trabajar dentro y fuera del estado. Los obreros de La Estrella eran ejemplo de este fenómeno (Santibáñez \oldstylenums{2013}).

Al ser instalada la fábrica, el pequeño comercio de Chiautempan se desarrolló ampliamente, la situación económica fue mejorando para la fábrica y también para la comunidad. El tren debía detenerse obligadamente en la estación del ferrocarril para la llegada de la materia prima y para llevarse las telas que ahí se producían:

\begin{quotation}
\noindent Las telas se llevaban a varias partes de la República, pero también fuera del país, se las llevaban por ferrocarril. Yo pocas veces vi un camión carguero llevando mercancía de la fábrica, todo se transportaba o casi todo por el ferrocarril. Además, llegaba el algodón para producir las telas, y comentan que también dejaban petróleo para las necesidades de la fábrica, dejaban un furgón para el consumo [...] Era su principal medio de transporte para la materia prima y para el producto acabado (Andrés López Rosas, comunicación personal, \oldstylenums{23 de junio 2017}).
\end{quotation}

\noindent La Estrella fue la primera fábrica de hilados y tejidos de algodón en Santa Ana Chiautempan. José Corona (ya fallecido) comentaba al recordar su ingreso a la fábrica sobre los salarios y las primeras fábricas:

\begin{quotation}
\noindent En ese tiempo, cuando empecé a trabajar, ya estaba La Providencia, la Xicoténcatl, que ahora es la Lanera Moderna, pero todos sabemos que la primera y única fábrica de hilados y tejidos de algodón en el pueblo, en Chiautempan, fue La Estrella, yo trabajé allá [...] Cuando comencé a trabajar en La Estrella me pagaron, pero muy barato. Primero fui aprendiz, aprendí pronto, luego fui ayudante, luego ya fui oficial. Cuando era aprendiz me daban \oldstylenums{10-20} centavos, pero era dinero. Yo le daba el dinero a mi mamá. Después ya fui ayudante, ya me puse abusado. Cuando era yo ayudante, ya casi era yo oficial, pero no podía yo adquirir la plaza de oficial, tenía que pasar por la de ayudante. Después de como tres años, ya fui oficial, mi trabajo fue siempre en los telares. De ahí pasaba a la tintorería, al blanqueo, pero yo siempre estuve en el tejido, ahí era mi fuerte. En La estrella se producía manta de la buena, de \oldstylenums{50-60-70} cm de ancho. Anduve en la preparación, pero no trabajando, sólo aprendiendo, luego iba a meter las manos el señor Vicente López y yo, él ya es difunto, fuimos a estudiar tecnología a Tlaxcala, en la \textsc{crom}. Vino un señor de Orizaba a enseñarnos y sí aprendimos [...] Para ir a aprender nos nombró el sindicato, para que después viniéramos a explotar el Contrato Colectivo de Trabajo. Entonces ya era representante López Galindo (José Corona, comunicación personal, \oldstylenums{20 de junio de 2010}).
\end{quotation}

\subsection*{\mdseries\large\textsc{2.2.1. El aprendizaje del diálogo a señas}}
\addcontentsline{toc}{subsection}{2.2.1. El aprendizaje del diálogo a señas}

\noindent Una parte negativa de estos trabajos era el ruido, y considerando que los trabajadores debían permanecer mucho tiempo en un ambiente cerrado y ensordecedor, además de que precisaban cumplir con el horario laboral, era necesario tener y entender un lenguaje que les permitiera comunicarse entre sí. Es por eso que comenzaron a practicar un lenguaje mediante señas.

Las medidas exactas del salón donde trabajaban se desconocen, pero según los que conocieron el lugar, éste era muy grande, hecho que complicaba más la comunicación oral. José Corona recuerda lo siguiente:

\begin{quotation}
\noindent Allá en la fábrica nos comunicábamos por señas de lado a lado. Era muy fácil comunicarnos así, aunque había personas que no entendían, pero la mayoría platicábamos así. Cuando empecé a trabajar no sabía las señas, pero dilaté poco: en medio año ya las conocía, para entonces ya era ayudante (José Corona, comunicación personal, \oldstylenums{20 de junio de 2010}).
\end{quotation}

\noindent La jornada de trabajo era de ocho horas, que debían cumplir junto a las máquinas ruidosas. Ellos tenían que satisfacer sus necesidades fisiológicas básicas, pero también las sociales. Sobre estas expresiones, el cronista de Chiautempan comentaba algo muy interesante:

\begin{quotation}
\noindent Hay un relato de los primeros tiempos de las fábricas, creo que hasta había una canción, donde pasaba la mujer con un cubo. Obviamente el cubo de madera y una jícara, y al que pedía agua le daba agua con la jícara, y la canción, si no mal recuerdo, se cantaba primero en náhuatl, pero se entendía a base de señas. Ellos hablaban y leían los labios. Otra cosa es que para poder ir al baño le tenían que decir a su compañero con señas, para que éste le cuidara su máquina (Rogelio Flores, comunicación personal, \oldstylenums{30 de octubre de 2010}).
\end{quotation}

\noindent A pesar de la creciente economía, la presencia del ferrocarril en Tlaxcala y el impulso a la industria de Porfirio Díaz, también se vivían tiempos difíciles. Los cambios positivos no se dieron de inmediato:

\begin{quotation}
\noindent Las primeras fábricas contaron con mano de obra barata: artesanos desocupados, campesinos desposeídos de sus medios de producción y campesinos cuyas parcelas no les proporcionaban los recursos necesarios para su subsistencia, pero a las cuales continuaron aferrados (Santibáñez \oldstylenums{1991, 63}).
\end{quotation}

\noindent En los años posteriores la situación continuaba complicada para los trabajadores, las oportunidades no eran muchas. De alguna manera se adecuaban a lo que había; por ejemplo, José Corona expresaba: <<yo me hice obrero porque no había otra cosa, lo único era eso, por eso la
mayoría íbamos para allá>> (José Corona, comunicación personal, \oldstylenums{20 de junio de 2010}).

En \oldstylenums{1933 había entre 150 y 160} obreros en La Estrella y se trabajaban tres turnos: el primero era de mayor importancia, porque en él estaba el mayor número de trabajadores de base y quienes tomaban las decisiones importantes. En el segundo turno había menos trabajadores, y en el tercero, todavía menos:

\begin{quotation}
\noindent Yo trabajaba ocho horas, empezaba a las siete de la mañana, salíamos a comer a la una, y a las dos entrábamos [...] Cuando salíamos a comer se paraban las máquinas, nadie se quedaba a trabajar, todos salíamos a comer [...] De trabajar salíamos a las cuatro o a las cinco, porque pagábamos una hora para que el sábado nomás fuera medio día (José Corona, comunicación personal, \oldstylenums{20 de junio de 2010}).
\end{quotation}

\noindent Otra parte importante dentro de la vida laboral era el sindicato. Al respecto, Corona menciona cómo se conformaba:

\begin{quotation}
\noindent Elegíamos al representante sindical por votación, sólo se decía: ¡a votación! Esto se hacía nomás con el primer turno, para elegir al Secretario General. Para Secretario Auxiliar ya se convocaba para elecciones al segundo turno también. Después del trabajo se hacía todo, salíamos a las \oldstylenums{16:00 o a las 17:00 horas}, y nos íbamos al sindicato. Todos concurríamos, porque entonces votábamos sólo una vez, sólo un acuerdo, y dijimos ¡el que no venga, lo vamos a castigar un día! Decíamos: ¡aquí es obligación venir a la asamblea!, ya por el miedo, allí los teníamos. Yo fui secretario por seis meses. Se nombraban: Secretario General, Secretario de Interior, Secretario del Exterior, Secretario de Actas. Cada seis meses era cambiar, dilataba la representación seis meses y no había reelección. Antes peleábamos la secretaría en asamblea, con política peleábamos y en la votación. La última vez, el último secretario fui yo, el que estuvo por votación de los trabajadores, fue en \oldstylenums{1931 o 1932}. Después fue el señor López Galindo, él sí se quedó mucho tiempo, él era el que dirigía. Yo trabajé en La Estrella \oldstylenums{33} años, dilaté mucho tiempo, ya no aguanté allá. Me pasé a la Xicoténcatl, me decían que ahí ganaban bien, era igual. Luego me decían que la Santa Teresa era igual [...] Por fin me aburrí y me salí, dejé las fábricas y empecé a trabajar por mi cuenta. Yo andaba vendiendo ropa en algunos lugares. Pero me cansé y dejé los viajes y puse mi tocinería; vendía mucha longaniza y chorizo, porque era muy sabrosa la longaniza. Me han pedido la receta, pero no se la he dado a nadie (José Corona, comunicación personal, \oldstylenums{20 de junio de 2010}).
\end{quotation}

\subsection*{\mdseries\large\textsc{2.2.2. El deporte dentro de la fábrica}}
\addcontentsline{toc}{subsection}{2.2.2. El deporte dentro de la fábrica}

\noindent Como una medida para evitar que el obrero faltara al trabajo el lunes, se probó con la práctica de un deporte: el beisbol. José Corona nos narró un poco sobre este asunto:

\begin{quotation}
\noindent Yo empecé a jugar en mi barrio, el barrio de San Onofre. Ahí crecí y nos reuníamos todos los muchachos, formaron un equipo en La Estrella y me llamaron a que jugara con ellos. Trajeron a elementos de afuera para jugar en el equipo de La Estrella, y sí, fue una novena muy fuerte. Afortunadamente Dios me ayudó mucho y yo era superior a los refuerzos que habían llegado. Entonces los de los refuerzos me hicieron política y yo que me chiqueo y ya no me presenté a jugar. Empecé a los \oldstylenums{18} años, a los \oldstylenums{20} ya jugaba con pelota buena, yo fui \textit{pícher}, y el \textit{pícher} de los refuerzos no pudo sostener nada, le metieron seis carreras, y nomás hizo dos \textit{auts}, no pudo hacer el tercero. Y yo, obligado a meterme, no me metieron ninguna carrera. Ya después el administrador les dijo, refiriéndose a mí, ¡por qué no lo metieron desde el principio, ya les demostró que es superior a ellos! Cuando trabajé en La Estrella sí me alcanzaba el dinero. En esa época todo era económico, y yo era uno de los mejores que ganaba. En los años de \oldstylenums{1933-34} la empresa simpatizaba conmigo, porque daba buena producción. Los empresarios, para mí, eran buenas gentes. El señor don Fermín era el administrador, para mí fue siempre una buena persona, y los empleados también, para qué me voy a quejar de ellos... no. Había unos flojos, nomás les gustaba platicar, se iban a fumar, por eso el patrón les exigía (José Corona, comunicación personal, \oldstylenums{20 de junio de 2010}).
\end{quotation}

\subsection*{\mdseries\large\textsc{2.2.3. Vida familiar y trabajo}}
\addcontentsline{toc}{subsection}{2.2.3. Vida familiar y trabajo}

\noindent La narración de la historia de vida de José Corona nos da cuenta de la forma de vida que se llevaba en aquella época entre el trabajo, las relaciones sociales y la familia, situación que estuvo marcada e influida por la vida laboral obrera.

\begin{quotation}
\noindent Cuando me casé, tenía \oldstylenums{33} años. El dinero que yo ganaba le daba a mi mamá semanario. Cuando me casé, le seguí dando pero ya no quiso mi mamá el semanario, me dijo: <<ahora dale a tu mujer>> y mi mujer no lo quería recibir. Pero siempre seguimos mirando a mi mamá. Yo tuve una hermana, pero yo era el único hombre, yo cuidé a mi mamá hasta que se murió, yo la atendí en todo. En esa época me ayudó el doctor Prado, en ese tiempo vino de nuevo, yo fui su primer paciente-cliente, y lo seguí viendo, después que atendió a la difunta de mi mamá. Mi esposa nomás tuvo cuatro hijos. Y alcanzaba bien el dinero. Cuando me casé ganaba \oldstylenums{17-18} pesos, era yo uno de los mejores y ganaba más. Eso era porque producía yo. Otros ganaban \oldstylenums{10-12} pesos cuando mucho; otros ganaban \oldstylenums{13-14}, ellos tenían cuatro máquinas. Yo tenía dos y producía bastante. Después tenía un buen lugar, pero las envidias, ya no sabían cómo quitarme de ese lugar. Y me asignaron tres telares, pero yo no los quería, porque eran planos, yo quería de \textit{catiles}, y me dijeron: <<pues si quieres, si no, no hay nada>>. Bueno, \textit{pos} me pasé, eso fue de acuerdo al sindicato, el administrador no se metía en eso. En todas partes hay políticas. \textit{Pos} no hubo más que dejar mi buen lugar y me pasé a los tres. Pero como yo sabía trabajar, comencé a arreglar los telares, a apretar bandas, y comencé a arreglar todo bien, y después los telares trabajaban divinamente. Eso les daba envidia. Que cómo lo hacía. ¿Cómo? \textit{Pos} arreglando las máquinas, entonces las arreglaba a mi modo \textit{pa} que me rindiera la producción. Y \textit{pa} que tuviera más velocidad la máquina, ajustaba yo bandas, y aparte de bandas le metía yo jabón a la polea pa que jalara más la banda, pero como antes ya las había arreglado de las protecciones, \textit{pa} que no tuviera problema de machucones, pero todo eso era con tiempo. Los iba yo arreglando, yo no decía nada, y sólo yo metía mano, arreglaba yo mis telares, y a producir. Eso me daba gusto y el administrador me decía que era yo el mejor trabajador, y que muchos de ellos, los que ya tenían años, no podían producir lo que yo producía. Yo siempre me manejé bien con los trabajadores, y me apreciaban mucho. Hubo compañeros que me traían el pollito o el totolito, yo les dije: <<¡miren esto! Llévenselo a sus hijos, háganle una comida sabrosa, y cómanselo ustedes. Yo nunca les recibí nada, yo les decía mejor cuando nos encuéntremos (\textit{sic}) en el camino [...] Yo nomás les decía así. Me invitan un pulque, yo se los acepto, pero sólo era la plática, porque nunca lo hice. Yo tenía compadritos de los pueblos, casi por lo regular hubo mucha gente de Contla, San Bernabé, Guadalupe Ixcotla; de aquí eran pocos>>.
\end{quotation}

\subsubsection{\mdseries Los amigos}

\begin{quotation}
\noindent Honradamente sí me gustaban, y de alcohol poco, yo no me aparto de la razón. El sábado yo me juntaba con los amigos, pero antes yo les decía: <<¡espérenme, voy al baño!>> Iba yo al baño y me apartaba yo el dinero \textit{pa} la casa, para todo y ya nomás me quedaba con unos cuantos centavos, y es lo que gastaba, pero de la raya y todo, no agarraba nada. (José Corona, comunicación personal, \oldstylenums{20} de junio de 2010).
\end{quotation}

\subsection*{\mdseries\large\textsc{2.2.4. Algunos cambios en la comunidad}}
\addcontentsline{toc}{subsection}{2.2.4. Algunos cambios en la comunidad}

\noindent Con las actividades de carga y descarga en la estación del ferrocarril, también se abrió paso el pequeño comercio. Llegaban mercancías baratas, plátanos de diferentes tipos del vecino estado de Veracruz y se llevaban productos artesanales de la comunidad. Este intercambio favoreció al pequeño comercio, porque se amplió el mercado de los artesanos, lo que trajo grandes beneficios a Chiautempan y a sus alrededores (Flores \oldstylenums{2010}).

Por otro lado, los principales protagonistas, los obreros, pudieron ver en las fábricas fuentes de trabajo e ingresos. Con la llegada de la industria textil se dieron cambios profundos para toda la comunidad. De acuerdo con Santibáñez:

\begin{quotation}
\noindent Se rompió la rutina y organización tradicional de esas comunidades. Ahora la agricultura y las artesanías, que por siglos dieron ocupación a la inmensa mayoría de sus habitantes, se combinaban con el trabajo industrial. En esa situación, las fábricas se beneficiaron actuando como imán en la captación de mano de obra segura y redituable, en una zona donde las perspectivas de progreso económico eran precarias (Santibáñez \oldstylenums{1991, 65}).
\end{quotation}

\noindent De esa manera se benefició la industria textil y la población mejoró en sus ingresos. Los obreros, por su parte, ya con más entradas de dinero, pudieron crear cierto bienestar para las familias tlaxcaltecas. Trabajar en la fábrica fue una actividad que permitió al trabajador textil tener un salario fijo y así continuar viviendo con sus labores del campo y sus compromisos religiosos.

\subsection*{\mdseries\large\textsc{2.2.5. Los trabajadores y sus primeros años de trabajo}}
\addcontentsline{toc}{subsection}{2.2.5. Los trabajadores y sus primeros años de trabajo}

\noindent Los obreros empezaron a trabajar en La Estrella muy jóvenes, entre \oldstylenums{14 y\,15} años, aunque de manera informal, desde más chicos; en aquel entonces era posible incorporarse en las actividades fabriles desde los \oldstylenums{12} años. Se empezaba como aprendiz, ganando \oldstylenums{10-20} centavos; luego se convertían en ayudantes. Después podían llegar a ser oficiales, aunque ya también podían ser aspirantes a obreros, pues ya habían adquirido el tiempo, la experiencia necesaria y el \textit{derecho} para serlo.

Un grupo de extrabajadores comentó que la faena en la fábrica no era fácil, tenían que saber defenderse. Recordaban cuando los cambiaron de departamento, para ellos el cambio no resultó favorable, era menos dinero. Recordaba Polo: <<porque a mí me llevaron con mi salario de aquel entonces, \$\oldstylenums{1.65}. Me cambiaron al secador, y ahí era el salario de \$\oldstylenums{1.05}>> (Polo, comunicación personal, \oldstylenums{27 de agosto de 1989}).

Asimismo, el señor Isabel añadía:

\begin{quotation}
\noindent Pero después nos juntábamos y hacíamos presión para que nos aumentaran a \$\oldstylenums{\thinspace9.90}, que es lo que sacábamos antes a la semana, así que eran \$\oldstylenums{1.65} por día. Nosotros sólo teníamos que decir: <<no tiene caso venirse a amolar en el calor, mejor estar en nuestro otro lugar>>. Allá no estábamos tan fatigados con el calor (Isabel Lima, comunicación personal, \oldstylenums{27 de agosto de 1989}).
\end{quotation}

\noindent En ese tiempo, para mejorar las condiciones, los obreros presionaban al maestro, con quien tenían contacto directo; después, al representante sindical, argumentando sus motivos y desacuerdos con el salario. El representante era quien se encargaba de gestionar ante la empresa un mejor salario para los obreros.

En ese tiempo, para mejorar las condiciones, los obreros presionaban al maestro, con quien tenían contacto directo; después, al representante sindical, argumentando sus motivos y desacuerdos con el salario. El representante era quien se encargaba de gestionar ante la empresa un mejor salario para los obreros.

\subsection*{\mdseries\large\textsc{2.2.6. Los obreros y el campo}}
\addcontentsline{toc}{subsection}{2.2.6. Los obreros y el campo}

\noindent A pesar de que en Tlaxcala el campo no ha sido redituable por los precios del maíz, del frijol y por la siembra de temporal, los obreros querían continuar ligados a la tierra. Esto, sin embargo, no ocurría del todo: al ingresar a la fábrica, hacían de la actividad fabril la más importante de su vida.

A través del tiempo, trabajar en La Estrella comenzó a formar parte de la tradición familiar. Fue entonces cuando en la misma fábrica se podían encontrar a padres e hijos, hermanos, tíos y sobrinos, porque era frecuente que los mismos trabajadores recomendaban a sus familiares con el representante sindical, de esa manera compartían y heredaban a la familia el trabajo fabril.

\subsection*{\mdseries\large\textsc{2.2.7. Educación}}
\addcontentsline{toc}{subsection}{2.2.7. Educación}

\noindent Cabe mencionar que la educación fue un factor muy importante entre los obreros de ese tiempo, el nivel de escolaridad no era alto. Conjuntamente, había una economía incipiente en el país y no se generaban muchas oportunidades. Por otra parte, ellos necesitaban trabajar y
la fábrica requería trabajadores. Fue una oportunidad para los obreros, que podían realizar una labor honesta, que los hacía ser y sentirse personas significativas en su comunidad. Algunos de ellos, como Isabel Lima, aun después de la salida de la fábrica, continuaron en la lucha para beneficio de los jubilados y sus familias.

Los obreros de La Estrella, aunque no tenían muchos estudios ni contaban con diplomas, sacaron adelante la producción por muchos años; además, dieron buen mantenimiento a los telares viejos y siguieron produciendo en esas condiciones.

\section*{\mdseries\large\textsc{2.3. Santo patrón de la fábrica}}
\addcontentsline{toc}{section}{2.3. Santo patrón de la fábrica}

\noindent En esa época, era común que se realizaran fiestas patronales en las fábricas. En la ciudad vecina de Puebla se veneraba con devoción las imágenes católicas y construían capillas como máximo culto al Santo Patrón respectivo. Los dueños de La Estrella, al vivir en la ciudad de Puebla, también implantaron la misma costumbre en Chiautempan:

\begin{quotation}
\noindent El dueño y su familia vivían en la ciudad de Puebla. Fueron las hijas del dueño, quienes tenían mucha devoción al Sagrado Corazón de Jesús, las que organizaban en un principio. Eran ellas las que invitaban a las autoridades eclesiásticas de la ciudad de Puebla para la celebración, que en ese entonces era financiada por los dueños, y se realizaba la misa en el interior de la fábrica (Martín Ramírez, comunicación personal, \oldstylenums{29 de junio de 2014}).
\end{quotation}

\noindent Los dueños de La Estrella eran españoles y posiblemente tenían mucha devoción al Sagrado Corazón. En los años en que estaba activa la empresa se mantenían dos imágenes en su interior. Una de ellas, la más grande, tenía su altar fijo en el salón grande, y la otra se encontraba en un salón de menor tamaño. La imagen grande tenía su fiesta cada año en el interior de la fábrica, donde se realizaba una misa con las altas autoridades eclesiásticas; cuando no había obispo en Tlaxcala, iba el de la ciudad de Puebla y después, en los últimos años, el de Tlaxcala.

Martín Ramírez Reyes empezó a trabajar en la fábrica en \oldstylenums{1945}. Conocía bien a los dueños porque era chofer del administrador, y el administrador era hermano del dueño.

\begin{quotation}
\noindent Desde que yo recuerdo, más o menos en \oldstylenums{1930}, ya celebraban la fiesta, porque esa imagen del Corazón de Jesús la donaron las patronas, las hijas de Don Cobo Secada. Los dueños de la fábrica eran de España y muy católicos. Cuando vivían, la misa se hacía a las \oldstylenums{12:00 o a las 13:00, aunque mayormente a las 13:00} horas. Entonces las patronas traían padres de Puebla. Había cocina, traían a su cocinera y la cocinera hacía algunos platillos para comer. Ellas fueron muy allegadas a la iglesia, se llevaban mucho con el señor obispo de Puebla, cuando él venía se quedaba a comer, o los sacerdotes también (Martín Ramírez, comunicación personal, \oldstylenums{29 de junio de 2014}).
\end{quotation}

\section*{\mdseries\large\textsc{2.4. La fiesta del Sagrado Corazón en el ámbito laboral}}
\addcontentsline{toc}{section}{2.4. La fiesta del Sagrado Corazón en el ámbito laboral}

\noindent Las fiestas religiosas se hacen principalmente como culto de agradecimiento a la deidad, pero también como agradecimiento al trabajador. En muchos pueblos se acostumbra a hacer una fiesta con cohetes, con teponaztle, comida, vendimia, buñuelos y tamales, siempre respetando la misa llevada a cabo en las fábricas. Como ejemplo de estas fiestas, la fábrica La Luz rendía culto a Nuestra Señora de La Luz y La Estrella al Sagrado Corazón de Jesús.

En el caso que nos atañe, dueños y obreros dedicaban un día de junio para dar gracias. Los obreros podían convivir entre ellos en la fiesta, a la que llevaban a sus familias. La fiesta era esperada por todo el pueblo, una especie de feria que transcurría dentro y fuera de la fábrica.
Se ponían los caballitos, la lotería, el palo encebado y había espacio para otros juegos (Rogelio Flores, comunicación personal, \oldstylenums{30 de octubre de 2010}):

\begin{quotation}
\noindent Había una misa, aquí nunca falta la misa, es por nuestra cultura [...] Los obreros hacían su convivio durante la mañana, por la tarde había palo encebado, el juego del barril. Era un barril engrasado y ganaba el que más aguantara, entonces no venía la rueda de la fortuna. Eran los juegos naturales de las costumbres de los pueblos. Había también lucha libre, \textit{box} y lo que ellos podían hacer. Eran los ciudadanos de un pueblo [...] Entonces se hacía el festejo y el patrón presidía, pero se retiraba temprano. Y los otros se quedaban con su fandango hasta la noche. Era una fiesta del pueblo, y claro porque La Estrella llegó a ser una fábrica importante por su producción de algodón, y el pueblo se congregaba a participar de la felicidad de los obreros (Fabio Gracia, comunicación personal, \oldstylenums{1 de julio de 2014}).
\end{quotation}

\noindent Son muchos los recuerdos que los extrabajadores tienen sobre la fiesta del santo patrón. El señor Isabel recordaba muy emocionado que:

\begin{quotation}
\noindent En ese entonces se hacían muchas invitaciones a todas las fábricas: a La Luz, a La Lanera Moderna, a La Teresa, a Telafil. Todas acudían con su comisión y su pabellón, como si fuera una fiesta patria. Ya después prohibieron que salieran los pabellones a las fiestas profanas. Los pabellones debían salir netamente, solamente en días patrios. Por eso nosotros mandamos a hacer una bandera con el Sagrado Corazón de Jesús. La fábrica de Santa Teresa tenía su bandera con la Virgen de Santa Teresita. Y la Xicoténcatl, esta salía con su pabellón. Pero todos nos presentábamos a los ocho días de nuestra fiesta de la fábrica de La Estrella a la festividad de la Xicoténcatl, o
sea, La Lanera Moderna. También acudíamos con nuestra gruesa de cohetones, y ahí vamos. El desfile iniciaba en la fábrica de La Estrella, de donde era la festividad; pasábamos por El Refugio, de Don Rufino Moreno; y ya luego bajábamos a traer a los de La Providencia, y luego nos bajábamos para dar la vuelta a traer a la Xicoténcatl. Y luego pasábamos atraer a los de La Santa Teresa. Y luego ya nos íbamos a traer a los de Telafil; no estaba Texfil ni todas esas. De manera que el desfile empezaba, se organizaba con las diferentes comisiones. Los encargados teníamos la obligación de pasar por un pabellón, por uno y por otro. Porque entonces se hacía allá cerquita de la estación del ferrocarril. La última misa la vino a celebrar el señor Obispo. Está el zaguán aquí, se hizo un campito aquí a un lado, se pusieron manteados, pegadito a la entrada de la fábrica. Había que organizar juegos de cucaña. Los juegos de cucaña pues era el barril con determinados premios. Había encostalados, carreras de encostalados, también había palo encebado. Para mí era más bonito entonces, había toda esa clase de juegos. Eso ya era en la noche, cuando estábamos todavía en actividad había baile. Nomás que aquí está parejito, pero ahí estaba empedrado [...] Y así, pues
ahí bailar. A veces nos íbamos al sindicato pues estaba cerquita. Había una orquesta que tocaba para el baile, mas antes no se conocían los conjuntos musicales. Hoy ya se conocen los conjuntos. Un conjunto de una forma, de otra forma y de otra forma. Como luz y sonido, no se acostumbró en ese entonces. El baile se terminaba como a la \oldstylenums{1:00 o 2:00}. Acabábamos porque hacíamos la festividad el domingo y el día lunes no trabajábamos (Isabel Lima, comunicación personal, \oldstylenums{27 de agosto de 1989}).
\end{quotation}

\noindent De la misma manera, Fabio recuerda:

\begin{quotation}
\noindent Entonces se hacía el festejo y el patrón presidía, pero se retiraba temprano. Y los otros se quedaban con su fandango hasta la noche. Los que mantenían el orden eran los veladores de la fábrica, cuando alguien tenía mal comportamiento. Era una fiesta del pueblo, y claro que La Estrella llegó a ser una fábrica importante, y el pueblo se congregaba a participar de la felicidad de los obreros (Fabio Gracia, comunicación personal, \oldstylenums{1 de julio de 2014}).
\end{quotation}

\noindent Rogelio Flores comenta que, entre \oldstylenums{1945 y 1950}:

\begin{quotation}
\noindent Se acomodaba un \textit{ring} y se hacían unas peleas de \textit{box}. En esta calle, que es muy larga, se ponía un gallo amarrado de cabeza y pasaba el jinete y le cortaba la cabeza al gallo. Se llamaba <<corta gallo>>, <<pasa gallo>>, se conoce de muchas maneras. También había un juego en donde se ponía un palo, un poste y se llenaba de cebo y arriba se ponían unas valijas, unas maletas, un
envoltorio. Se llamaba el palo encebado. El que se subía al palo se robaba o se ganaba lo que había allá arriba. De modo que cuando llegaba una persona a una fiesta y se ponía un traje chico, le decían que en dónde había sido el palo encebado, dando a entender que había ido a una fiesta y se había subido al palo para quedarse con el premio (Rogelio Flores, comunicación personal, \oldstylenums{30 de octubre de 2010}).
\end{quotation}

\noindent Rosita Lima recuerda así esos días de fiesta:

\begin{quotation}
\noindent Yo me acuerdo de que mi mamá nos hacía nuestro vestido y ya nos llevaba a la misa a la fábrica [...] Adornaban también adentro de la fábrica, se hacía la misa, luego la comida, luego había función de \textit{box}, había baile. La fiesta se terminaba con baile, hacían como una \textit{kermesse} y todo lo que se consumía era regalado (Rosita Lima, comunicación personal, \oldstylenums{3 de noviembre de 2010}).
\end{quotation}

\noindent La gastronomía era abundante, había una variedad de comida típica:

\begin{quotation}
\noindent Aquí son muy característicos los tamales, buñuelos, el atole y algo que se ha perdido son los famosos tacos de estación que hicieron famosos los de La Estrella: dos tortillas con papas, con rajas, papa capeada o con bistec. Así eran muchas veces los trabajadores de la fábrica La Estrella, así era la fiesta. Se jugaba también a la lotería. En ese entonces estuvo en auge la famosa Lotería Nacional, así empezaban diciendo: ¡el barril, la luna! Todo se jugaba en tandillas en mesas largas, con semillas (Rogelio Flores, comunicación personal, \oldstylenums{30 de octubre de 2010}).
\end{quotation}

\subsection*{\mdseries\large\textsc{2.4.1. La fiesta y su patrocinio}}
\addcontentsline{toc}{subsection}{2.4.1. La fiesta y su patrocinio}

\noindent Fabio Gracia menciona que:

\begin{quotation}
\noindent La Fiesta del Sagrado Corazón de La Estrella \textit{la hacían muy en grande}, seguramente por el entusiasmo que se vivía en ese tiempo en la fábrica, cuando había pocas distracciones ---yo recuerdo que en ese entonces la única distracción que había era ver pasar el tren--- seguramente que influía, pero también por otras razones es que se recuerda con tanto cariño la fiesta, en la que podía participar toda la población de Chiautempan (Fabio Gracia, comunicación personal, \oldstylenums{1 de julio de 2014}).
\end{quotation}

\noindent En principio, lo más probable es que fuera patrocinada por los propietarios de la fábrica. El Sagrado Corazón tenía un nicho permanente en la fábrica, eso explica la devoción de los dueños por la imagen, como manifiesta Martín Ramírez:

\begin{quotation}
\noindent El dueño y su familia vivían en la ciudad de Puebla y fueron las hijas del dueño quienes tenían mucha devoción al Sagrado Corazón de Jesús, las que organizaban en un principio. Eran ellas las que invitaban a las autoridades eclesiásticas de la ciudad de Puebla para la celebración, que en ese entonces era financiada por los dueños, y se realizaba la misa en el interior de la fábrica. Había carreras de encostalados, palo encebado, mucha música [...] Pero lo que más me gustaba era el barrilito. El barrilito era un barril común y corriente, tenía agujeros por en medio, le ponían una pieza de franela; el que se la quería ganar, tenía que subir al barril. Si lograba llegar a la meta, se lo ganaba, si no se caía. Todo era para diversión de la gente. El palo encebado, le ponían cosas también, y el que lograba subir se llevaba todas las cosas [...] Se fueron acabando las patronas y quedó nada más la tradición, la costumbre y después el Secretario General, al que le llamaban \textit{El Líder}, agarró la costumbre de organizar las fiestas. El sindicato, con los años, fue tomando fuerza, y fue cuando también los dirigentes de La Estrella, que también tenían devoción al Sagrado Corazón, dirigieron también la festividad de la fábrica dedicada al Sagrado Corazón. Después la fiesta ya la organizaba el líder, los del sindicato, participaban las familias; invitaban a comer, allá comían, hacían su mole, su fiesta. Acabando de eso ya venían de enfrente de la fábrica a ver las luchas, el \textit{box} y los otros juegos, hasta que se acababa. Era de cada año, ahora ya nomás como viene quedando los que tuvimos suerte de ver [...] Todavía nos acordamos de cómo fue. Los que no, ni lo saben. Muchos niños, muchos jóvenes ya no saben de eso (Martín Ramírez, comunicación personal, \oldstylenums{30 de junio de 2014}).
\end{quotation}

\noindent Años más tarde, los sindicatos ya se habían fortalecido, y fue entonces que las autoridades sindicales de La Estrella tomaron la dirección de la celebración al Sagrado Corazón.

\begin{quotation}
\noindent En ese entonces las festividades las costeaba el sindicato, entonces no dábamos nada, nosotros nada más éramos servidores, lo que nos asignaba el comité. Que hay que invitar caballitos y todas esas cositas (Isabel Lima, comunicación personal, \oldstylenums{1989}).
\end{quotation}

\subsection*{\mdseries\large\textsc{2.4.2. La función de los obreros en la fiesta}}
\addcontentsline{toc}{subsection}{2.4.2. La función de los obreros en la fiesta}

\noindent Por haber tenido la experiencia de la realización de la fiesta año tras año, ya tenían establecidas las comisiones para su realización. Eran los mismos obreros, guiados por sus dirigentes sindicales, los encargados de organizarse entre ellos. Se hacía una reunión previa para formar las comisiones que les permitían abarcar todas las tareas necesarias:

\begin{quotation}
\noindent Nombrábamos comisiones para atender. Entonces se recibían los pabellones de todas las fábricas de aquí, en el municipio, para que vinieran a traer nuestro pabellón, para poder hacer el recorrido, para que netamente se hiciera después la misa [...] Entonces precisamente nombrábamos comisiones en una asamblea para que se identificaran, no todos metían mano ahí, la comisión cuidaba a todos los compañeros que llegaban. Quiera o no, teníamos que convivir con los de las demás factorías para ver si necesitaban esto o lo otro (Polo, comunicación personal, \oldstylenums{27 de agosto de 1989}).
\end{quotation}

\noindent Para el lunes, después de la fiesta, Isabel Lima mencionaba que la comisión de organización tenía todavía muchas tareas que cumplir, como era regresar a la iglesia lo que les habían prestado:

\begin{quotation}
\noindent Entregar los ornamentos de los sacerdotes, el pie de altar; subir al Sagrado Corazón, porque tenía su nicho grande. El Sagrado Corazón es el que está ahora en el convento. Se ponían asientos para las familias el día de la festividad. El salón era largo, y venía el señor Obispo a celebrar. Desde que se hizo Obispo lo invitamos, le gustaba y venía a veces hasta a comer, pero ahora está enfermo (Isabel Lima, comunicación personal, \oldstylenums{1989}).
\end{quotation}

\subsection*{\mdseries\large\textsc{2.4.3. Las particularidades de la fiesta}}
\addcontentsline{toc}{subsection}{2.4.3. Las particularidades de la fiesta}

\noindent Como parte de la fiesta patronal, hacían algo muy importante, una misa de difuntos para los compañeros fallecidos.

\begin{quote}
Se hacía misa de tres ministros, ésta se hacía dentro del salón de la fábrica y por este lado estaban los telares y de este lado estaba la preparación. Al día siguiente, no íbamos a trabajar, íbamos a descomponer, pero primero había una misa dedicada a los difuntos, los que habían muerto y habían colaborado en la fábrica, entonces el día siguiente era dedicado a los difuntos... le llamábamos <<La Pegua>>, se les hacía una misa de réquiem, el día martes ya nos presentábamos a trabajar (Polo, comunicación personal, \oldstylenums{27 de agosto de 1989}).
\end{quote}

\subsection*{\mdseries\large\textsc{2.4.4. Sistema de cargos religiosos dentro de la fábrica}}
\addcontentsline{toc}{subsection}{2.4.4. Sistema de cargos religiosos dentro de la fábrica}

\noindent En ese tiempo las fábricas en Chiautempan tenían su día de fiesta para celebrar al Santo Patrón. En La Estrella el sistema de cargos no existía como tal. Inicialmente la Fiesta del Sagrado Corazón se hizo debido a la forma particular de pensar de los dueños de La Estrella, los señores Secada, que eran católicos, y las hijas del dueño, que le tenían mucha devoción al Sagrado Corazón.

En esta fiesta se invitaba a los obreros y a sus familias, quienes asistían la mayoría de veces. Era una conmemoración en la que podían convivir los dueños, los obreros y sus familias. De hecho, en las memorias de algunas personas todavía están los recuerdos de la fiesta de La
Estrella. Recuerdan con gusto, dicen que era como la fiesta del pueblo. De acuerdo con lo anterior, es notable que la fiesta se haya realizado de diferentes maneras, pero ya en los años en que era el sindicato el que dirigía la celebración:

\begin{quotation}
\noindent Ellos eran quienes tomaban la batuta y decían: <<tú te encargas del desayuno de la comida, de los \textit{cuetes}; tú te encargas del altar, de las flores; tú te encargas de invitar al sacerdote>>, porque el sacerdote que oficiaba las misas era el principal párroco. Nunca se le confió a cualquier sacerdote, era el principal cura en Chiautempan el que hacía las fiestas del Sagrado Corazón de Jesús y, además, tenía que combinarla con las otras fiestas del Sagrado Corazón de Jesús. Y siempre fue así y era un pueblo contento (Rogelio Flores, comunicación personal, \oldstylenums{30 de octubre de 2010}).
\end{quotation}

\noindent Los hijos de los obreros recuerdan también, con alegría y orgullo, la presencia de su padre para enseñar a sus hijos cómo era su trabajo. Porque además era muy emocionante cuando, en algunas de las fiestas, ponían a trabajar la maquinaria para regocijo de los visitantes. Cabe aclarar que los trabajadores se encargaban de organizar todas las actividades de la celebración, pero económicamente no invertían dinero.

\begin{center}
\dag
\end{center}

\chapter{\mdseries La desaparición de la fábrica y la continuidad de la fiesta}\label{Capitulo_3}
\pagestyle{fancy}
\fancyhf{}
\fancyhead[RO,LE]{\hfill \textit{Capítulo 3. La desaparición de la fábrica y la continuidad de la fiesta} \hfill}
\fancyfoot[RO,LE]{\hfill \thepage \hfill}
% \markboth{Capítulo 3. La desaparición de la fábrica y la continuidad de la fiesta}{Capítulo 3. La desaparición de la fábrica y la continuidad de la fiesta}
\section*{\mdseries\large\textsc{3.1. Las fiestas de \oldstylenums{1989, 2014 y 2015}}}
\addcontentsline{toc}{section}{3.1. Las fiestas de 1989, 2014 y 2015}

\noindent Una vez que los patrones de La Estrella no llegaron a un acuerdo con los trabajadores sobre sus salarios, finalmente el \oldstylenums{17 de noviembre de 1970} estalló la huelga y los trabajadores permanecieron por más de dos años cuidando y haciendo guardias, resguardando el inmueble mientras hacían las negociaciones para la liquidación, que se dio hasta marzo de \oldstylenums{1973}:

\begin{quotation}
\noindent Seria crisis económica vienen confrontando desde hace algún tiempo los \oldstylenums{144} obreros de la fábrica de hilados y tejidos La Estrella (una de las factorías más antiguas de Santa Ana se teme que desaparezca precisamente por su maquinaria que no ha sido modernizada), de forma que el dueño de forma inexplicable les viene pagando tan sólo el \oldstylenums{50}\% de sus salarios y temen esta irregularidad persista y se agudice su situación, por lo que piden sus dirigentes una intervención efectiva a efecto de que se normalice el cierre (\textit{El Sol de Tlaxcala}, \oldstylenums{5 de noviembre de 1970}).
\end{quotation}

\noindent Cresencio Cuautle, mayordomo en \oldstylenums{2015}, señala que desconoce si en el tiempo de la huelga se hacía la conmemoración al Sagrado Corazón. Hay que considerar que si los extrabajadores habían perdido su fuente de ingreso y que para realizar las actividades rituales se necesitaba dinero, probablemente debieron suspender las fiestas en ese tiempo (Cresencio Cuautle, comunicación personal, \oldstylenums{27 de junio de 2014}).

Posteriormente, una vez cerrada la fábrica, muchos obreros obtuvieron su jubilación, como lo establece la ley laboral, y desde entonces se ha venido realizando la festividad del Sagrado Corazón. Es frecuente encontrar en varias invitaciones de la celebración estas palabras: <<A los \oldstylenums{47} años de haber perdido nuestra fuente de trabajo, aún perdura la devoción al Sagrado Corazón de Jesús>> (\emph{véase} \oldstylenums{Imagen \ref{invitación_frente}, pág. \pageref{invitación_frente} e Imagen \ref{invitación_interior}, pág. \pageref{invitación_interior}}). A partir de entonces, las misas se han celebrado en otro sitio, porque ya no se pudo hacer dentro de las instalaciones de la fábrica, ni tampoco en sus espacios exteriores.

Pasaron aproximadamente tres años para que los extrabajadores (que entonces eran \oldstylenums{144} obreros en total, cuando la fábrica cerró, y muy pocos son los que no continuaron con el grupo para la festividad) se organizaran con la dirección de quien fue su líder, Nicolás López Galindo, y reanudaran la fiesta. Desafortunadamente, no hay personas ni documentos para con firmar esta información.

Después de la conmemoración del Santo Patrón, se hacían reuniones para elegir a la comisión del siguiente año, como antes se hacía, coordinadas todavía por el dirigente sindical. Cabe mencionar que la sesión de \oldstylenums{1987} fue fundamental para la permanencia de la fiesta, porque ahí se propuso elegir a Alejandro Benítez para la comisión de \oldstylenums{1988}.

Isabel Lima expone cómo eran las vivencias de los extrabajadores de La Estrella de \oldstylenums{1987}:

\begin{quotation}
\noindent Don Nicolás López le decía a Don Guadalupe: <<Mira diputado>> ---le decía diputado de apodo, así le decía siempre: el \textit{Diputadito}. Entonces le dijo: <<Tú eres el primero, tu hijo el segundo y aquí Don Isabel que sea el tercero>>. Y por eso yo le entré como el tercero. Yo no había pensado serlo, pero él me propuso y yo tuve que aceptarlo. Siempre he sido amante de esas cosas. A mí solamente me tocó dar mi cuota, es así cuando le entramos con el primero y de ahí... él sabe, le sobre o le falte, él sabe, nosotros nunca le pedimos cuentas. Le faltó o le sobró; si le faltó, \textit{pos} nos vemos en la obligación de pedir cuentas. Y si sobró, ahí quedamos. Como el principal es el que tiene la obligación de llamar a los tres primeros, de manera que [...] El primero es el mayordomo, el segundo el diputado y el tercero es colaborador, como todos, todos los demás son colaboradores (Isabel Lima, comunicación personal, \oldstylenums{27 de agosto de 1989}).
\end{quotation}

\noindent Cuando la fábrica estaba activa, todo era muy fácil para la realización de la festividad del Sagrado Corazón de Jesús. Había acuerdo entre las fuerzas de La Estrella. En ese entonces se unían los dueños de la empresa con los obreros y su dirigente sindical, todos con una misma intención: la fiesta. Los obreros tenían reuniones con antelación, regidos por los directivos del sindicato. En una asamblea se nombraban comisiones exclusivamente para la planeación y cumplir cada una de las responsabilidades. La celebración se hacía el domingo siguiente del viernes del Sagrado Corazón, seguramente por cuestiones laborales, pero después del cierre, la festividad comenzó a realizarse el mismo viernes, como marca el calendario.

Todavía bajo la dirigencia de su líder sindical y con el acuerdo de los extrabajadores, la imagen a la que habían festejado y venerado por tantos años en su centro de trabajo había sido donada al templo de Nuestra Señora de la Asunción. Con el problema del largo litigio, la
imagen tuvo ahí un lugar seguro y favorable.

Al permanecer en un templo, la imagen ya tenía determinado el lugar para la celebración de la misa y la iglesia. Ya el convite se hacía en casa de alguno de los extrabajadores. Generalmente el festejo se hacía en la casa del principal de la comisión de la fiesta, no había un lugar fijo porque la comisión iba cambiando año con año. Exactamente, no se podría decir cuántos extrabajadores había o cuántos participaban, probablemente lo sabía López Galindo, pero ya falleció.

\subsection*{\mdseries\large\textsc{3.1.1. Fiesta \oldstylenums{1989}}}\label{Fiesta_1989}
\addcontentsline{toc}{subsection}{3.1.1. Fiesta 1989}

\noindent Un viernes de junio de \oldstylenums{1989}, aproximadamente a las \oldstylenums{13:00} horas, llegaron en procesión los jubilados de la fábrica La Estrella para la celebración dedicada a su Santo Patrón. Algunos iban con su familia, otros solos. La procesión recorrió las calles más grandes de Chiautempan con la imagen del Sagrado Corazón. El recorrido fue <<desde la casa del exobrero elegido hasta el convento del \textit{Padre Jesús de la ciudad de Santa Ana Chiautempan}>>, como dice textualmente la invitación.

Fue una procesión alegre que llevaba una imagen pequeña del Sagrado Corazón. Estuvo acompañada con banda de música, teponaztle, chirimía y muchos cohetes. Todos caminaban más o menos al ritmo de la música, hasta llegar al templo franciscano.

Reunidos en este, la misa transcurrió en calma: todas las personas rezaban y cantaban. Mientras tanto, afuera continuaba la banda de música, señores con teponaztle y chirimía, y los cohetes.

Al terminar la misa, se inició el regreso de la misma manera. Ya en la casa del mayordomo, comenzó la socialización, el reencuentro con los compadres, amigos, los camaradas. Pero como tenían la responsabilidad de continuar con la celebración para el siguiente año, los exobreros ponían poco cuidado a la comida y daban prioridad a la designación del encargado.

Antes de comer, los extrabajadores se agruparon para iniciar la \textit{asamblea} y así tomar las decisiones de quién sería el mayordomo para el siguiente año. Uno de los extrabajadores ya había convenido con el actual mayordomo en pedirlo. Aun así, debía someterse a consenso y votación.

Las propuestas se hacían y las decisiones se tomaban en grupo. Había voces fuertes, podría decirse que algunos gritaban. Otros murmuraban, dando atención al acontecimiento, justamente como si fuera una asamblea de trabajo. Se podría decir que discutían al intentar ponerse de acuerdo, era una reunión cerrada. Parecía un remanente de las asambleas en la fábrica, era una remembranza de otros tiempos, de cuando fueron obreros y tenían que tomar decisiones importantes:

\begin{quotation}
\noindent En este año había dos aspirantes a la mayordomía, pero entonces se determinó que fuera uno para este año y el otro para el año siguiente. Para la próxima festividad ya no se tiene que buscar otro mayordomo, ya que tenemos a quién recurrir: será el señor que quedó pendiente para el año siguiente. Aunque él no es jubilado de la fábrica La Estrella. A él lo invitaron porque es muy entusiasta. Todas las adaptaciones que tiene el Sagrado Corazón de Jesús, él las hizo: las cortinitas las mandó hacer, el nicho él también lo hizo y lo pintó. Él era trabajador de la fábrica Telafil, ahorita también es jubilado, ya en cosa de jubilación todos somos una sola cosa. Para la festividad del siguiente año quedó nombrada una comisión para ayudarnos, que es la de teponaztle, que forma parte Lorenzo Ramos, Ignacio Hernández y son cuatro, y ahorita no me acuerdo de los otros nombres (Isabel Lima, comunicación personal, \oldstylenums{27 agosto 1989}).
\end{quotation}

\noindent De esta forma, se daba continuidad a una fiesta patronal de una fábrica textil que había cerrado \oldstylenums{19} años atrás. La fiesta continuaba por la devoción de los extrabajadores, sin importar que ya no estuviera en función. Por otro lado, es necesario aclarar que en Chiautempan se tienen muchas mayordomías. Los mismos extrabajadores manejaban la expresión de <<mayordomo>> dirigiéndose al principal de la comisión. Pero Isabel Lima explicaba que:

\begin{quotation}
\noindent Para la festividad no había una mayordomía propiamente, era una comisión de diez o doce elementos. Teníamos nuestra primera plática y entonces acordábamos cuánto íbamos a dar. Yo al menos comencé a dar \$\oldstylenums{10\thinspace000}, para los gastos cooperábamos todos [...] De eso sacábamos el pago del padre para la misa, pagábamos el coro o la cosa de los cantores, de ahí salía para los cohetes, y también para el agasajito que se hacía. Esto no es nuevo (Isabel Lima, comunicación personal, \oldstylenums{27 agosto 1989}).
\end{quotation}

\subsection*{\mdseries\large\textsc{3.1.2. Fiesta \oldstylenums{2014}}}\label{Fiesta_2014}
\addcontentsline{toc}{subsection}{3.1.2. Fiesta 2014}

\subsubsection{\mdseries\textsc{Los preparativos para la mayordomía del Sagrado Corazón de Jesús}}

\noindent Para organizar las festividades los mayordomos se pusieron de acuerdo sobre las necesidades de la mayordomía, como la compra de cohetes, el pago de la banda de música, los arreglos florales de la iglesia y los cantores para la misa.

Desde que toman el cargo, las personas que integran la mayordomía quebrantan su vida normal, porque se van mentalizando para el día de la festividad. En estos casos resulta interesante porque se relacionan con personas de los pueblos circunvecinos, pues los extrabajadores eran originarios de los alrededores de Chiautempan.

Movidos por la fe y devoción al Sagrado Corazón de Jesús, los mayordomos y familiares cercanos se muestran felices por haber tomado el cargo y tener la imagen en su casa. Los preparativos son variables para los extrabajadores y su familia. Sin embargo, todos están de acuerdo porque buscan agradecer los favores recibidos durante su vida; es la idea principal al recibir la mayordomía. En la fiesta los mayordomos buscan, sobre todo, fortalecer la unión entre ellos y su santo protector, y agradecer que les haya bendecido a ellos y a sus ancestros en su lugar de trabajo.

Por otro lado, los mayordomos siempre quieren dar el mayor lucimiento posible a cada fiesta, respetando el tiempo y la forma. Para su realización, empiezan por invitar a otros miembros de la familia y a los amigos más cercanos. Ya establecida la fecha y los horarios de
las dos misas, mandan a hacer las invitaciones y las reparten. Cuando ya se acerca el día, las mujeres de la familia se reúnen para planear la comida y los detalles necesarios que deben cuidar.

El compromiso mayor ha sido cumplir con las obligaciones de mayordomos, es decir, con gastos directos de la celebración. Lo demás es a voluntad propia, sin reglas, por lo que cada mayordomo dispone en su casa según su situación económica y circunstancias. Todo depende de lo que se va a dar de comer, cada platillo merece hacer determinadas compras. Por ejemplo, si se deciden a hacer tamales para el desayuno, deben preparar un gran número de hojas de maíz, chiles secos, especias para el mole que va de relleno; además del maíz para la masa y el atole. Ya en los días cercanos van adelantando en la preparación de los alimentos, y también es cuando se unen en familia para rezar.

Todo esto implica inversión de mucho trabajo para la preparación de alimentos, por otro lado, el gasto económico es considerable, por lo que es la economía de cada mayordomo la que determina el tipo de comida a ofrecer. Para la preparación de los alimentos casi siempre son personas mayores, en estas circunstancias para cumplir con el compromiso resultan ser personas dependientes de la familia.

En el trabajo de esta mayordomía hay un responsable, pero todos los miembros de la familia sienten el compromiso y cooperan para la misma causa; además, disfrutan cuando se apoyan unos con otros en estos compromisos, por lo que cada uno da su mejor esfuerzo y atención. Es normal que hagan una reunión familiar antes y después de haber tomado el cargo, con esto cada integrante puede encargarse de sus tareas.

\begin{quotation}
\noindent Una mañana nos unimos las familias, yo vengo de una familia numerosa, y tengo \oldstylenums{10} hermanos, cada uno tiene su familia y tiene sus nueras también, y nos llamamos todos, y ya sabiendo qué se va a dar, alguien dice: <<yo te pongo las tortillas, yo pongo el café>>, y entre todos nos ayudamos para solventar el gasto y la preparación de la comida (Chelo, comunicación personal, \oldstylenums{27 de junio de 2014}).
\end{quotation}

\noindent El trabajo empieza desde que toman el cargo. Comienzan a comprar y guardar alimentos imperecederos que sirvan hasta llegar el día. Aproximadamente medio año antes, los mayordomos se ponen de acuerdo para apartar las dos misas. Explican al párroco los requerimientos especiales, ya que se deben nombrar a todos los extrabajadores difuntos de La Estrella. También deben arreglar otros detalles de la iglesia: los cantores, arreglos florales, etcétera. Los mayordomos siempre buscan contar con la presencia de don Alejandro Benítez, el \textit{tiaxca}, por tener conocimiento de los detalles de la celebración. Él es responsable de la continuidad de la fiesta. Estos son unos de los detalles que cuidan para tener una celebración exitosa, sin errores el día de la fiesta.

También se planean otras tareas secundarias pero que son importantes: si se reza cada día, mientras permanece la imagen en su casa, preparar el \textit{nichito}, adornar con colores rojo y blanco, propios de la celebración del Sagrado Corazón; poner un manteado, en caso necesario, para protegerse del sol o de la lluvia; conseguir o alquilar un gran número de mesas, sillas, manteles, platos, cubiertos, todo para aproximadamente \oldstylenums{300} personas o más; preparar el espacio para los invitados, cortar el césped, lavar el piso y tener todo limpio y ordenado. Son personas muy cuidadosas de los detalles.

Como en la fábrica hubo trabajadores de diferentes lugares, después del cierre la fiesta se ha realizado en distintas partes. En \oldstylenums{2014} los mayordomos fueron de Guadalupe Ixcotla y es costumbre entre ellos repartir invitaciones, por lo que desde un mes antes empezaron con esa tarea. De manera que para el \oldstylenums{27} de junio de ese año, las personas ya tenían conocimiento sobre la celebración.

Para Austreberto Ahuatzi fue una buena oportunidad para estar en familia y convivir con sus vecinos. Desde que llevaron al \textit{nichito} a su casa, se reunían cada lunes para rezar el rosario, aunque lo hicieron de manera más solemne el día anterior a la festividad:

\begin{quotation}
\noindent Se rezaba el rosario entre toda la familia. No venían los extrabajadores a rezar. Se repartieron los días, para cada hijo o hija, según fuera, pero cada lunes. Por ejemplo, cuando le tocó a mi hija la mayor, ella, su esposo y sus hijos rezaban los misterios, y al último, las oraciones dedicadas al Sagrado Corazón las rezábamos todos. Después de cada rosario teníamos un convivio, por eso se les decía con anterioridad: <<te toca tu día>>, con eso ellos eran responsables de todo. Podían traer lo que ellos quisieran, pero era según el tiempo en que estábamos. Podía ser cafecito con pan, chileatole o atolito; si hace calor teníamos helado, fruta picada, más que nada lo que cada uno puede ofertar. La noche anterior, antes del festejo, estábamos todos juntos, porque además estábamos adelantando para la comida del día siguiente. En la tarde, para el rosario, invitamos a los vecinos a rezar, porque hay niños, los llamamos a rezar el rosario (Chelo, comunicación personal, \oldstylenums{27 de junio de 2014}).
\end{quotation}

\subsubsection{\mdseries\textsc{Celebración \oldstylenums{2014}}}

\noindent La celebración del Sagrado Corazón se divide en dos partes: la primera es una misa en la mañana del viernes y se realiza por la necesidad de recordar y nombrar a un gran número de exobreros difuntos. La segunda parte es una misa alrededor de las \oldstylenums{13:00} horas, considerada de gala, ya que se conmemora al Santo Patrón de La Estrella (\emph{véase} \oldstylenums{Imagen \ref{misa_de_gala}, pág. \pageref{misa_de_gala}}).

En el día de la festividad se prepara desayuno para las personas que acompañan a la misa de la mañana, y comida para las de la misa de gala. El desayuno fue en la casa del diputado y la comida, en casa del mayordomo. Cada uno la organizó con la ayuda de sus respectivas familias.

El homenaje al Sagrado Corazón empezó desde temprano, se reunieron a las ocho de la mañana, pero esta vez en el templo de Guadalupe Ixcotla. En la misma misa se festejaba también la mayordomía del Sagrado Corazón del pueblo. Al unirse las dos mayordomías, resultó una misa con mucha gente, mariachis, banda de música, coros y rezanderas. Esto dio como resultado una festividad más solemne y animada. En el interior del templo hubo misa con cantos y mariachis; afuera, banda de música, cohetes y teponaztle, acompañado del tambor.

Al terminar la misa, llevaron en procesión al \textit{nichito} a la casa del segundo mayordomo, Austreberto Ahuatzi (hijo de un extrabajador), para disfrutar de los tradicionales tamales con atole, gelatinas de sabores y unas pechugas rellenas. Al desayuno acudieron algunos jubilados, pero sobre todo sus familiares, las rezanderas que acompañaron, invitados, vecinos, la banda de música y voluntarios que nunca faltan. Asistieron aproximadamente \oldstylenums{300} personas.

Para la misa de gala, la procesión se dirigió desde la casa del segundo mayordomo, en Guadalupe Ixcotla, al templo del convento en Chiautempan. Iban por las calles rezando y cantando, la banda de música iba tocando fuerte, todos caminando por la vía más corta, atravesando parte del centro y continuando hasta llegar al templo franciscano de Nuestra Señora de los Ángeles; llegaron a las \oldstylenums{12:47} del día. Cargaban al \textit{nichito} dos jovencitos de aproximadamente \oldstylenums{15} años. Durante la procesión fueron apoyados por una camioneta de Protección Civil, desde Ixcotla hasta la llegada del templo en Chiautempan (\emph{véase} \oldstylenums{Imagen \ref{llegada_del_nichito}, pág. \pageref{llegada_del_nichito}}).

Ya afuera de la iglesia había personas esperando. Los que llegaron con el nichito se detuvieron en la puerta principal del templo. A la una en punto el sacerdote llegó a su encuentro, hizo unas oraciones y bendijo la imagen y a las personas, rociándolos con agua bendita.

Minutos después, empezaron a entrar, casi en formación, primero el sacerdote, siguiendo las personas con la imagen y, al final, las demás personas. Ahí estaba la imagen del Sagrado Corazón de Jesús, en el lugar principal del altar mayor, con muchas flores alrededor. Todos la respetan y le tienen mucho cariño, por ser la imagen que estuvo en la fábrica por muchos años. En la misa había algunos ancianos que tenían la mirada cansada, incluso se veían lágrimas en sus ojos (\emph{véase} \oldstylenums{Imagen \ref{saliendo_de_misa_1}, pág. \pageref{saliendo_de_misa_1}}).

También estaban hijos, nietos y bisnietos de los extrabajadores de La Estrella. Todavía permanecen los recuerdos de sus ancestros, memorias de cuando estuvieron con sus padres y su familia en la misma celebración del Sagrado Corazón de Jesús. Al respecto, Rufina Benítez menciona:

\begin{quotation}
\noindent Recuerdo que cuando mi papá me llevaba nos decía: <<aquí se va a hacer la misa del Sagrado Corazón de Jesús>>. A mí me da muchísimo gusto, yo era una niña, pero sentí algo muy bonito en mi corazón, no me importaba si me daban de comer o no, me gustaba ir a la misa, tenía esa fe. Y aparte, iba a revisar la fábrica, pero eso llega después, y eso sí nos daban permiso, pero a mí me encantaba la fiestecita que hacían, era algo lindo y muy bonito. Y qué bueno que nuestros padres nos inculcaron amar a Dios, porque no importa el santo que sea, lo que importa es la fe, y el acercarse a Dios y no perder la fe, por eso desde niña me gustó (Rufina Benítez Romano, comunicación personal, \oldstylenums{24 de junio de 2014}).
\end{quotation}

\noindent Regresando a la celebración, en el interior del templo había dos personas cantando a dos voces; uno de ellos acompañaba con el órgano. Afuera estaba la banda de música que alegraba el ambiente y, como es costumbre en las festividades importantes, el teponaztle acompañado del tambor y los cohetes.

\subsubsection{\mdseries\textsc{Ceremonia de transferencia del cargo}}

\noindent Casi al final de la misa, antes de la bendición, se realizó la \textit{ceremonia de la transferencia del cargo}, de manera muy tranquila. Sin duda, fue un momento muy significativo para los mayordomos y sus respectivas familias; al mismo tiempo, se daba testimonio de la fe y la unidad del pueblo. En ese momento, los actos de los mayordomos estaban vinculados con algo sagrado y con algo material. Un momento crucial que marcó el compromiso para dar continuidad a la siguiente celebración. Si bien fue un compromiso voluntario, expresaba la aceptación de un deber para dar cumplimiento a la celebración del año siguiente.

Cerca del altar estaban los mayordomos: Rufino Ahuatzi Meneses y su sobrino, hijo de Austreberto Ahuatzi Meneses, en representación de éste. Austreberto se encontraba en el acto, pero al ver el entusiasmo de su hijo, lo dejó fungir como mayordomo para entregar el cargo. Fue algo muy significativo. Frente a ellos estaban también los mayordomos para el siguiente año: Cresencio Cuautle y su esposa, la señora Martha Secua.

Fue una ceremonia corta, que consistió en repetir unas palabras y la entrega de dos ceras grandes y gruesas que ya tenían los mayordomos de \oldstylenums{2014} para entregar a los mayordomos de \oldstylenums{2015} (\emph{véase}\oldstylenums{Imagen \ref{entrega_de_cargo}, pág. \pageref{entrega_de_cargo}}). El sacerdote dirigió la ceremonia diciéndoles las palabras que debían repetir. Fue de manera simultánea la entrega de las ceras y lo que ellos pronunciaron:\\

---Te entrego esta luz que representa la luz de Cristo ---pronunció el sacerdote.

---Yo la recibo ---respondieron los mayordomos.

---Al recibir la luz, reciben también la imagen. La imagen es un símbolo que representa a Cristo en el Sagrado Corazón de Jesús. Es un compromiso de dedicación a Cristo y a Nuestro Padre Celestial.\\

\noindent El sacerdote bendijo a los mayordomos y a los que recibían el cargo; también las ceras, como lo hizo en la entrada del templo. La ceremonia se llevó a cabo con la mayor seriedad, devoción y emoción entre los cuatro mayordomos, que aceptaron con humildad las palabras del sacerdote.

La entrega del cargo significa la transición de la responsabilidad para organizar la fiesta y darle continuidad a la tradición. <<Esta cera está bendecida por el padre y el celebrante. Con ella se cierra el compromiso para que el año siguiente se tenga que hacer la fiesta>> (Martín
Ramírez Reyes, comunicación personal, \oldstylenums{29 junio 2014}).

Después de la misa, todos se encaminaron al barrio de Ixcotla, a la casa de Rufino Ahuatzi; se colocó la imagen en un altar que estaba ya preparado, rápidamente la esposa del mayordomo acercó veladoras e incienso para la imagen. Había un altar con muchas flores, hubo más rezos y cantos dirigidos por las mujeres rezanderas (\emph{véase} \oldstylenums{Imagen \ref{llegada_del_nichito_casa_mayordomo}, pág. \pageref{llegada_del_nichito_casa_mayordomo}}). Mientras esto sucedía, en el interior la mayoría de la gente estaba sentada a la mesa.

Para dar la bienvenida a la imagen, en la sala estaban presentes Rufino Ahuatzi (primer mayordomo y dueño de la casa) y su esposa; también estaba Austreberto, su esposa e hijo, así como los mayordomos que recibieron el cargo para el próximo año: Cresencio Cuautle, su esposa, y Alejandro Benítez, el \textit{tiaxca} de esta celebración; además de algunos familiares cercanos e invitados.

Fueron minutos de mucha emoción. Rufino, con tristeza, recordaba a su papá difunto y agradecía a las señoras que rezaron y cantaron en los rosarios de cada semana, durante el medio año que permaneció la imagen en su casa. Más que agradecimiento (\emph{véase} \oldstylenums{Imagen \ref{agradecimientos}, pág. \pageref{agradecimientos}}), había una plegaria en el tono de sus palabras. Decía:\\

---Esperemos que no se termine esta hermandad que nuestros antepasados dejaron fundada y seguiremos nosotros, siempre que Dios nos dé licencia. Ojalá que nos vayamos apoyando tanto uno como otro y a los exobreros que Dios nuestro señor los bendiga.

---La mayor parte ya no están, Don Rufino ---respondió Alejandro---. Que Dios Nuestro Señor les bendiga, y a toda la humanidad. Por ahora, muchísimas gracias. Sí muchas gracias a todos. Con esta unión, que nada más es cada año, que tenemos esa dicha de vernos unos a otros. Y qué tristeza porque ya no vemos a todos.

\noindent ---No. Ya no. Ya van desapareciendo ---afirmó Rufino.

\noindent ---Cada día somos menos ---dijo, preocupado, Alejandro---. Sólo nos queda agradecer a todos su buena voluntad. \\

\noindent Todos, contentos, aplaudieron y acordaron la fecha para el rosario de coronación: el \oldstylenums{1 de julio a las 7:00} de la noche. Mientras tanto, afuera ya estaban sirviendo la comida, para lo cual contrataron meseros. Todas las mesas estaban llenas. Se sirvieron los alimentos de manera formal: las mesas estaban cubiertas con manteles de tela y se usó loza. La comida consistió en una sopa de verduras, arroz con chícharos, mixiote de pollo y agua de frutas.

Como es costumbre en la región, sirvieron a todos los que se acercaron a la casa, aun sin ser formalmente invitados. Algo muy interesante es que muchas de las personas que estaban en la mesa comiendo no habían ido a la misa. Había muchas personas con niños, en familia, jóvenes, ancianos.

Los dos mayordomos de la celebración del Sagrado Corazón de Jesús son hijos de Domingo Ahuatzi Pinillo, ya fallecido, quien trabajó en La Estrella por \oldstylenums{40} años y obtuvo una pensión por su trabajo. En la misma fábrica trabajaron tres de sus hijos.

Rufino, el primer mayordomo, nació en \oldstylenums{1936}. A los \oldstylenums{14} años ya trabajaba en la fábrica como aprendiz, pero no tuvo oportunidad de hacer carrera, sólo estuvo unas temporadas cortas como eventual. Está casado, tiene varios hijos mayores, dos en el extranjero. Sobre los gastos de la festividad, dijo que no altera su manera de vivir, que tienen un año para prepararse. Además, cuenta con una pensión y sus hijos le apoyaron para afrontar los gastos.

El segundo mayordomo, también conocido como \textit{Diputado}, es Austreberto Ahuatzi Meneses, hermano menor de Rufino Ahuatzi e hijo también de Domingo Ahuatzi. Al igual que su familia, trabajó en La Estrella por cinco años, empezando a los \oldstylenums{15} años de edad. Para esta conmemoración, su esposa, Nicolasa Hilda Cuautle Romano, ayudó en la organización de los rosarios y el desayuno.

\subsection*{\mdseries\large\textsc{3.1.3. Fiesta \oldstylenums{2015}}}\label{Fiesta_2015}
\addcontentsline{toc}{subsection}{3.1.3. Fiesta 2015}

\noindent El \oldstylenums{12 de junio de 2015, a las 6:55} de la mañana, fueron llegando las personas al templo de Nuestra Señora de los Ángeles para la celebración de la misa en honor a los difuntos de los extrabajadores de La Estrella, en el día del Sagrado Corazón de Jesús. Unos minutos antes habían salido de la casa del mayordomo, Cresencio Cuautle, acompañado de su esposa, Martha Secua. Ellos y su familia viven a pocas cuadras del templo, en el barrio de Texcacoac, pasando la vía del ferrocarril.

Para poder llegar al templo, fueron caminando en procesión por las calles del centro de Chiautempan; llevaban al \textit{nichito}, lo cargaban dos jóvenes. Alrededor, todos podían notar la presencia del desfile por la banda de música y la gente (\emph{véase} \oldstylenums{Imagen \ref{banda_de_musica}, pág. \pageref{banda_de_musica}}). La procesión la encabezaron las personas del teponaztle y el tambor. Le seguían los que cargaban la imagen, rodeados de la familia de los mayordomos. Había también unas señoras rezando y cantando alabanzas dirigidas al Sagrado Corazón de Jesús. Casi al final iba la banda de música tocando. Alejandro Benítez se mantenía siempre cerca de la imagen.

En el templo había aproximadamente \oldstylenums{50} personas esperando para la celebración de la misa, parecían ser familiares de los obreros fallecidos. Las personas de la procesión que acompañaban al nichito esperaron unos minutos en la puerta principal del convento para que llegara el sacerdote y los recibiera (\emph{véase} \oldstylenums{Imagen \ref{mayordomos_esperando_frente_templo}, pág. \pageref{mayordomos_esperando_frente_templo}}).

Adelante estaba una persona sosteniendo el estandarte de los extrabajadores de La Estrella. Parecía que todos hacían guardia cerca del \textit{nichito}: estaba a la izquierda el mayordomo, Cresencio Cuautle, y su esposa a la derecha, Martha Secua; muy cerca de ellos estaban también sus hijos y nietos.

Al llegar el sacerdote, hizo unas oraciones, dio bendiciones a la imagen y a las personas que permanecían en la entrada. Se dirigieron hacia el interior del templo y colocaron al \textit{nichito} una vez más en el altar mayor, cerca de otra imagen del Sagrado Corazón de Jesús.

Cabe señalar que la imagen del Sagrado Corazón de Jesús que estuvo en la fábrica, se encuentra actualmente en el templo franciscano. Para la festividad la pasaron al altar mayor en la parte superior, con arreglos florales blancos.

Antes de empezar la misa, llegaron representantes de otro templo, pertenecientes también a asociaciones religiosas católicas. Llevaban sus \textit{varas y banderas} de mayordomos: era la Cofradía, autoridades de la parroquia de Santa Ana Chiautempan, que llevaban las varas que se usan como símbolo de mando en la iglesia católica de Chiautempan. Se presentaron rápidamente con Cresencio Cuautle, mayordomo de la celebración, y le entregaron una vara.

Ellos se colocaron en las primeras bancas del templo. Fue entonces cuando subió al altar mayor Cresencio Cuautle, quien habló con el sacerdote entre uno y dos minutos; regresó y tomó su lugar junto a los otros mayordomos, enseguida inició la misa. El sacerdote señaló que la intención de la misa era agradecer al Sagrado Corazón en su festividad, y después nombró a muchos extrabajadores de La Estrella ya fallecidos.

Alejandro Benítez mencionaba: <<tenemos una lista como de \oldstylenums{65} personas fallecidas en la que siempre se le hace presente al padre y por eso se le pide a tiempo, es decir, con anticipación [...] Es una misa exclusivísima para el Sagrado Corazón y los fieles difuntos de la empresa>> (Alejandro Benítez, comunicación personal, \oldstylenums{8 de febrero de 2015}).

La ceremonia era amenizada por un grupo de músicos que tocaban guitarra, panderos y cantaban. Los asistentes combinaban la conmemoración al Sagrado Corazón de Jesús en su día, las oraciones a sus difuntos y el recuerdo de lo que fue en su tiempo la solemne festividad en la fábrica. Durante toda la misa permanecieron los sonidos del teponaztle, la banda de música y los cohetes. Un exobrero expresaba que el sonido de los cohetes es una manera de oración a Dios.

Eran las \oldstylenums{7:55} de la mañana cuando la misa terminó (\emph{véase} \oldstylenums{Imagen \ref{saliendo_de_misa_2}, pág. \pageref{saliendo_de_misa_2}}), y continuaban presentes el fiscal y los mayordomos que pertenecen al templo de Santa Ana, quienes estaban de visita para dar realce a la celebración (según palabras de Cresencio). Estas personas fueron invitadas por el mayordomo de la festividad del Sagrado Corazón de la fábrica La Estrella. Todos ellos se dirigieron en procesión acompañando al \textit{nichito} hasta la casa del mayordomo. Se escuchaban cantos, rezos, y los sonidos del teponaztle y el tambor, seguidos por la banda de música. El señor del teponaztle iba marcando el paso, pero también caminando rápido, de esa manera cruzaron el centro apresuradamente.

Al llegar a la casa de los mayordomos, instalaron con devoción la imagen en el altar, colocándole veladoras e incienso. Hubo cantos y rezos una vez más, nuevamente dirigidos por un grupo de rezanderas. Fueron alrededor de \oldstylenums{15} minutos. Los cantos eran alusivos al Sagrado Corazón, al igual que los adornos rojos y blancos de la casa. Afuera, en un patio grande, ya estaban las mesas dispuestas con manteles, sillas y el altar. Con anterioridad, ya habían puesto una lona para protegerse del sol o de la lluvia, es frecuente que llueva en esos días. El desayuno se empezó a servir a las \oldstylenums{8:33} de la mañana para todas las personas que llegaron a la casa, aunque no hubieran asistido a misa. Estábamos aproximadamente \oldstylenums{150} personas para el desayuno y se respiraba un fuerte olor a chicharrones y carnitas, aunque en el desayuno se sirvió atole, tamales y gelatinas. Cerca de la casa había unos guardias de policía y tránsito para cuidar de la seguridad de la población.

\subsubsection{\mdseries\textsc{Para la siguiente misa de celebración al Sagrado Corazón}}

\noindent Minutos antes de las \oldstylenums{13:00} horas del mismo día, llegaron en procesión los mayordomos y un gran grupo de personas al templo, llevando el \textit{nichito}. Alrededor estaban los mayordomos visitantes con sus banderas. Esto daba una apariencia de orden y solemnidad. Acompañaban también la familia del mayordomo, personas rezando y cantando, todos los músicos. Esencialmente, era igual que por la mañana: personas rezando y cantando, la banda de música, cohetes y teponaztle.

Hubo bendiciones en la puerta principal. Se realizó la misa de gala dando atención solamente al aniversario de la celebración al Sagrado Corazón de Jesús. Después de la homilía, el sacerdote dirigió la ceremonia de la \textit{entrega del cargo}; fue una ceremonia solemne y emotiva.

Eran las \oldstylenums{14:01} horas cuando finalizó la misa y todos se dispusieron a regresar. Destacaba la presencia de los señores fiscales y mayordomos con sus varas y banderas. Un momento interesante fue cuando Cresencio Cuautle entregó su vara a otro mayordomo y decidió cargar él mismo el \textit{nichito} por la parte del frente; otro señor le ayudó. Sin problema alguno, regresaron a casa de los mayordomos, nuevamente con la vigilancia de la policía y de tránsito de Chiautempan.

Las señoras rezanderas oraron y cantaron con el tema del Sagrado Corazón para dar la bienvenida a la imagen; encendieron veladoras e incienso. Todo se realizó en poco tiempo.

Las personas ya estaban sentadas en las mesas y rápidamente se sirvió la comida: espagueti seco, ensalada de nopalitos, salsa verde y roja, carnitas, chicharrones y frijoles. Había también tortillas azules, recién hechas; así, calentitas, las ponían en la mesa. Asistieron a la
comida los mayordomos de las banderas, algunos extrabajadores y sus familiares, vecinos, amigos y personas que no estuvieron en la misa (aunque también hubo personas en la misa que no fueron a comer).

La imagen permaneció en la casa del mayordomo Cresencio por un año, hasta la festividad de junio de \oldstylenums{2016}. Cresencio afirmó que ya había tomado la mayordomía anteriormente, en \oldstylenums{2015} fue la segunda vez, según sus palabras: <<Si Dios me presta vida, a la mejor sí la vuelvo a tomar, así es mi devoción al Sagrado Corazón de Jesús, más que nada la devoción. Ahí trabajamos, ahí sufrimos, ahí hicimos muchas cosas>> (Cresencio Cuautle, comunicación personal, \oldstylenums{12 de junio de 2015}).

Enseguida de la comida, los mayordomos agradecieron a las personas que acompañaron, sobre todo a las que rezaron, y se pusieron de acuerdo para la coronación del Sagrado Corazón. Algunos se quedaron a platicar de temas variados. Entre ellos comentaban el estrés que da preparar cantidades grandes de comida, del problema que ocurre cuando se busca a personas que ayuden. Alejandro Benítez comentó al respecto: <<se invita a la gente a ayudar, y lo hacen hasta que quieren. Más que nada eso es lo que les complica cuando las familias no son muy grandes o cuando tienen que salir a alguna comisión por su trabajo>> (Alejandro Benítez, comunicación personal, \oldstylenums{12 de junio de 2015}).

En la plática se recordaban eventos anteriores, cuando tenían que preparar mucha comida y la preocupación y el estrés se hacían presentes:

\begin{quotation}
\noindent Así nos pasó a nosotros, pero igual, de alguna manera vencimos todos esos obstáculos. No nos caímos al \oldstylenums{100}\% porque nuestros hijos nos ayudaron, nos apapacharon. Y así nos dijeron a mi esposa y a mí: <<no tienen por qué molestarse, y no tienen por qué preocuparse. Dejan listas las cosas y ya nosotros nos coordinamos, realmente no hay problema>> (Alejandro Benítez, comunicación personal, \oldstylenums{12 de junio de 2015}).
\end{quotation}

\section*{\mdseries\large\textsc{3.2. Las innovaciones para la permanencia}}
\addcontentsline{toc}{section}{3.2. Las innovaciones para la permanencia}

\noindent Han sido muchas las innovaciones para la permanencia de la celebración, y comenzaron después del cierre de la fábrica. En ésta siempre se realizaba en domingo, y si el calendario marcaba un día diferente, el festejo era transferido al domingo siguiente a ese día. Después del cierre se ha venido celebrando siempre en viernes, según indica Isabel Lima:

\begin{quotation}
\noindent La festividad se hacía el día domingo, para no perder el sábado o el viernes que caía. Ahora no, ahora es el día que cae. Cae siempre en viernes. Y ese día tenemos que ir al convento para rezar. El Sagrado Corazón de Jesús era patrón de dos fábricas: de La Estrella y de La Lanera Moderna (Isabel Lima, comunicación personal, \oldstylenums{27 de agosto de 1989}).
\end{quotation}

\noindent Un subsiguiente cambio necesario fue rescatar la imagen. En el interior del inmueble había una imagen del Sagrado Corazón de Jesús, imagen muy querida por los dueños y por los trabajadores. Al cerrar la fábrica, dicha imagen fue llevada a un templo, donada al convento del Padre Jesús, en la misma Chiautempan. Pensando un poco en esos días que seguramente fueron difíciles para los trabajadores, ciertamente tomaron una buena decisión. Por lo que ahora recuerdan las atinadas palabras de Nicolás, quien era su dirigente sindical en el tiempo del cierre: <<Cuantas y tantas veces las personas quieran adorar al Sagrado Corazón de Jesús está en el convento, lo pueden hacer ahí mismo, en el templo>> (Alejandro Benítez, comunicación personal, \oldstylenums{2 de julio 2015}).

En los primeros años después del cierre, inclusive muchos de los exobreros competían por tomar el cargo. Se formó un grupo de personas, con anuencia de Nicolás López Galindo, para la organización de la fiesta. La mayoría de los extrabajadores que quisieron continuar con la celebración nombraron una comisión para hacerlo:

\begin{quotation}
\noindent Con la fábrica cerrada también había muchos extrabajadores que querían continuar con la celebración, aunque algunos de ellos no tuvieron interés, pero eran la minoría. Había entonces muchos compañeros con buena voluntad para dar continuidad a la festividad, se nombraban comisiones de entre \oldstylenums{10 y 12} personas para la festividad (Isabel Lima, comunicación personal, \oldstylenums{27 de agosto de 1989}).
\end{quotation}

\noindent Actualmente y después del cierre de la empresa, la conmemoración se ha realizado un viernes del mes de junio, según marca el calendario. Ciertamente ha sido un cambio aceptado porque después del cierre ya no había necesidad de cambiar la fiesta al domingo.

Las festividades en la fábrica eran organizadas por los dirigentes sindicales, quienes convocaban a los trabajadores para los preparativos y la organización. Había diferentes juegos afuera de la fábrica y comida en el sindicato, pero en ese tiempo: <<En la festividad de la fábrica los que podían ir a comer eran solamente los familiares de los trabajadores. Se hacía invitación a toda la familia, también a los dueños que les gustaba participar>> (Andrés López Rosas, comunicación personal, \oldstylenums{12 de junio de 2015}). Tras el cierre de La Estrella, en noviembre de \oldstylenums{1970}, los jubilados involucraron a la familia para la preparación de la comida.

Por otro lado, en \oldstylenums{1986-87} todavía eran muchos los extrabajadores que asistían con sus familias para la celebración de la misa y a la procesión por las calles. Aún vivía el que fue su representante sindical, Nicolás López Galindo. Fue él quien aprobó integrar como mayordomo a una persona que no había sido parte del gremio obrero: Alejandro Benítez, quien con el tiempo tuvo mucha importancia para la realización de la festividad.

El número de jubilados se fue haciendo menos, como es natural. Por lo mismo, ya no fue posible hacer la festividad con grupos de \oldstylenums{10 y 12} personas. Aun así, eran los extrabajadores quienes disponían cómo dar continuidad a la festividad. El número de la comisión que organizaba fue disminuyendo y continuaba según las circunstancias.

Entre \oldstylenums{1988 y 1989}, cuando planeaban la adquisición del \textit{nichito}, se organizó otra comisión con una finalidad distinta: la idea era tener una imagen para poder llevarla a su casa y traerla por las calles en procesión. Como es costumbre en Chiautempan, invitaron a un extrabajador de otra fábrica para hacer la imagen. Él les apoyó en todo lo necesario para la adquisición y decoración del \textit{nichito} (\emph{véase} \oldstylenums{Imagen \ref{replica_del_sagrado_corazón}, pág. \pageref{replica_del_sagrado_corazón}}).

Aquella comisión estaba integrada por Vicente Conde, Antonio Tornel, David Canales y Álvaro Torres. Sus nombres aparecen en la parte de atrás del \textit{nichito}. De esta manera, los extrabajadores y su familia continuaron con devoción y entusiasmo la celebración al Sagrado Corazón de Jesús. Eso fue muy positivo para la permanencia de la fiesta.

Invitaron como padrino de la imagen a Irineo Valencia, jubilado de la Telafil, otra fábrica textil. El mismo señor Valencia hizo el \textit{nichito}, de acuerdo con el gusto y las necesidades de la agrupación. De esta manera, pudieron tener lo que necesitaban para hacer la procesión por las calles el día de la celebración de la fiesta del Sagrado Corazón de Jesús.

En \oldstylenums{1989} la imagen fue bendecida en la misa. Los extrabajadores y sus familias estaban muy contentos, le llamaron el <<nichito>>, y con eso pudieron tener su procesión en las calles de Chiautempan. Martín Ramírez Reyes (antiguo chofer del administrador de la fábrica), ya fallecido, comentaba que desde entonces:

\begin{quotation}
\noindent Cada año lo llevan a la iglesia, con sus mañanitas y sus felicitaciones, con mucho gusto, casi igual que en la fábrica. Pero ahora lo llevan cargando a misa, con música, teponaztle y cohetes. Ya después de la misa, lo llevan con el mayordomo y desde entonces así se ha venido celebrando (Martín Ramírez, comunicación personal, \oldstylenums{29 de junio 2014}).
\end{quotation}

\noindent Por otro lado, había también un grupo de jubilados que dirigía la festividad anual. Con los años también disminuyó, aunque todavía contaban con la presencia de Nicolás López y David Canales. El primero era el representante sindical y el segundo extrabajador de La Estrella, quien dirigía las fiestas del Sagrado Corazón. Entonces ya estaba enfermo y cansado, fue cuando ambos hicieron un acuerdo y juntos invitaron a Alejandro Benítez para participar en la organización de las festividades. Este recuerda:

\begin{quotation}
\noindent Entonces precisamente Don Nicolás me dijo: <<pues vente acá con nosotros>>, entonces yo me quedé con Don David, porque él me decía: <<¡ay! yo ya no puedo>>. Así pasó el tiempo y el señor Canales anduvo conmigo más de cinco años antes de que falleciera. Me enseñó dónde vivían los que fueron ex trabajadores de la fábrica, me explicaba en qué departamento habían trabajado según la organización de la fábrica, podría ser en el teñido o en la preparación, él sabía bien (Alejandro Benítez, comunicación personal, \oldstylenums{23 de junio 2017}).
\end{quotation}

\noindent Otra notable innovación fue cuando la celebración de los extrabajadores cambió a mayordomía, cambio muy significativo y necesario para continuar con la festividad. Esto sucedió por la disminución del número de extrabajadores. Alejandro era más joven y tenía visión para
continuar la festividad. Era necesario dar una particularidad específica que favoreciera la celebración. Fue cuando decidieron involucrar a más personas de una manera <<legal>>.

Lo significativo fue que con ese cambio los familiares pudieran tomar el cargo, que había sido exclusivo de los extrabajadores. A partir de entonces ya no fueron responsables únicamente los exobreros de la fábrica La Estrella. Alejandro relató al respecto:

\begin{quotation}
\noindent Pasaron los años y se estaban acabando los compañeros, ya se hacían las comisiones de menos colaboradores, entonces decía Don David: <<¿qué hacemos?>>. Platicamos él y yo, y estuvimos de acuerdo en hacer una mayordomía, porque ya casi no había obreros, y se hizo lo necesario para fundar la mayordomía (Alejandro Benítez, comunicación personal, \oldstylenums{23 de junio 2017}).
\end{quotation}

\noindent Actualmente toman el cargo algunos de los muy pocos extrabajadores que están vivos y en condiciones, pero son principalmente hijos, hijas, viudas y hasta los nietos. No es posible contar con detalles exactos de las fiestas: fechas, nombres o cualquier otra información. La
agrupación tenía un cuaderno con todos los detalles de las festividades al Sagrado Corazón después del cierre de la empresa, pero desafortunadamente uno de los mayordomos lo perdió.

\section*{\mdseries\large\textsc{3.3. Los actores de la fiesta: jubilados y pensionados de la exfábrica}}
\addcontentsline{toc}{section}{3.3. Los actores de la fiesta: jubilados y pensionados de la exfábrica}

\noindent Los actores principales de la fiesta al Sagrado Corazón han sido los trabajadores. Todo el tiempo han estado presentes. Muchos años y a pesar del cierre de la empresa, estuvieron dirigidos para la celebración por el que fue su asesor permanente en la fábrica, Nicolás López
Galindo. No es posible hablar de cambios en la celebración sin mencionar a cada persona que ha participado, eso los convierte automáticamente en los actores de la fiesta.

Los exobreros y las personas ligadas a la fiesta han buscado cambios necesarios para la continuidad de la fiesta. Ha habido muchas personas que han contribuido para seguir conservando la fiesta del Santo Patrón de la fábrica La Estrella, que actualmente se encuentra sin la presencia de su administrador, sin los dueños y sin líder sindical.

Todo esto ha implicado la reorganización continua. Ya son \oldstylenums{48 años, casi 50}, después del cierre de la empresa, y muchos jubilados han fallecido. Por lo que es necesario decir que es por ellos, por el cariño a ellos y su devoción, que la fiesta continúa. Además, esta fiesta ha cumplido con la parte social que es importantísima, porque ha sido y sigue siendo \textit{un hecho social, con expresión ritual, simbólica, sagrada y profana} (Martínez \oldstylenums{2004}).

Hoy en día la fiesta prevalece por la fe, el esfuerzo, buena voluntad y bondad de Alejandro Benítez y su esposa. Ellos son muy creyentes y tienen una economía que les permite dar tiempo y atención cuando es necesario para poder continuar con la festividad. Desde la pasada celebración del Sagrado Corazón de Jesús (junio de \oldstylenums{2017}), tienen en su hogar al \textit{nichito}, porque son los mayordomos para la siguiente celebración del \oldstylenums{8 de junio de 2018}. Alejandro nos comentaba:

\begin{quotation}
\noindent Desde \oldstylenums{1988} empecé a recorrerlos, de ahí para acá han transcurrido muchos años, esos años me han servido para poder identificarlos, pero en esos años que he convivido con ellos, ya no existen muchos de ellos [...] Antes me decían: <<Yo quiero que a mí me des el cargo, ¿por qué a mí no me lo das?>> (Alejandro Benítez, comunicación personal, \oldstylenums{8 de febrero 2015}).
\end{quotation}

\noindent Después de tanta insistencia para tomar el cargo, no pasaron muchos años, cuando ya empezaron las dificultades, pues los jubilados habían disminuido. Lógicamente, las energías de los extrabajadores se habían debilitado. Años después, cuando Alejandro les invitaba, ya no eran las mismas circunstancias, por lo que la tarea se fue haciendo más difícil. Muchos de ellos decían: <<Ya no puedo, yo ya estoy solo, ya no vive mi esposa, yo no tengo quién me ayude, yo no puedo solo, mis hijos no quieren>> (Alejandro Benítez, comunicación personal, \oldstylenums{8 de febrero 2015}).

Y era verdad. Con tristeza, Alejandro comentaba que llegó el tiempo en que iban desapareciendo \oldstylenums{4 o 5} en un año, después fueron de \oldstylenums{6 a 7}, y el siguiente año, hasta \oldstylenums{10} extrabajadores sucumbieron. Por ejemplo, de los extrabajadores de la comisión para la adquisición del \textit{nichito}, el único que sobrevive es Álvaro Torres.

Por esa razón, se convirtió en \textit{mayordomía}, con eso se crearon nuevas circunstancias para la festividad del Sagrado Corazón, involucrando a parientes directos e indirectos para tomar el cargo. De manera que el principal responsable ahora es el mayordomo, el segundo el diputado, y los demás, como decía Isabel Lima, sólo son colaboradores. Pero la tónica ha venido cambiando, porque en dos de los últimos años parejas matrimoniales están tomando el cargo de la mayordomía. Así fue en \oldstylenums{2015} y así ha sido en la festividad de junio de \oldstylenums{2018}.

Por la importancia que ha tenido Alejandro Benítez en la celebración, quien ha puesto su mejor esfuerzo para dar continuidad a los festejos, es necesario expresar un poco más de él. En los últimos años, él organizó a los extrabajadores y a sus familias para seguir cumpliendo con la celebración, y aunque no trabajó en La Estrella, ha puesto su mejor esfuerzo y ha hecho todo lo necesario para dar continuidad a la celebración.

Su papá, ahora difunto, Dámaso Guadalupe Benítez Pinillo, fue obrero muy conocido y reconocido entre los extrabajadores de La Estrella de Chiautempan por su devoción a la imagen del Sagrado Corazón de Jesús. Fue un hombre muy entusiasta para la organización y participación en las fiestas, antes y después del cierre de la fábrica.

Alejandro nació en \oldstylenums{1937} y recuerda con alegría las festividades de la fábrica el día del Sagrado Corazón. Cuando era pequeño asistía a dichos festejos con sus padres y con sus siete hermanos. Por lo anterior, siente mucha satisfacción al recordar aquellas fiestas y también aquel día en que solicitó el cargo de mayordomo por primera vez, en \oldstylenums{1987}. Decía que estaba profundamente emocionado porque él realmente anhelaba tomar el cargo:

\begin{quotation}
\noindent Yo lo solicité en el año de \oldstylenums{1987}, todavía estaba el señor Don Nicolás López, que era el representante sindical, y platicando mi esposa y yo sobre tomar el cargo, ella me dijo: <<Si quieres lo hacemos entre los dos>>. Yo pensé que me iban a dejar. Después hablé con mi papá y le dije: <<Me nace de corazón, quiero agarrar este compromiso>>. Mi papá me dijo: <<No te lo van a dar, porque los muchachos (así les decía a sus compañeros) son muchos y ¿crees que a ti te lo van a dar?>>. Le dije a mi papá: <<La lucha se hace, ¡que la suerte sea la mala!>>. Le dije a mi papá: <<A nombre mío usted hable, si aceptan está bien, si no, nada se pierde>>. Me dijo: <<Está bien>>. Fuimos a la festividad, llegó la comisión, llegaron todos, fue en Ixcotla, en la casa del señor Antonio Rodríguez. Al estar ahí, dieron de comer a todos, a nosotros también (Don Alejandro y su esposa). Al terminar la comida empezaron a hablar, mi papá sabía que yo no tenía ni voz ni voto. Entonces él fue y pidió el cargo. Don Nicolás dijo: <<Sí, está bien, si tú quieres>>. Entonces mi papá dijo: <<Pido, pero no es para mí. Mi hijo, que está aquí presente, tiene esa gran alegría, tiene ese gusto, esa satisfacción, de formar parte de la comisión>>. Don Nicolás preguntó: <<¿Tú crees que tu hijo sea capaz de llevar a cabo esto? Acuérdate que él nunca ha sido obrero, acuérdate que no puede comparecer con todos, porque él no vivió la vida fabril>>. Entonces yo hablé. Le dije: <<Mi esposa y yo tenemos esa grande alegría. Mi esposa viene aquí
conmigo, y mi esposa es como el bastón, porque tengo el bastón de apoyo>>. Me dijo: <<Con más ganas, tendrás que pedirle permiso a tu mujer>>. Yo le dije: <<No se trata de permiso, se trata de que usted dé su anuencia, que usted diga ¡sí o no!>>. Unos dijeron: <<¡Él ni fue obrero!>>. Y Don Nicolás dijo: <<¡Él no fue obrero, pero sí es hijo de Don Guadalupe, por qué no lo hemos de sobrellevar!>>. Después de ponerse de acuerdo, se nombró una comisión de \oldstylenums{12}, ni uno más. Había gente suficiente para completar \oldstylenums{12}. Entonces no se visitaba a nadie, no se molestaba a nadie, no se pedía a nadie, simple y llanamente entre las \oldstylenums{12} personas sufragábamos todos los gastos. Así fue como los empecé a conocer. Terminando nosotros vino otra comisión de la misma manera. Así se hacía porque había más de \oldstylenums{100} obreros (Alejandro Benítez, comunicación personal, \oldstylenums{8 de febrero 2015}).
\end{quotation}

\noindent En los últimos años Alejandro funge como mediador entre la fiesta y los sobrevivientes, además de ser muy conocedor de los detalles de la celebración. Actúa como intermediario cuando un mayordomo no cuenta con suficiente dinero, consigue donadores, además apoya en todo lo que sea necesario. Por todos los esfuerzos que ha hecho Alejandro durante tantos años para la persistencia de la festividad del Sagrado Corazón de Jesús, en la celebración de \oldstylenums{2017}, Andrés le llamó el <<\textit{Tiaxca}>>. Minutos más tarde, explicaron lo que significa:

\begin{quotation}
\noindent Tiaxca viene del náhuatl, significa el hermano mayor, el más sabio, el que rige la fiesta. El que siempre está al frente del grupo, para lo que haga falta, el que está yendo a ver a uno y a otro, para que se vayan cambiando entre ellos mismos para poder continuar con la festividad dignamente (Andrés Marcos López, comunicación personal, \oldstylenums{23 de junio 2017}).
\end{quotation}

\noindent En pocas palabras, Alejandro Benítez se ha ganado ese título porque es prácticamente el único responsable de que la celebración del Sagrado Corazón continúe. Asumir el cargo, señaló, no fue nada fácil. Nicolás López Galindo aceptó que tomara el cargo pero puso como condición que su papá, Guadalupe, fuera el primero de la comisión, es decir, el principal, y Alejandro el segundo.

La celebración de \oldstylenums{2017} fue diferente porque, al no haber casi obreros en buena situación para aceptar el cargo, Alejandro tuvo que invitar a su hermana para que aceptara también la mayordomía, Rufina Benítez Romano. El cargo de mayordomo principal fue compartido entre ella y su hija, y como diputado fungió Andrés Marcos López.

En esta celebración los actores de la fiesta dieron un toque diferente. La misa de la mañana transcurrió normalmente, pero cuando terminó, los músicos de la banda entraron al templo con sus instrumentos a cantar Las mañanitas al Sagrado Corazón. Después, en procesión, llevaron al \textit{nichito} por las calles. Se veía todo animado por las banderitas de color rojo y blanco que repartieron los familiares de Andrés después de la misa, cuando todos se dirigían al desayuno. Andrés llevaba el estandarte de los extrabajadores de la fábrica La Estrella con una
imagen del Sagrado Corazón; se veía pesado, a veces le ayudaba su hijo a cargarlo; sus nietos llevaban el \textit{nichito}. Andrés y familia invitaron el desayuno de la Fiesta del Sagrado Corazón.

Para el desayuno hubo gelatinas transparentes con un corazón rojo en medio, tamales de salsa verde y de mole. Para beber había atole, champurrado con chocolate y arroz con leche. Se podría decir que el desayuno fue exitoso, a pesar de dificultades previas:

\begin{quotation}
\noindent En el mes de noviembre mi esposa estuvo a punto de sufrir un infarto, la vio su doctora y la mandó con el cardiólogo. Y la atendieron de emergencia, la sacaron adelante pero le pusieron unos balones o balines y le volvieron al ritmo cardiaco. Todavía faltaba para la celebración pero como se iba acercando el tiempo, mis hijas hablaron con el cardiólogo y él dijo que este tipo de presiones no son recomendables. Y eso movió a mis hijas a entrarle. Aparte de eso, sí me da gusto que mi familia es muy unida, entonces para cada compromiso nos reunimos para trabajar en él y ahí lo vamos sacando. Y hoy no fue la excepción (Andrés Marcos López, comunicación personal, \oldstylenums{23 de junio 2017}).
\end{quotation}

\noindent Andrés también explicó cuáles fueron las razones para haber tomado la mayordomía:

\begin{quotation}
\noindent Son tres motivos que tenemos: principalmente la fe, luego es la añoranza de nuestros antepasados y la otra es que es un gusto compartir con todos, porque ahora sí que atendemos a propios y a extraños. Por eso lo hacemos, es un sentimiento también que el mismo Dios dice hay que compartir con tu hermano, entonces son tres motivos, principalmente (Andrés Marcos López, comunicación personal, \oldstylenums{23 de junio 2017}).
\end{quotation}

\noindent Para Rufina, su hija y toda la familia fue agradable tomar la mayordomía. Se rezó el rosario todo junio dedicado al Sagrado Corazón, con el nichito en su casa. Ellas comentaban que el día anterior a la festividad, además del rosario, estuvieron juntos en familia haciendo los preparativos para la comida del día siguiente. Fue una comida muy laboriosa, lo más entretenido fue limpiar los nopales para preparar una ensalada.

Se organizaron bien, cada día de junio había rosario y después convivio, algo muy notable en esta región de Chiautempan, donde la mayoría de las personas están animadas y dispuestas para las fiestas. Entre los hijos de Rufina se repartieron la tarea de llevar comida para el final de cada rosario: <<Uno de mis hijos traía memelitas, otro galletas, otro pastelitos, y guardamos los tamales y atole para el día último, porque es el más solemne>> (Rufina Benítez Romano, comunicación personal, \oldstylenums{24 de junio de 2017}).

\section*{\mdseries\large\textsc{3.4. La mayordomía del Sagrado Corazón}}
\addcontentsline{toc}{section}{3.4. La mayordomía del Sagrado Corazón}

\noindent El culto al Sagrado Corazón de Jesús tuvo su origen en Francia, entre los años \oldstylenums{1673 y\,1690}. La devoción se extendió paulatinamente, y tiempo después, llegó a España, en \oldstylenums{1733}, donde se fue extendiendo a través de cofradías y congregaciones, de manera que en poco tiempo tuvo gran arraigo la fe y devoción al Sagrado Corazón.

No obstante, tuvieron que pasar muchos años hasta lograr el reconocimiento total por parte de la Iglesia misma. Fue hasta \oldstylenums{1856} cuando el Papa Pío IX instauró la fiesta al Sagrado Corazón de manera absoluta y formal para toda la Iglesia (Bernabé y otros \oldstylenums{2011}).

Actualmente, es una celebración variable en el calendario, aunque siempre se festeja en junio y en viernes: <<Los textos que se leen en la misa de la festividades al Sagrado Corazón están cargados de significado de lo que esta fiesta representa: amor, perdón, misericordia del Señor>> (Bernabé y otros \oldstylenums{2011}).

\section*{\mdseries\large\textsc{3.5. El patrocinio}}
\addcontentsline{toc}{section}{3.5. El patrocinio}

\noindent Cuando cerró la fábrica había muchos extrabajadores con buena disposición para dar continuidad a la festividad. Ya después del cierre, se nombraban comisiones de \oldstylenums{10 a 12} personas, para la realización de la festividad. En este tiempo cada persona daba su cooperación y con ella se sufragaban los gastos de la festividad. Años más tarde, cuando se adquirió la responsabilidad del cargo por mayordomía, los elegidos fueron aprendiendo la mejor manera de organizarse para la fiesta.

Actualmente, ellos tienen una dinámica bien establecida. Durante todo el año ahorran para financiar los gastos principalmente de comida, que para ellos resulta en el compromiso más importante: <<Porque para los frijolitos, uno se remedia a lo que alcance y así se hace, hasta donde alcance se da>> (Esposa de Alejandro Benítez, comunicación personal, \oldstylenums{8 de febrero 2015}):

\begin{quotation}
\noindent Para nosotros, mayordomo y diputado, los gastos de la misa, de las flores, de la música, de los cantores, del teponaztle, van por mitad. Ya para la organización y preparación de la comida, cada quién sabe qué se dispone en su casa (Austreberto Ahuatzi, comunicación personal, \oldstylenums{24 de junio de 2014}).
\end{quotation}

\noindent Cuando ya han ajustado los gastos de comida, empiezan a pensar en los gastos de la iglesia: misas, banda de música, adornos, entre otros. Aunque, claro está, esto siempre depende de la economía de cada mayordomo y de su familia. Por ejemplo, en las dos celebraciones de \oldstylenums{2014 y 2015}, los mayordomos no pidieron ayuda para los gastos de la celebración, de acuerdo con los informes de los mayordomos.

Algunos de los mayordomos hacen el esfuerzo para la comida, pero no ajustan económicamente para los gastos de música, flores y demás, entonces es labor de Alejandro Benítez, coordinador de la celebración, encontrar personas del grupo que quieran cooperar:

\begin{quotation}
\noindent Cuando la gente agarra el cargo de todo corazón, pero es sencilla, entonces ya la podemos apoyar. Se les da una cooperación de dinero, porque las flores se pagan con dinero, la música, la misa... entonces unos cooperan para las flores, otros para la música y otros para la iglesia, que son los gastos más altos (Alejandro Benítez, comunicación personal, \oldstylenums{8 de febrero 2015}).
\end{quotation}

\noindent Aun así, con los diferentes gastos, trabajo, tiempo y esfuerzo que implica recibir este cargo, los mayordomos y sus familias se manifiestan muy contentos y satisfechos. Para todos ellos la mayordomía es una manera de agradecimiento a Dios, por medio del Sagrado Corazón: <<Es para Dios, no perdemos nada. No soy rico, pero tampoco me voy a quedar pobre por hacer los gastos de la celebración>> (Rufino Ahuatzi, comunicación personal, \oldstylenums{27 de junio de 2014}).

% Invitación (Frente).

\begin{figure}
\centering
\includegraphics[width=13.5cm]{02}
\renewcommand{\figurename}{Imagen}
\caption[Invitación (Frente)]{Invitación (Frente). Foto: Blanca Irma Alejo Aguilar\thinspace\copyright}
\label{invitación_frente}
\end{figure}

% Invitación (Interior).

\begin{figure}
\centering
\includegraphics[width=14.6cm]{03}
\renewcommand{\figurename}{Imagen}
\caption[Invitación (Interior)]{Invitación (Interior). Foto: Blanca Irma Alejo Aguilar\,\copyright}
\label{invitación_interior}
\end{figure}

% Misa de gala.

\begin{figure}
\centering
\includegraphics[width=14.6cm]{04}
\renewcommand{\figurename}{Imagen}
\caption[Misa de gala]{Misa de gala. Foto: Blanca Irma Alejo Aguilar\,\copyright}
\label{misa_de_gala}
\end{figure}

% Llegada del nichito.

\begin{figure}
\centering
\includegraphics[width=12.2cm]{05}
\renewcommand{\figurename}{Imagen}
\caption[Llegada del \textit{nichito} al templo]{Llegada del \textit{nichito} al templo. Foto: Blanca Irma Alejo Aguilar\,\copyright}
\label{llegada_del_nichito}
\end{figure}

% Saliendo de misa.

\begin{figure}
\centering
\includegraphics[width=14.6cm]{06}
\renewcommand{\figurename}{Imagen}
\caption[Saliendo de misa]{Saliendo de misa. Foto: Blanca Irma Alejo Aguilar\,\copyright}
\label{saliendo_de_misa_1}
\end{figure}

% Entrega de cargo

\begin{figure}
\centering
\includegraphics[width=14.6cm]{07}
\renewcommand{\figurename}{Imagen}
\caption[Entrega de cargo]{Entrega de cargo. Foto: Blanca Irma Alejo Aguilar\,\copyright}
\label{entrega_de_cargo}
\end{figure}

% Llegada del nichito a casa del mayordomo.

\begin{figure}
\centering
\includegraphics[width=14.6cm]{08}
\renewcommand{\figurename}{Imagen}
\caption[Llegada del \textit{nichito} a casa del mayordomo]{Llegada del \textit{nichito} a casa del mayordomo. Foto: Blanca Irma Alejo Aguilar\,\copyright}
\label{llegada_del_nichito_casa_mayordomo}
\end{figure}

% Agradecimientos

\begin{figure}
\centering
\includegraphics[width=14.6cm]{09}
\renewcommand{\figurename}{Imagen}
\caption[Agradecimientos]{Agradecimientos. Foto: Blanca Irma Alejo Aguilar\,\copyright}
\label{agradecimientos}
\end{figure}

% Banda de música afuera del templo.

\begin{figure}
\centering
\includegraphics[width=14.6cm]{10}
\renewcommand{\figurename}{Imagen}
\caption[Banda de música afuera del templo]{Banda de música afuera del templo. Foto: Blanca Irma Alejo Aguilar\,\copyright}
\label{banda_de_musica}
\end{figure}

% Mayordomos esperando en la entrada del templo.

\begin{figure}
\centering
\includegraphics[width=14.6cm]{11}
\renewcommand{\figurename}{Imagen}
\caption[Mayordomos esperando en la entrada del templo]{Mayordomos esperando en la entrada del templo. Foto: Blanca Irma Alejo Aguilar\,\copyright}
\label{mayordomos_esperando_frente_templo}
\end{figure}

% Saliendo de misa.

\begin{figure}
\centering
\includegraphics[width=14.6cm]{12}
\renewcommand{\figurename}{Imagen}
\caption[Saliendo de misa]{Saliendo de misa. Foto: Blanca Irma Alejo Aguilar\,\copyright}
\label{saliendo_de_misa_2}
\end{figure}

% Replica del Sagrado Corazón.

\begin{figure}
\centering
\includegraphics[width=11.8cm]{13}
\renewcommand{\figurename}{Imagen}
\caption[Réplica del Sagrado Corazón]{Réplica del Sagrado Corazón. Foto: Blanca Irma Alejo Aguilar\,\copyright}
\label{replica_del_sagrado_corazón}
\end{figure}

%\newpage
%\pagestyle{empty}
%\null\vfill

%\chapter*{Epílogo}
%\pagestyle{fancy}
%\fancyhf{}
%\fancyhead[RO,LE]{\hfill \textit{Epílogo} \hfill}
%\fancyfoot[RO,LE]{\hfill \thepage \hfill}
%% \markboth{Conclusiones}{Conclusiones}
%\addcontentsline{toc}{chapter}{Epílogo}
%\section*{Título principal de nuestro epílogo}
%\addcontentsline{toc}{section}{Título principal de nuestro epílogo}
%\subsection*{Título de nuestra primera <<sección>>}
%\addcontentsline{toc}{subsection}{Título de nuestra primera <<sección>>}
%\subsubsection*{Título principal de nuestra primera <<subsección>>}
%\addcontentsline{toc}{subsubsection}{Título de nuestra primera <<sección>>}
%
%\noindent Aquí escribimos el contenido de nuestro epílogo.

\chapter*{\centering\mdseries\Large\textsc{Conclusiones}}
\pagestyle{fancy}
\fancyhf{}
\fancyhead[RO,LE]{\hfill \textit{Conclusiones} \hfill}
\fancyfoot[RO,LE]{\hfill \thepage \hfill}
% \markboth{Conclusiones}{Conclusiones}
\addcontentsline{toc}{chapter}{\mdseries Conclusiones}

\noindent La importancia de la Antropología yace en el estudio del ser humano, por lo que es interesante abordar fenómenos sociales relacionados con las costumbres y tradiciones de distintos grupos sociales.

En nuestro caso de estudio analizamos la trascendencia de una práctica religioso-cultural, la fiesta patronal del Sagrado Corazón de Jesús de los extrabajadores de la fábrica de hilados y tejidos de algodón La Estrella, que funcionó en la población de Santa Ana Chiautempan, Tlaxcala.

Analizamos también relatos históricos, la industrialización de Puebla y Tlaxcala, vida y costumbres de los obreros, así como ritos y costumbres que aún perviven de la mayordomía. Para ello se revisaron diversas fuentes (libros, hemeroteca, revistas, \textit{Internet}). Las entrevistas y la observación directa nos dieron los datos más significativos para interpretar el hecho de la pervivencia de una práctica religioso-cultural; cabe comentar que estas se realizaron de manera grupal aprovechando los espacios de reunión de los extrabajadores.

El uso de técnicas etnográficas favoreció el trabajo empírico; se trató que las entrevistas fueran libres \textit{para comprender e interpretar lo mejor posible el punto de vista de los informantes}. Así, los testimonios históricos son muy valiosos para el análisis de nuestro caso de estudio.

De igual forma, se debe recordar también la importancia de la reflexividad como parte del lenguaje, antes de producir conocimientos, y para que resulte una investigación seria en cuanto a conceptos y sin confusiones. Gracias a estas entrevistas fue como surgió el interés por la fábrica y lo que había más allá de su vida cotidiana y dentro de ella. Esto permitió profundizar en la parte más arraigada de los exobreros, conocer sus costumbres como personas religiosas y la devoción que los patrones transmitieron a sus empleados.

La mayor dificultad en esta investigación ha sido la falta de bibliografía relacionada con la temática, no existe trabajo académico que anteceda el tema en específico, además de las limitaciones propias de este trabajo de tesis donde un gran porcentaje del grupo de estudio ya ha fallecido.

Asimismo, al inicio de esta investigación, existían dificultades al abordar al grupo de estudio (mayormente del sexo masculino), ya que no parecía oportuno cuestionarlos por motivo de género; finalmente se logró obtener la confianza y amistad de las personas involucradas. En definitiva, y después de haberlos tratado, se siente su preocupación porque perdure la fiesta; es notorio sobre todo en Alejandro Benítez y su esposa, quienes han cuidado mucho de la celebración, aun sin haber sido trabajadores de La Estrella.

Hace muchos años la fiesta nació en grande, bajo la devoción de las hijas del dueño (y seguramente también del dueño). Se nombró entonces al Sagrado Corazón de Jesús Patrón de la fábrica, por tanto, era una celebración religiosa dentro de una empresa, que daba todo el apoyo económico. En ese entonces, el sindicato era débil e incipiente, pero con los años fue tomando fuerza. Posteriormente, la celebración fue organizada por los dirigentes del sindicato. Fue entonces cuando se convirtió en algo trascendental, por involucrar a varios segmentos de la sociedad de Chiautempan.

Después del cierre de la fábrica se dio un cambio en la estructura y organización de la fiesta que afectó las condiciones laborales y económicas de los extrabajadores; entonces la Fiesta del Sagrado Corazón estuvo a punto de desaparecer. Pero sobrevivió. Y pasó de ser una celebración que parecía una fiesta del pueblo a una fiesta más pequeña; los extrabajadores incluyeron a la familia para asegurar su continuidad.

Esto fue palpable, por ejemplo, en la preparación de los alimentos. Todas las familias participaron y se extendió como un legado para el pueblo, las fiestas eran aún grandes porque vivía la mayor parte de los exobreros.

La celebración de esta fiesta se ha venido transformando, poco a poco se ha convertido en el legado de una devoción del sector laboral de la fábrica La Estrella; donde los familiares absorbieron tareas encomendadas por los exobreros, labores que ahora ya no lo son.

Actualmente, aunque la fábrica ya no opera, se vive un compromiso muy fuerte para la celebración; los extrabajadores heredaron conocimientos, costumbres, ritos y deberes para llevar a cabo la fiesta. Es notable que cuando la persona no cuenta con los medios necesarios para la realización de esta fiesta, la comunidad le da su apoyo.

Un aspecto significativo es que cuando la fábrica cerró, fue necesario que la celebración se transformara en mayordomía para mantenerse, involucrando a toda la familia; así no hubo cambios en su organización y se incorporaron más actores (viudas, hijos, nietos). Se observa
aquí una adaptación cultural en la que se creó una estrategia para evitar que la fiesta desapareciera.

Desde los años ochenta algunos académicos suponían que las fiestas de mayordomías estaban condenadas a desaparecer, como la Fiesta del Sagrado Corazón. Desde mucho tiempo atrás algunos extrabajadores querían desistir, afortunadamente ha habido personas entusiastas que disfrutan de ella y, gracias a eso, persiste.

La población de Chiautempan es amante de conservar sus tradiciones y costumbres, basta visitar el pueblo en Viernes Santo. Otro ejemplo de esta devoción lo da Fabio Gracia, quien decía que hasta el señor obispo, ahora difunto, originario de Chiautempan, creía capaz a los mayordomos de Chiautempan de vender un terreno por quedar bien con una mayordomía, lo que, en definitiva, desaprobaba.

Hoy la fiesta continúa por varias razones, la principal es la fe en Dios. Los exobreros y sus familiares están agradecidos con la imagen del Sagrado Corazón y la exfábrica La Estrella, porque de ahí salió el dinero para el sustento diario; se sienten felices al continuar la fiesta
porque no sólo es una tradición, sino también porque les permite tener momentos de regocijo. Así lo señala la hija de un extrabajador cuando afirma que <<el tiempo y todo pertenece a Dios, y Dios quiere que seamos felices>>.

Los humanos han creado diversas formas de organización social entre las cuales destacan las mayordomías, vinculadas casi siempre con las fiestas patronales y pueden coincidir o no, con las celebraciones organizadas por los gobiernos locales (profanas). Por otra parte, existen
mayordomías que nacieron de rendir culto a un Santo Patrón ubicado en la iglesia local, y otras que pese a su carácter religioso, se originaron \textit{fuera} de ella. Este es el caso de la mayordomía del Sagrado Corazón, que tuvo su origen en una fábrica textil y no dentro del conjunto de mayordomías adscritas a la parroquia de la comunidad.

Por lo tanto, el aporte principal de esta investigación fue señalar que la fiesta del Sagrado Corazón de Jesús nació en una factoría y no dentro de las celebraciones patronales orquestadas por la iglesia del pueblo. En la actualidad opera como un legado de los exobreros a sus familiares, por lo que toda la organización está dirigida por ellos, y no por personas, organizaciones o autoridades de la iglesia local.

Después de \oldstylenums{48} años del cierre de la fábrica, los exobreros y sus familias han venido realizando la festividad del Sagrado Corazón; actualmente preparan una gran fiesta para conmemorar los \oldstylenums{50} años. Esta práctica se ha transmitido a las nuevas generaciones, y las familias cumplen la tarea de realizar este rito religioso con fe y devoción. La celebración del Sagrado Corazón de Jesús es finalmente un legado de la actividad devocional de la fábrica La Estrella.

\begin{center}
\dag
\end{center}

% Apéndices

%\cleardoublepage
%\pagestyle{fancy}
%\fancyhf{}
%\fancyhead[RO,LE]{\hfill \textit{Apéndices} \hfill}
%\fancyfoot[RO,LE]{\hfill \thepage \hfill}
%\markboth{Fuentes consultadas}{Apéndices}
%\addcontentsline{toc}{chapter}{Apéndices}
%\appendix
%\chapter{Noel Merino Hernández}
%\addcontentsline{toc}{chapter}{Noel Merino Hernández}
%Lorem lipsum
%\section*{Edmundo Hernández Amador}
%\addcontentsline{toc}{section}{Edmundo Hernández Amador}
%Lorem lipsum
%\subsection*{María de Lourdes Morales Flores}
%\addcontentsline{toc}{subsection}{María de Lourdes Morales Flores}
%Lorem lipsum
%\subsection*{Valentina Hernández Morales}
%Lorem lipsum
%\chapter{Noel Merino Hernández}
%\addcontentsline{toc}{chapter}{Noel Merino Hernández}
%Lorem lipsum
%\section*{Edmundo Hernández Amador}
%\addcontentsline{toc}{section}{Edmundo Hernández Amador}
%Lorem lipsum
%\subsection*{María de Lourdes Morales Flores}
%\addcontentsline{toc}{subsection}{María de Lourdes Morales Flores}
%Lorem lipsum
%\subsection*{Valentina Hernández Morales}
%Lorem lipsum

% Podríamos usar cualquier gestor de bibliografías como bibtex, natbib, etc. ¡Mucho desmadre!. Sólo nos interesa colocarlas al final.

\chapter*{\centering\mdseries\Large\textsc{Fuentes consultadas}}
\pagestyle{fancy}
\fancyhf{}
\fancyhead[RO,LE]{\hfill \textit{Fuentes consultadas} \hfill}
\fancyfoot[RO,LE]{\hfill \thepage \hfill}
% \markboth{Fuentes consultadas}{Fuentes consultadas}
\addcontentsline{toc}{chapter}{\mdseries Fuentes consultadas}
%
\noindent Marc Augé, 2006. \textit{Hacia una antropología de los mundos contemporáneos}. México: Gedisa. \\ 

\noindent Bernabé, Juan, Jaime Garmendia y Mikel Ornazabal. <<La devoción al Sagrado Corazón de Jesús>>. \url{http://www.gecoas.com/religion/Trabajos/sgado_corazon/solemnidad.htm} (Consultado el \oldstylenums{12 de febrero de 2019}) \\

\noindent Borobio, Dionisio. 2011. <<Familia, ritos y fiesta>>. \textit{Familia} 43, \url{https://core.ac.uk/download/pdf/50605336.pdf} (Consultado el \oldstylenums{12 de febrero de 2019}) \\

\noindent Cajero, Mateo. 2009. \textit{Historia de los Otomíes en Ixtenco}. Bristol, Inglaterra, Reino Unido. \\ % \url{http://uncomp.uwe.ac.uk/genaro/publicaciones/publicaciones_files/MateoCajero.pdf} \\

\noindent Caillois, Roger. 2006. \textit{El hombre y lo sagrado}. México: Fondo de Cultura Económica.\\

\noindent Estrada, Rosalina. 1997. \textit{Del telar a la cadena de montaje}. México: Benemérita Universidad Autónoma de Puebla.\\

\noindent Florescano, Enrique y Barbara Santana. 2016. \textit{La fiesta mexicana}. México: Fondo de Cultura Económica.\\

\noindent Gamboa, Leticia. 1985. \textit{Los empresarios de ayer}. México: Benemérita Universidad Autónoma de Puebla.\\

\noindent --- 2001. \textit{La urdimbre y la trama}. México: Fondo de Cultura Económica.\\

\noindent Guber, Rosana. 2001. \textit{La etnografía. Método, campo y reflexividad}. Bogotá: Norma.\\

\noindent Ibarra, Miguel. 2010. <<Cosmovisión, ordenación territorial y ritual en San Felipe Cuauhtenco>>. \textit{Antropología e Historia de Tlaxcala. Memorias del Foro Interno de Investigaciones del Centro-INAH Tlaxcala}. \\ % \url{https://docplayer.es/24903291-Cosmovision-ordenacion-territorial-y-ritual-en-san-felipe-cuauhtenco-miguel-angel-ibarra-garcia.html} (Consultado el \oldstylenums{12 de febrero de 2019}) \\

\noindent Instituto Nacional de Estadística y Geografía. 2010. \textit{Marco Geoestadístico Municipal \oldstylenums{2010, versión 5.0}}. \url{http://cuentame.inegi.org.mx/mapas/pdf/entidades/div_municipal/tlxmpios.pdf} (Consultado el \oldstylenums{8 de mayo de 2018}) \\

\noindent Madrigal, David. 2011. <<Sistema de cargos y cambio social: etnografía de la fiesta patronal en el barrio de San Miguelito de la ciudad de San Luis Potosí>>. \textit{Revista de El Colegio de San Luis} Núm.\thinspace 1: 133-154. \url{https://dialnet.unirioja.es/descarga/articulo/4988821.pdf} (Consultado el \oldstylenums{12 de febrero de 2019}) \\

\noindent Martínez, José. 1990. <<Fiesta, tradición e identidad local>>. \textit{Cuadernos de etnología y etnografía en Navarra 55}. \url{http://www.vianayborgia.es/bibliotecaPDFs/CUET-0055-0000-0043-0058.pdf} (Consultado el \oldstylenums{12 de febrero de 2019}) \\

\noindent Medina, Andrés. 2007. \textit{La memoria negada de la Ciudad de México: sus pueblos originarios}. México: Universidad Nacional Autónoma de México.\\

\noindent Meza, Carmen. 2008. Fábrica textil La Estrella. \textit{La Industria Textil en Tlaxcala}. Tlaxcala: Gobierno del Estado de Tlaxcala: Nueva Época.\\

\noindent Moctezuma, Sergio. 2013. <<San Francisco Tepeyanco: ambiente, cultura y agricultura>>, tesis de doctorado, Universidad Iberoamericana.\\

\noindent Moreno, Paz. 2014. \textit{De lo lejano a lo próximo. Un viaje por la antropología y sus encrucijadas}. Madrid: Editorial Universitaria Ramón Areces.\\

\noindent Ortega, Mario y Mora, Fabiola. 2014. <<Mayordomías y fiestas patronales en los pueblos originarios de Santa Ana Tlacotenco y Santiago Tzapotitlan, nahuas del Distrito Federal, México>>. \textit{Diálogo Andino, Revista de Historia, Geografía y Cultura Andina}, \url{https://www.redalyc.org/articulo.oa?id=371333936005} (Consultado el \oldstylenums{12 de febrero de 2019}) \\

\noindent Osorio, Lorena. 2014. <<El sistema de cargos en Jurica: entre la fuerza de la tradición y el proceso de urbanización>>. \textit{Nueva Antropología}, Vol. \textsc{XXVII}, Núm.\,8: 201-207. \url{https://www.redalyc.org/articulo.oa?id=15936205010} (Consultado el \oldstylenums{12 de febrero de 2019}) \\

\noindent Paz, Octavio. 2004. \textit{El laberinto de la soledad}. México: Fondo de Cultura Económica.\\

\noindent Pérez, Herón. 1998. \textit{México en la fiesta}. México: Gobierno del Estado de Michoacán.\\

\noindent Suárez, Ella. <<Estilo Chicago>>. 2015. En \textit{Manual de citas y referencias bibliográficas}, editado por Margarita Pérez, Manuel Romero, Ella Suárez y Nicolás Vaughan, 33-55. Bogotá: Universidad de los Andes. \url{https://bibliografiaycitas.unir.net/documentos/VANCOUVER_normativa_citas.pdf} \\

\noindent Pozzio, María. 2005. <<Prácticas etnográficas. Ejercicios de reflexividad de antropólogas de campo>>. \textit{Alteridades} Vol.\thinspace 25, Núm.\thinspace 50: 131-133. \url{https://www.redalyc.org/pdf/747/74743764011.pdf} (Consultado el \oldstylenums{12 de febrero de 2019}) \\

% \newpage
\noindent Santibáñez, Blanca. 2013. \textit{Industria y trabajadores textiles en Tlaxcala: Convergencias y divergencias en los movimientos sociales}. Puebla: Benemérita Universidad Autónoma de Puebla.\\

\noindent Van Gennep, Arnold. 2008. \textit{Ritos de paso}. Madrid: Alianza.\\

% Índice analítico (onomástico, orográfico, etc.)

\newpage
\pagestyle{fancy}
\fancyhf{}
\fancyhead[RO,LE]{\hfill \textit{Índice onomástico} \hfill}
\fancyfoot[RO,LE]{\hfill \thepage \hfill}
%\indexprologue{\noindent Aquí encontrarás una lista detallada de los principales nombres usados en esta tesis.}
\printindex[nombres]
\addcontentsline{toc}{chapter}{Índice onomástico}

\newpage
\pagestyle{fancy}
\fancyhf{}
\fancyhead[RO,LE]{\hfill \textit{Índice toponímico} \hfill}
\fancyfoot[RO,LE]{\hfill \thepage \hfill}
%\indexprologue{\noindent Aquí encontrarás una lista detallada de los principales lugares usados en esta tesis.}
\printindex[lugares]
\addcontentsline{toc}{chapter}{Índice toponímico}

\newpage
\pagestyle{empty}
\null\vfill

\newpage
\pagestyle{empty}
\null\vfill

\begin{center}
\begin{minipage}{5cm}
\footnotesize \textit{La fiesta del Sagrado Corazón de Jesús: un legado de la actividad devocional de la fábrica La Estrella, Tlaxcala}, de Blanca Irma Alejo Aguilar, se terminó de maquetar en Cerrada de Colima \oldstylenums{7301}, Col. Universidades, Puebla, México. La captura del texto se realizó con el editor \href{https://www.xm1math.net/texmaker/}{\hologo{TeX}maker (\textsc{4.5})} y se diagramó, finalmente, en el sistema de composición tipográfica \href{https://www.latex-project.org/}{\hologo{LaTeXe}}.  En su formación se emplearon tipos \href{https://es.wikipedia.org/wiki/Linux_Libertine}{Linux Libertine} para el texto principal y \textsf{\href{http://libertine-fonts.org/}{Linux Biolinum}} para la portada; las imágenes y fotografías se manipularon en \textsc{\href{https://www.gimp.org/}{gimp}} (\oldstylenums{2.8}). La impresión en \href{https://en.wikipedia.org/wiki/Laser_printing}{\textsc{laser printing}} estuvo bajo el cuidado del autor y se realizó sobre papel \emph{Bond} ($216 \times 279 $ mm) de 75\thinspace gr.
\end{minipage}
\end{center}
\begin{center}
\footnotesize Corrección ortográfica y estilo: \\ \href{julian2che@gmail.com}{Julián y Víctor Osorno}
\end{center}
\begin{center}
\footnotesize Diseño y maquetación: \\ \href{noel_merino@yahoo.com.mx}{Noel Merino Hernández}
\end{center}
\begin{center}
\href{https://www.xm1math.net/texmaker/}{\includegraphics[width=2.5cm]{15}}
\end{center}
\begin{center}
\huge \href{https://www.latex-project.org/}{\hologo{LaTeXe}}
\end{center}
\end{document}